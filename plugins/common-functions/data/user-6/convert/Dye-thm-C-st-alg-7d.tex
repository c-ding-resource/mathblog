\documentclass[a4paper,reqno]{amsart}
\usepackage{amsmath, amssymb, eucal, amscd, amstext, verbatim, enumerate, color, comment}

\topmargin -0.1in
\textwidth 6.25in
\textheight 8.5in
\oddsidemargin -0.2in
\evensidemargin -0.2in

\newtheorem{thm}{Theorem}[section]
\newtheorem{lem}[thm]{Lemma}
\newtheorem{eg}[thm]{Example}
\newtheorem{prop}[thm]{Proposition}
\newtheorem{cor}[thm]{Corollary}
\newtheorem{defn}[thm]{Definition}
\newtheorem{rem}[thm]{Remark}
\newtheorem{ntn}[thm]{Notation}
\newtheorem{quest}[thm]{Question}
\newtheorem{conj}[thm]{Conjecture}

\newtheorem{thm2}{Theorem}
\newtheorem{prop2}[thm2]{Proposition}
\newtheorem{conj2}[thm2]{Conjecture}
\newtheorem{defn2}[thm2]{Definition}

%\numberwithin{equation}{section}

\newcommand{\abs}[1]{{\lvert#1\rvert}}

\newcommand{\smnoind}{\smallskip\noindent}

\newcommand{\id}{{\rm id}}
\newcommand{\sa}{{\rm sa}}
\newcommand{\Ad}{{\rm Ad}}
\newcommand{\rmi}{\mathrm{i}}
\newcommand{\ti}{\tilde}
\newcommand{\la}{\langle}
\newcommand{\ra}{\rangle}
\newcommand{\mo}{\mathrm{o}}
\newcommand{\mc}{\mathrm{c}}
\newcommand{\mm}{\mathbf{m}}
\newcommand{\mq}{\mathrm{q}}


\newcommand{\Cu}{\mathcal{U}}

\newcommand{\ad}{{\rm Ad}\ \!}

\newcommand{\RP}{\mathbb{R}_+}

\newcommand{\BC}{\mathbb{C}}
\newcommand{\BZ}{\mathbb{Z}}
\newcommand{\BN}{\mathbb{N}}
\newcommand{\BR}{\mathbb{R}}
\newcommand{\BM}{\mathbb{M}}
\newcommand{\BP}{\mathbb{P}}


\newcommand{\CS}{\mathcal{S}}
\newcommand{\CA}{\mathcal{A}}
\newcommand{\CC}{\mathcal{C}}
\newcommand{\CF}{\mathcal{F}}
\newcommand{\CH}{\mathcal{H}}
\newcommand{\CK}{\mathcal{K}}
\newcommand{\CP}{\mathcal{P}}
\newcommand{\CB}{\mathcal{B}}
\newcommand{\CO}{\mathcal{O}}
\newcommand{\CQ}{\mathcal{Q}}
\newcommand{\CR}{\mathcal{R}}
\newcommand{\CL}{\mathcal{L}}
\newcommand{\CT}{\mathcal{T}}
\newcommand{\CI}{\mathcal{I}}
\newcommand{\CU}{\mathcal{U}}

\newcommand{\KL}{\mathfrak{L}}
\newcommand{\KA}{\mathfrak{A}}
\newcommand{\KI}{\mathfrak{I}}
\newcommand{\KJ}{\mathfrak{J}}
\newcommand{\KF}{\mathfrak{F}}
\newcommand{\KK}{\mathfrak{K}}
\newcommand{\KG}{\mathfrak{G}}
\newcommand{\KS}{\mathfrak{S}}
\newcommand{\KC}{\mathfrak{C}}
\newcommand{\Kt}{\mathfrak{t}}
\newcommand{\KB}{\mathfrak{B}}
\newcommand{\KN}{\mathfrak{N}}
\newcommand{\KM}{\mathfrak{M}}
\newcommand{\KH}{\mathfrak{H}}
\newcommand{\KP}{\mathfrak{P}}

\newcommand{\bs}{\mathbf{s}}
\newcommand{\bz}{\mathbf{z}}
\newcommand{\be}{\mathbf{e}}
\newcommand{\f}{\mathbf{f}}
\newcommand{\ba}{\mathbf{a}}
\newcommand{\dist}{\mathrm{d}}
\newcommand{\R}{\mathbf{R}}
\newcommand{\bt}{\mathbf{t}}
\newcommand{\hull}{\mathrm{hull}}

\newcommand{\Proj}{\mathcal{P}}
\newcommand{\Cent}{\mathcal{Z}}

\begin{document}
\title{Quantum-sets and non-commutative Gelfand theorem}


\author{Chun Ding \and Chi-Keung Ng}

\address[Chun Ding]{Mathematical Institute, Leiden University, The Netherlands.}
\email{cding@mail.nankai.edu.cn; c.ding@math.leidenuniv.nl}

\address[Chi-Keung Ng]{Chern Institute of Mathematics and LPMC, Nankai University, Tianjin 300071, China.}
\email{ckng@nankai.edu.cn; ckngmath@hotmail.com}

\keywords{$C^*$-algebras, closed projections, Dye's theorem, quantum-sets, quantum topological spaces, quantum spectra, ortholattices, orthomodular lattices, non-commutative Gelfand theorem}

\subjclass[2010]{Primary: 06C15, 54A05, 46L05, 46L30, 81P10; Secondary: 03G12, 81P05, 81P16}

\date{\today}
\maketitle

\begin{abstract}
We introduce the notion of quantum-sets, and show that the collection of all quantum subsets of a given quantum-set (respectively, hereditary quantum-set) forms a complete ortholattice (respectively, complete orthomodular lattice). 
Conversely, there is an injective correspondence from the collection of ortholattices to that of quantum-sets such that the original lattice is a sub-ortholattice of the ortholattice of quantum subsets of that quantum-set. 
Furthermore, complete atomistic ortholattices (respectively, complete atomic orthomodular lattices) are in bijective correspondence with atomic quantum-sets (respectively, atomic hereditary quantum-sets). 

On a quantum-set, we introduce the notion of \emph{quantum topology}, which is a collection of quantum subsets satisfying certain properties similar to those of ordinary closed subsets. 
%A \emph{quantum homeomorphism} is a quantum bijection preserving quantum closed subsets in both directions. 
Let $B$ be a $C^*$-algebra.
The set $\mathfrak{P}^B$ of all pure states of $B$ is a quantum-set in a canonical way.  
We also equip $\mathfrak{P}^B$ with a quantum topology which is an analogue of the Jacobson topology:  
%When $B$ is commutative, $\neq_\mathrm{o}$ coincides with the relation $\neq$. 
$\big\{\mathrm{hull}(L): L\text{ is a closed left ideal of }B\big\}$, where 
$$\mathrm{hull}(L):= \big\{\omega\in \mathfrak{P}^B: L \subseteq L_\omega\big\} \quad \text{and} \quad L_\omega:=\{x\in B: \omega(x^*x) = 0 \}.$$

Let $A$ and $B$ be  $C^*$-algebras. 
%It is clear that a Jordan $^*$-isomorphism from $B$ to $A$ induces a quantum homeomorphism from $\mathfrak{P}^A$ to $\mathfrak{P}^B$. 
We obtain the following non-commutative Gelfand theorem: if $\mathbb{M}_2$ is not a quotient $C^*$-algebra of $A$ and $\Psi: \mathfrak{P}^A \to \mathfrak{P}^B$ is a ``quantum homeomorphism'', then there is a Jordan $^*$-isomorphism $\Theta:B\to A$ with $\Psi = \Theta^*|_{\mathfrak{P}^A}$. 
If $A$ is a simple $C^*$-algebra (could be $\mathbb{M}_2$) and there is a quantum homeomorphism from $\mathfrak{P}^A$ to $\mathfrak{P}^B$, then $A$ and $B$ are either $^*$-isomorphic or $^*$-anti-isomorphic.

The mains tools is the following Dye theorem for $C^*$-algebras: a bijection from the quantum-set of non-zero closed projections of $A$ to that of $B$ that respects the quantum-set structure
is induced by a Jordan $^*$-isomorphism from $B$ to $A$, when $A$ has no 2-dimensional irreducible representation. 
\end{abstract}

\medskip

\section{Introduction}

\medskip

In the classical world, two points are either the same or distinct; but in the quantum world, one sometimes wants to consider a more restrictive form of ``distinctness''; e.g., two quantum states are thought to be ``really distinct''  if they have zero transition probability. 
We give a mathematical presentation of this by introducing the notion of ``quantum-sets''. 
We will show that the theory of quantum-sets captures almost everything about quantum logic. 
We will then introduce quantum topological spaces, and show that the natural quantum topological spaces associated with $C^*$-algebras remember the original algebras up to Jordan $^*$-isomorphisms, under a mild assumption. 

\medskip

Let us state this in more precise terms. 
We call a symmetric subrelation $\neq_\mq$ of the usual distinctness relation $\neq$ on a set $X$ a q-distinctness relation; i.e., we assume that $x\neq_\mq y$ implies $x\neq y$ and $y\neq_\mq x$. 
In this case, we say that $(X,\neq_\mq)$ is a quantum-set, and a bijection between two quantum-sets is called a ``quantum bijection'' if it preserves the q-distinctness relations in both directions. 
Note that quantum-sets in our sense is different from the varies notions of quantum set theory found in the literature; e.g., \cite{Gud,Schles,Takeuti,TK}.
It is also different from quantum sets as defined in \cite[Definition 9]{DRZ} (note that the later is a $\vee$-semilattice equipped with a relation $\bot$). 

\medskip

Several natural examples of quantum-sets will be given in Section 3. 
Observe that one may identify a quantum-set as a graph with no loop (where two elements in the set are joined by an edge if they are q-distinct). 
In this respect, ``quantum automorphism groups'' of finite quantum-sets have already been studied in the literature (see, e.g., \cite{Ban, BBC, BGS, MRV, Sch}). 


\medskip

The q-distinctness relation on $X$ induces the ``q-complement'', denoted by $^\mc$, on the collection $\CP(X)$ of all subsets of $X$. 
Clearly, $S\mapsto (S^\mc)^\mc$ is a closure operator on $\CP(X)$.
Thus,  the image $\CQ(X)\subseteq \CP(X)$ of this closure operator is a complete lattice under the ordinary conjunction and an adapted disjunction (which will be called the q-union; see \eqref{eqt:def-q-union}). 
Elements in $\CQ(X)$ are called quantum subsets of $X$. 
%In the case when the q-distinctness relation is the usual distinctness relation $\neq$, the q-complement of a subset is its ordinary complement and the q-union coincides with the ordinary union. 

\medskip

We say that the q-distinctness relation $\neq_\mq$ is ``atomic'' if all the singleton subsets are quantum subsets. 
Moreover, we say that $\neq_\mq$ is ``hereditary'' if for every element $T\in \CQ(X)$, one has 
$$\CQ(T) = \{S\in \CQ(X): S\subseteq T \},$$ 
when $T$ is equipped with the q-distinctness relation induced from $X$. 

\medskip

The following result  (which can be found in Lemma \ref{lem:lattice} and Proposition \ref{prop:atom-hered-q-distinct}) tells us that, similar to the relation between ordinary set theory and ordinary logic, quantum-set theory can be regarded as the ``set theory''  behind quantum logic.



\medskip

\begin{prop2}\label{prop:quantum-set-ortho-lat}
(a) If $(X, \neq_\mq)$ is a quantum-set, then $\CQ(X, \neq_\mq)$ is a complete ortholattice, under the orthocomplementation $^\mc$.
Moreover, $(X,\neq_\mq)$ is hereditary if and only if $\CQ(X)$ is an orthomodular  lattice

\smnoind
(b) For an ortholattice $\CL$, there is a quantum-set $\CL^\star$ with $\CL$ being a sub-ortholattice of $\CQ(\CL^\star)$. 
The assignment $\CL \mapsto \CL^\star$ is an injective correspondence from the collection of ortholattices to that of quantum-sets. 
When $\CL$ is complete,  one has $\CL = \CQ(\CL^\star)$.


\smnoind
(c) There is a canonical bijective correspondence between the collection of complete atomistic ortholattices and that of atomic quantum-sets (respectively, complete atomic orthomodular lattices and atomic hereditary quantum-sets). 
\end{prop2}


\medskip

A Hilbert geometry can be seen as a quantum-set equipped with a compatible collinearity structure (see e.g \cite[Definition 5.1]{SV07}). 
In this paper, we will consider quantum-sets equipped with ``compatible quantum closed subset structures'', and call them quantum topologies. 
More precisely, a subcollection $\CC\subseteq \CQ(X)$ is called a ``quantum topology'' if $\CC$ contains both $\emptyset$ and $X$ such that the intersection of any family of elements in $\CC$ belongs to $\CC$, and that the q-union of two q-commuting elements (see Definition \eqref{defn:q-comm}) in $\CC$ also belongs to $\CC$. 
%Actually, one can also define ``quantum topology'' on general ortholattices (and perhaps, it is better to define it this way), but we will consider the special case when this lattice is the collection of quantum subsets of a quantum-set in this article. 
%Proposition \ref{prop:quantum-set-ortho-lat} tells us that this special case is general enough, especially when the lattice is atomistic. 

\medskip

If the q-distinctness relation $\neq_\mq$ on $X$ is the usual distinctness relation $\neq$, then any two quantum subsets (actually, all subsets are quantum subsets in this case) q-commute.
Hence, quantum topologies on $(X, \neq)$ are precisely ordinary topologies on $X$. 

\medskip

Using the notion of quantum topology, we introduce ``quantum spectra'' for $C^*$-algebras as follows. 
Suppose that $B$ is a (complex) $C^*$-algebra and $\KP^B$ is the set of all pure states on $B$. 
For any $\phi,\psi \in \KP^B$, we denote $\phi\neq_\mo \psi$ if $\phi$ and $\psi$ have orthogonal support projections. 
In physics terms, this means that $\phi$ and $\psi$ has zero transition probability.  %(recall that the transition probability is the absolute value of the inner product of the corresponding cyclic vectors in the Hilbert spaces of the GNS constructions of $\phi$ and $\psi$, and this inner product is set to zero when the GNS representations of $\phi$ and $\psi$ are not unitarily equivalent). 
For any left closed ideal $L\subseteq B$, we set 
$$\hull(L):=\{\phi\in \KP^B: \phi(x^*x) = 0, \text{ for every }x\in L \}.$$
We will show in Proposition \ref{prop:quantum top} that $\hull(L)$ is a quantum subset of $(\KP^B, \neq_\mo)$, and the collection of all such quantum subsets form a quantum topology $\CC^B$ on $(\KP^B, \neq_\mo)$. 

\medskip

Notice that our notion of quantum spectra is similar to the ideas in \cite{Ake71}. 
However, unlike \cite{Ake71}, where the atomic part of the bidual of  $B$ together with the set of q-open projections are considered, our quantum spectrum can be regarded as a ``semi-classical image'' of this structures on the pure state space of $B$ (see Lemma \ref{lem:PS-min-proj}). 
On the other hand, a similar structure on the pure state space of $B$ was considered in \cite{GK}. 
We note the different between \cite{GK} and our quantum spectra that the q-distinctness relation $\neq_\mo$ was not considered in \cite{GK}. 


\medskip

%Furthermore, our notion of quantum topological spaces looks similar to ordinary topological spaces, and hopefully some topological studies can be extended to quantum topological spaces.   
%Notice also that the definition of quantum spectra mimic precisely the hull-kernel topology for spectra of $C^*$-algebras.  
We will show that the quantum spectrum for a $C^*$-algebra captures the original algebra up to a Jordan $^*$-isomorphism (under a mild assumption), which is good enough for the consideration of physical structure modeled on $C^*$-algebras. 
More precisely, we obtain the following non-commutative generalization of the Gelfand theorem for commutative $C^*$-algebras (see Theorem \ref{thm:main2} and Corollary \ref{cor:main2}). 


\medskip

\begin{thm2}\label{thm:quantum spec}
Let $A$ and $B$ be two $C^*$-algebras.
Suppose that there is a quantum bijection  $\Psi:\KP^A \to \KP^B$ satisfying 
$\CC^B = \big\{\Psi(C): C\in \CC^A \big\}$.

\smnoind
(a) If $A$ has no 2-dimensional irreducible $^*$-representation, then there is a Jordan $^*$-isomorphism $\Gamma: B\to A$ such that $\Psi(\omega) = \omega\circ \Gamma$ ($\omega\in \KP^A$). 

\smnoind
(b) 
%Suppose that $A$ has no 2-dimensional irreducible $^*$-representation. 
%If $\Psi$ also preserves the orientation as defined in \cite{AHS}, then $\Psi$ is a $^*$-isomorphism. 
%
%\smnoind
%(c)
If $A$ is primitive (including the case when $A=\BM_2$), then $A$ and $B$ are either $^*$-isomorphic or $^*$-anti-isomorphic. 
\end{thm2}

\medskip

Observe the differences between part (a) of the above and \cite[Corollary 3]{Shu82}, in the case when $A$ and $B$ are unital: we do not consider ordinary topology but quantum topology here; we do not assume $\Psi$ to preserve the transition probabilities of all pairs, but only those with zero transition probability; we do not consider the preservation of the orientations on $\KP^A$. 
%In this respects, the relation between Theorem \ref{thm:quantum spec}(b) and \cite[Corollary 3]{Shu82} is similar to the relation between Uhlhorn's theorem and Wigner's theorem. 

\medskip

The proof of the above theorem requires results concerning closed projections and q-closed projections as studied in \cite{Akemann68, Ake69, Ake71, APT73}, as well as the Dye theorem for von Neumann algebras. 
Let us first say a few words about the later. 


\medskip

A celebrated result of H.A. Dye (\cite{Dye}) states that the orthogonality structure of the set of projections $\CP_M$ of a von Neumann algebra $M$ having no type $\mathrm{I}_2$ summand remember $M$ up to Jordan $^*$-isomorphism in a strong way. 
This interesting result is related to the Mackey-Gleason problem (see \cite{BW92}), and has many applications (see, e.g., \cite{LNW16, LNW17, Miers, Yen}).
Recently, there has been extensions of this theorem to the case of $JW$-algebras (see \cite{BW93}) and $AW^*$-algebras (see \cite{Ham}). 
%Let us state Dye's theorem  clearly in the following. 

\medskip

%\begin{defn2}
%Let $\CS$ and $\CT$ be subsets of $\CP_{M}$ and $\CP_{N}$, respectively.
%A map $\Phi: \CS\to \CT$ is said to be orthogonality preserving if for any $p,q\in \CS$, the condition $pq=0$ implies $\Phi(p)\Phi(q)= 0$. 
%
%\smnoind
%(b) If the map $\Phi$ is bijective such that both $\Phi$ and $\Phi^{-1}$ are orthogonality preserving, then we say that $\Phi$ is an ortho-bijection.
%\end{defn2}


\begin{comment}
\medskip

Notice that ortho-bijections were called ``ortho-isomorphisms'' in \cite{Dye}.
We change the term to avoid its confusion with ``ortholattice isomorphism'', which will also be considered in this paper. 
Note also that ortho-bijections and ortholattice isomorphisms are basically the same thing when $\CS$ and $\CT$ are sub-ortholattices of $\CP_M$ and $\CP_N$, respectively (see Corollary \ref{cor:assoc-quantum-set}). 
However,  we will consider in this paper the case when they are not sub-ortholattices.


\end{comment}
\medskip

Let us remind the readers that a map $\Theta$ from a $C^*$-algebra $A$ to another $C^*$-algebra is a \emph{Jordan $^*$-homomorphism} (respectively, \emph{Jordan $^*$-isomorphism}) if $\Theta$ is a linear map (respectively, linear bijection) that preserves the involution and the Jordan product; i.e., $\Theta(a^*) = \Theta(a)^*$ and $\Theta(ab + ba)= \Theta(a)\Theta(b) + \Theta(b)\Theta(a)$ ($a,b\in A$). 

\medskip


\begin{thm2}\label{thm-Dye}
(Dye) Let $M$ and $N$ be two von Neumann algebras with $M$ not having a type $\mathrm{I}_2$ summand. 
Any quantum bijection from $\CP_M\setminus \{0\}$ onto $\CP_N\setminus \{0\}$ (under the q-distinctness relations induced by orthogonalities) extends uniquely to a Jordan $^*$-isomorphism from $M$ onto $N$. 
\end{thm2}

\medskip

Since a $C^*$-algebra needs not have any non-trivial projection, Dye's theorem cannot be extended to general $C^*$-algebras directly. 
As said in the abstract, in order to obtain Theorem \ref{thm:quantum spec}, we need a version of Dye's theorem relating to closed projections. 
We recall that an element $p\in \CP_{A^{**}}$ is a \emph{closed projection} of a $C^*$-algebra $A$ (where $A^{**}$ is the enveloping von Neumann algebra of $A$) if there is an increasing net $\{a_i \}_{i\in \KI}$ of positive contractive elements in $A$ such that $1-a_i$ weak-$^*$-converges to $p$ (see e.g. \cite[\S 3.11.10]{Ped79}). 
We denote by $\CC_0(A)$ the set of all closed projections of $A$. 
On $\CC_0(A) \setminus \{0\}$, we consider the q-distinctness relation defined by orthogonality. 
The following result will be established in Corollary \ref{cor:closed-proj}. 

\medskip

\begin{thm2}\label{thm:closed-proj}
Let $A$ and $B$ be $C^*$-algebras such that $\BM_2$ is not a quotient $C^*$-algebra of $A$.
For each quantum bijection $\Phi$ from $\CC_0(A)\setminus \{0\}$ onto $\CC_0(B)\setminus \{0\}$, there is a (necessarily unique) Jordan $^*$-isomorphism $\Theta: A\to B$ such that $\Phi(p) = \Theta^{**}(p)$ ($p\in \CC_0(A)$). 
\end{thm2}

\medskip

In fact, in order to prove Theorem \ref{thm:quantum spec}, we need a version of Theorem \ref{thm:closed-proj} that involves the notation of q-closed projections. 
This will be proved in  Theorem \ref{thm:main}. 

\medskip

Please note the difference between Theorem \ref{thm:main} and \cite[Corollary I.2]{Ake71}.
In \cite[Corollary I.2]{Ake71}, one starts with a $^*$-isomorphism between the atomic parts of the biduals of the $C^*$-algebras that preserves the corresponding q-closed projections. 
However, in Theorem \ref{thm:main}, we only need isomorphism for the graphs of closed projections (where two projections are joined by an edge if they are orthogonal). 



\medskip


Finally, let us declare that all vector spaces and algebras in this paper are over the complex field, unless otherwise specified.

\medskip

\section{Quantum-sets and their relation to ortholattices}

\medskip

When a set  $X$ is equipped with a symmetric subrelation $\neq_\mq$ of the usual distinctness relation $\neq$, we say that $(X,\neq_\mq)$ is a \emph{quantum-set} and $\neq_\mq$ is called a \emph{q-distinctness relation}. 
%A q-distinctness relation $\neq_\mq$ is said to be \emph{non-degenerate} if for every $x\in X$, there exists $y\in X$ with $x\neq_\mq y$. 
Needless to say, $\neq$ is itself a q-distinctness relation.
In the case when $\neq_\mq$ coincides with $\neq$, we say that the q-distinctness relation is \emph{classical}. 

\medskip

We define the \emph{q-complement} on the collection $\CP(X)$ of all subsets of $X$ as follows:
$$\begin{equation}\label{eqt:def-quantum compl}
D^\mc := \{y\in X: y\neq_\mq z,  \text{ for any }z\in D\} \qquad (D\in \CP(X)).
\end{equation}$$
Moreover, we set $D^{\mc\mc} := (D^\mc)^\mc$. 
%If $C, D\in \CP(X)$ satisfying $C\subseteq D^\mc$, then we denote $C\bot D$. 
%When $\neq_\mq$ is classical, we know that $D^\mc = X\setminus D$ and that $C\bot D$ if and only if $C\cap D = \emptyset$.  
The q-distinctness relation $\neq_\mq$ on $X$ induces a q-distinctness relation, again denoted by $\neq_\mq$, on $D$. 

\medskip

\begin{defn}
(a) If $S\in  \CP(X)$ satisfying $S = S^{\mc\mc}$, then $(S, \neq_\mq)$ is called a \emph{quantum subset} of $(X, \neq_\mq)$. 
The collection of all quantum subsets of $X$ will be denoted by $\CQ(X, \neq_\mq)$, or simply by $\CQ(X)$.

\smnoind
(b) If $(Y, \neq_\mq)$ is another quantum-set, then we say that a map $\Psi:X\to Y$ is a \emph{quantum bijection} if $\Psi$ is bijective and preserves the q-distinctness relation in both directions, i.e., 
$\Psi(\{x\}^\mc) = \{\Psi (x)\}^\mc$ $(x\in X).$
\end{defn}

\medskip


Note that $D\mapsto D^{\mc\mc}$ is a closure operator on $\CP(X)$ that sends $D\in \CP(X)$ to the smallest quantum subset of $X$ containing $D$. 

\medskip


Suppose that $C\in \CP(X)$. 
A subset $D\subseteq C$ belongs to $\CQ(C)$, when $C$ is equipped with the induced q-distinctness relation, if and only if 
$$\begin{equation}\label{eqt:quan-comp-in-quan-subset}
D=(D^\mc \cap C)^\mc \cap C.
\end{equation}$$
%If $E\in \CP(X)$ and $E\in \CQ\big((E^\mc)^\mc\big)$, then $E= (E^\mc)^\mc$. 
%Clearly, if $\neq_\mq$ is classical, then $\CQ(X) = \CP(X)$. 
There is no guarantee that $\CQ(C)\subseteq \CQ(X)$. 
Hence, it is natural to ask when we have $\CQ(T) = \{S\in \CQ(X): S\subseteq T \}$, for every quantum subset  $T\in \CQ(X)$. 

\medskip

On the other hand, it is natural to ask when every singleton subset of $X$ is a quantum subset. 


\medskip

\begin{defn}\label{defn:atomic-hered}
A q-distinctness relation $\neq_\mq$ on $X$ is said to be 

\begin{enumerate}[i)]
	\item \emph{atomic} if $\{x\}^{\mc\mc} = \{x\}$ for every $x\in X$ (i.e. all singleton sets are quantum subsets);
	
	\item \emph{hereditary} if for each $T\in \CQ(X)$, one has $\{S\in \CQ(X): S\subseteq T \} = \CQ(T)$.
	\end{enumerate}
We say that the quantum-set $(X, \neq_\mq)$ is \emph{atomic} (respectively, \emph{hereditary}) if $\neq_\mq$ is atomic (respectively, hereditary). 
\end{defn}

\medskip

Notice that $(X,\neq_\mq)$ is hereditary if and only if for every $T\in \CQ(X)$, the closure operator on $\CP(T)$ given by the q-distinctness relation on $T$ induced from $\neq_\mq$ coincides with the restriction of the closure operator on $\CP(X)$ to $\CP(T)$. 


\medskip

\begin{comment}
\begin{eg}\label{eg:a-distin-counter}
(a) Let $X_0:=\{a,b,c,d,e \}$. 
Equip $X_0$ with the q-distinctness relation $\neq_0$ given by the following collections of pairs:
$$\big\{(a,b),(a,c),(a,e),(b,a),(b,d),(b,e),(c,a),(d,b),(d,e),(e,a),(e,b),(e,d) \big\}.$$
%Clearly, $\neq_0$ is non-degenerate.
Then 
$\CQ(X_0)=\big\{\emptyset, \{a\}, \{b\}, \{e\}, \{a,d\}, \{b,e\}, \{a,b,d\}, \{a,d,e\}, \{b,c,e\}, X_0 \big\}.$
Hence, $\neq_0$ is not atomic. 
Moreover, as $\{a\}^\mc \cap \{a,d,e\} = \{e\}$ but $\{e\}^\mc \cap \{a,d,e\} = \{a,d\}$, we know that $\neq_0$ is not hereditary. %but the restriction of $\neq_0$ on $\{a,d,e\}$ is non-degenerate. 

\smnoind
(b) Let $X_1:= X_0\cup \{c_n:n\in \BN \} \cup \{d_n:n\in \BN \}$. 
We define a q-distinctness relation $\neq_1$ on $X_1$  as the following subset of $X_1\times X_1$: 
$$\begin{align*}
\big\{(c_i, c_j): i,j\in \BN\cup \{0\}; |i-j|=1 \big\} \cup \big\{(d_i, d_j): i,j\in \BN\cup \{0\}; |i-j|=1 \big\}\cup \neq_0,
\end{align*}$$
where $c_0:=c$ and $d_0:=d$. 
For any $C\in \CP(X_0)\setminus \big\{ \{c\}, \{d\} \big\}$, one has the q-complement of $C$ with respects to $\neq_0$ is the same as the q-complement of $C$ with respects to $\neq_1$. 
Moreover, the q-complements of $\{c\}$ and $\{d\}$ with respects to $\neq_1$
 are $\{a,c_1\}$ and $\{b,e,d_1\}$ respectively. 
Hence, 
$$\CQ(X_1) = \CQ(X_0) \cup \{X_1\} \cup \big\{ \{x\}: x\in X_1 \big\} \cup \big\{ \{c_i,c_i+1\}: i\in \BN\cup \{0\}  \big\} \cup 
\big\{ \{d_i,d_i+1\}: i\in \BN\cup \{0\}  \big\}. $$
Therefore, $\neq_1$ is atomic.
On the other hand, the q-distinctness relation $\neq_1$ is again not hereditary. 

\smnoind
(c) Let $Y:= \{1,2,3,4\}$ and $\neq_\mq$ is the following subset of $Y\times Y$:
$$\big\{(1,2),(1,3),(1,4),(2,1),(2,4),(3,1),(3,4),(4,1),(4,2),(4,3) \big\}.$$
%Then $\neq_\mq$ is non-degenerate.  
Since $\CQ(Y) = \big\{\emptyset, \{1\}, \{4\}, \{1,4\}, \{2,3\}, \{1,2,3\}, \{2,3,4\}, Y \big\}$, we see that $\neq_\mq$ is not atomic, but it is not hard to check that $\neq_\mq$ is hereditary.  

\smnoind
(d) Suppose that $Z$ is a set with more than one elements, and $\varphi:Z\to Z$ is a map such that $\varphi(z)\neq z$ and $\varphi(\varphi(z)) = z$ ($z\in Z$). 
For $x,y\in Z$, define $x\neq_\varphi y$ when $y =\varphi(x)$. 
Then $\neq_\varphi$ is a minimal q-distinctness relation among all the q-distinctness relations $\neq_\mq$ on $Z$ that are \emph{non-degenerate}, in the sense that for every $x\in Z$, there exists $y\in Z$ with $x\neq_\mq y$. 
Observe that $\CQ(Z) = \{\emptyset, Z \}\cup \big\{\{z\}:z\in Z \big\}$. 
Therefore, $\neq_\varphi$ is both atomic and hereditary.  
\end{eg}

\medskip

\end{comment}


\begin{defn}
Let $(\CL, \wedge, \vee)$ be a lattice with a smallest element $0$.

\smnoind
(a)  Let $\CL^\mm$ be the set of minimal elements in $\CL\setminus \{0\}$, called the \emph{atoms} of $\CL$. Then $\CL$ is said to be 
\begin{itemize}
	\item \emph{atomic} if for each $p\in \CL\setminus \{0\}$, one can find $e\in \CL^\mm$ with $e\leq p$;
	
	\item \emph{atomistic} if 
	$\bigvee \{e\in \CL^\mm: e\leq p \} \text{ exists and equals } p$, for every $p\in \CL\setminus \{0\}$. 
\end{itemize}

\smnoind
(b) Suppose that there is an operator $':\CL \to \CL$ satisfying $p = (p')'$, $0 = p \wedge p'$, and $p'\leq q'$  for every $p, q\in \CL$ with $q\leq p$, called an \emph{orthocomplementation}. 
Then $\CL$ is called an \emph{ortholattice}.
In this case, $0'$ will be denoted by $1$. 

\smnoind
(c) An ortholattice $\CL$ is called an \emph{orthomodular lattice} if for every $p,q\in \CL$ with $p\leq q$, one has $q = p \vee (q\wedge p')$ (this is called the \emph{orthomodular law}).
\end{defn}



\medskip



Some people define ``atomistic lattice'' in a way that every non-zero element is a finite join of atoms, but this definition is different from the above (unless $\CL$ is finite). 
Furthermore, as noted in \cite[p.140]{Kal83}, 
\begin{quotation}
a complete orthomodular lattice is atomistic if and only if it is atomic. 
\end{quotation}
More information on ortholattices and orthomodular lattices can be found in standard textbooks (e.g., \cite{Kal83} and \cite{Kal98}). 


\medskip

Since $D\mapsto D^{\mc\mc}$ is a closure operator on $\CP(X)$, it is well-known that $\CQ(X)$ is a complete lattice under the usual conjunction $\cap$ and the adapted disjunction $\vee$, called the \emph{q-union}, defined as follows (see \cite[p.254]{Kal83}): 
$$\begin{equation}\label{eqt:def-q-union}
S\vee T:= (S\cup T)^{\mc\mc} \qquad (S,T\in \CQ(X)). 
\end{equation}$$
%Furthermore, $\CQ(X)$ is an ortholattice under the orthocomplementation $^\mc$ (see, e.g., \cite{Kal83} and \cite{Kal98}). 

\medskip

\begin{lem}\label{lem:lattice}
Let $(X,\neq_\mq)$ and $(Y, \neq_\mq)$ be two quantum-sets. 
Let $\Phi:X\to Y$ be a bijection.  

\smnoind
(a) $(\CQ(X), \cap, \vee,\ \!^\mc )$ is a complete ortholattice. 

\smnoind
(b) The quantum-set $(X, \neq_\mq)$ is hereditary if and only if for every $S, T\in \CQ(X)$ with $S\subseteq T$ and $T\cap S^\mc = \emptyset$, one has $S = T$; which is equivalent to 
$\CQ(X)$ being an orthomodular lattice.  

\smnoind
(c) If $(X, \neq_\mq)$ is atomic,  then $\CQ(X)$ is an atomistic lattice, and one has $\CQ(X)^\mm = \big\{\{x\}: x\in X \big\}$.

\smnoind
(d) $\Phi$ is a quantum bijection if and only if it induces an ortholattice isomorphism from $\CQ(X)$ onto $\CQ(Y)$. 
\end{lem}
\begin{proof}
(a) It is easy to check that $^\mc$ is an orthocomplementation on $\CQ(X)$. 

\smnoind
(b) It is well-known that $\CQ(X)$ is an orthomodular lattice if and only if for every $S, T\in \CQ(X)$ with $S\subseteq T$ and $T\cap S^\mc = \emptyset$, one has $S = T$ (see, e.g., Theorem 2 of \cite[\S 1.3]{Kal83}). 

On the other hand, the orthomodular law can be rewritten as follows:   
$$S^\mc = T^\mc \vee (S^\mc\cap T), \ \text{for any $S,T\in \CQ(X)$ with $S\subseteq T$};$$ 
and the above equality is the same as $S = T\cap (T \cap S^\mc)^\mc$ (under the orthocomplementation $^\mc$). 
In other words, $\CQ(X)$ is orthomodular if and only if $\{S\in \CQ(X): S\subseteq T \} \subseteq \CQ(T)$ (see Relation \eqref{eqt:quan-comp-in-quan-subset}). 

Finally, suppose that $\CQ(X)$ is an orthomodular lattice. 
Thanks to the above, in order to verify that $\neq_\mq$ is hereditary, it suffices to establish the inclusion $\CQ(T)\subseteq \CQ(X)$. 
Consider $S\in \CQ(T)$. 
We have $S = (S^\mc \cap T)^\mc \cap T$  (see Relation \eqref{eqt:quan-comp-in-quan-subset}). 
Set $R:= S^{\mc\mc}$. 
The orthomodularity assumption implies that 
$S^\mc = (S^\mc \cap T)\vee \big(S^\mc \cap (S^\mc \cap T)^\mc \big).$
Hence, 
$R = (S^\mc \cap T)^\mc \cap \big(R \vee (S^\mc \cap T) \big)$. 
This gives
$$T\cap R = T\cap (S^\mc \cap T)^\mc \cap \big(R \vee (S^\mc \cap T)\big) = S\cap \big(R\vee (S^\mc \cap T) \big) = S,$$
%because $S \subseteq R \subseteq (S^\mc \cap T) \vee R$. 
and one has $S = T\cap R\in \CQ(X)$. 

\smnoind
(c) This part is easy to verify. 

\smnoind
(d) The forward implication is obvious (note that $\vee$ is defined through the q-complement). 
For the backward implication, we note that 
if $x\in X$, then $\Phi(\{x\}^\mc) = \Phi(\{x\}^{\mc\mc\mc}) = \Phi(\{x\}^{\mc\mc})^\mc \subseteq \{\Phi(x)\}^\mc$. 
This means that for any $z\in X$ with $z\neq_\mq x$, one has $\Phi(z) \neq_\mq \Phi(x)$. 
By considering $\Phi^{-1}$, we know that $\Phi$ is a quantum bijection.  
\end{proof}



\begin{comment}
\medskip
Notice that if one also has 
$$\big(\{x\}\cup S\big)^{\mc\mc} = \{x\}\cup S \qquad (x\in X, S\in \CQ(X)),$$ 
then $(\CP(X),\cap, \cup,\ \!^\mc)$ will be a Hilbert lattice, and thus $(\CQ(X), \cap, \vee,\ \!^\mc)$ will be a propositional system (see, e.g., \cite[Theorem 5.16]{SV07}). 
However, this requirement does not holds in general. 
\end{comment}

\medskip

Note that in part (d) above, we do not assume any of the q-distinctness relations to be atomic. 


%\medskip

%We warn the readers that although we have $C^\mc \vee D^\mc = \big((C^\mc)^\mc \cap (D^\mc)^\mc\big)^\mc$ for any $C,D\subseteq X$, they may not coincide with $(C\cap D)^\mc$ unless both $C$ and $D$ are quantum subsets. 



\medskip


\begin{eg}%\label{eg:q-dist}
Let $\KH$ be a Hilbert space. 
For any $\xi,\eta\in \KH^\star := \KH\setminus \{0\}$, we set $\xi \neq_\mo \eta$ if $\xi$ is orthogonal to $\eta$. 
Then $E\in \CQ(\KH^\star, \neq_\mo)$ if and only if $E \cup \{0\}$ is a closed subspace of $\KH$.
Thus, $\neq_\mo$ is not atomic.  
On the other hand, as $\CQ(\KH^\star, \neq_\mo)$ is isomorphic to the atomic orthomodular lattice of projections in $\CB(\KH)$, we know that $\neq_\mo$ is hereditary. 
This example also tells us that the converse of Lemma \ref{lem:lattice}(c) does not hold. 
\end{eg}

\medskip

Actually, a Hilbert space is a Banach space $E$ such that $E\setminus\{0\}$ is equipped with a compatible q-distinctness relation. 


\medskip

\begin{eg}
For $u,v\in \BR$, we define 
$$u\neq_\mathrm{cd} v\quad \text{when} \quad |u-v| \geq 1.$$ 
If $u,v \in \BR$ with $u-v \geq 2$,  then $\{u,v\}^\mc = (-\infty, v-1]\cup [v+1, u-1] \cup [u+1, \infty)$ and 
$$\{u,v\}^{\mc\mc} = \{u,v\}.$$
On the other hand, if $S\subseteq \BR$ with $|x - y| < 2$ ($x,y\in S$), then  
$$\begin{equation}\label{eqt:n-points-dist>2delta}
S^\mc = (-\infty, x_0 -1] \cup [y_0+1, \infty) \quad \text{and} \quad S^{\mc\mc} = [x_0,y_0], 
\end{equation}$$
where $x_0:= \inf S$ and $y_0:=\sup S$. 
In particular, $\{u\}^{\mc\mc} = \{u\}$ and we know that $\neq_\mathrm{cd}$ is atomic. 


We denote by $\CQ_1\subseteq \CP(\BR)$ the collection of subsets that are at most countable unions of  disjoint closed intervals (whose length could be zero, finite or infinite) such that any two elements in two distinct intervals is of distances bigger than or equal to $2$. 
For any $T\in \CQ_1$, it is not hard to check that $T^{\mc\mc} = T$.
This implies that $\CQ_1\subseteq \CQ(\BR, \neq_\mathrm{cd})$. 

Suppose that $S = \bigcup_{k\in \BZ} [x_k,y_k]$ satisfying $y_k < x_{k+1}$. 
By grouping together those intervals with $x_{k+1} - y_k < 2$, we see that $S^\mc \in \CQ_1$ and hence $S^{\mc\mc}\in \CQ_1$. 

Let $D\subseteq \BR$. 
For any $n\in \BZ$, we set $D_n:= D\cap (2n, 2n+2)$. 
Observe that 
$$D^{\mc\mc} \supseteq {\bigcup}_{n\in \BZ} D_n^{\mc\mc} \cup (D\cap 2\BZ)  \supseteq D.$$ 
It follows from Relation \eqref{eqt:n-points-dist>2delta} that ${\bigcup}_{n\in \BZ} D_n^{\mc\mc} \cup (D\cap 2\BZ)$ is of the form $\bigcup_{k\in \BZ} [x_k,y_k]$ with $y_k < x_{k+1}$.
Hence, $\big({\bigcup}_{n\in \BZ} D_n^{\mc\mc} \cup (D\cap 2\BZ)\big)^{\mc\mc}$ belongs to $\CQ_1$, and thus, $D^{\mc\mc} =\big({\bigcup}_{n\in \BZ} D_n^{\mc\mc} \cup (D\cap 2\BZ)\big)^{\mc\mc}\in \CQ_1$. 
Consequently, one has $\CQ_1 = \CQ(\BR, \neq_\mathrm{cd})$. 

Consider $T=[0,1]$ and $S=[0,1/2]$. 
Then $S\subseteq T$ and $S,T\in \CQ_1$, but $S\notin \CQ(T)$ (as $S^\mc\cap T = \emptyset$). 
This means that $\neq_\mathrm{cd}$ is not hereditary, and hence, the complete atomistic ortholattice $\CQ(\BR, \neq_\mathrm{cd})$ is not orthomodular. 

Notice that if we replace $|u-v| \geq 1$ with $|u-v| \geq \delta$ for some fixed $\delta > 0$, then the resulting ortholattice will be isomorphic to the above. 
However, if we consider the q-distinctness relation induced by $|u-v| > 0$, then the resulting ortholattice is the classical Boolean algebra $\CP(\BR)$. 
\end{eg}

\medskip

\begin{comment}
Similar phenomenon as in the above also exists for ortholattices associated with those q-distinctness relations determined by $|u-v| > \delta$ ($\delta\in \RP$). 
In this case, elements in $\CQ(X)$ are unions of countable disjoint families $\{I_n\}_{n\in \BZ}$ of not necessarily closed intervals  such that $I_{n}$ is in the left hand side of $I_{n+1}$ and that the distance between $I_n$ and $I_{n+1}$ is either $\geq 2\delta$ or $>2\delta$, depending on whether the right hand end-point of $I_n$ and the left hand end-point of $I_{n+1}$ are both ``open-ended''. 
Hence, the ortholattices determined by different $\delta$ are isomorphic when $\delta > 0$, while the case when $\delta = 0$ gives the classical Boolean algebra. 
As this is a bit complicated, we will not present it here. 

\end{comment}


This example indicate that, if in some system,  two points cannot be theoretically distinguished when they are too close together, then the logic governing this system may be different from the usual one. 


\medskip



Now, we will give the close connection between ortholattices and quantum-sets. 
This result might be considered as known. 
Actually, the fact that the map $\Xi_\CL$ in part (a) is an ortholattice isomorphism for a complete ortholattice $\CL$ is a disguised form of \cite[Corollary 5.4]{Walker}, and part (b) is similar to the bijective correspondence between Hilbert lattices and Hilbert geometries (see \cite[Theorem 5.16]{SV07}). 
However, since these statements were not explicitly stated in the literature, we give some words for their arguments here. 
%Although the argument for it is standard, we nevertheless present it here for completeness. 


\medskip

\begin{prop}\label{prop:atom-hered-q-distinct}
Let $\CL$ be an ortholattice, and $\CL^\star := \CL\setminus \{0\}$. 
For $p,q\in \CL^\star$, we set $p\neq_\CL q$ if $p\leq q'$. 

\smnoind
(a) Then $\neq_\CL$ is a q-distinctness relation on $\CL^\star$, and $\CL\mapsto (\CL^\star, \neq_\CL)$ is an injective correspondence from the collection of ortholattices to that of quantum-sets. 
Moreover, the assignment 
$$p\mapsto \Xi_\CL(p):= \big\{q\in \CL^\star: q\leq p \big\}$$
is an injective ortholattice homomorphism from $\CL$ to $\CQ(\CL^\star, \neq_\CL)$. 
If, in addition, $\CL$ is complete, then the map $\Xi_\CL$ is a surjection (and in this case, $\CL$ is orthomodular if and only if $\neq_\CL$ is hereditary). 

\smnoind
(b) Suppose that $\CL$ is atomistic with $\CL^\mm\subseteq \CL^\star$ being the set of all atoms. 
The assignment
$$p\mapsto \ti \Xi_\CL(p):= \{a\in \CL^\mm: a\leq p \}$$
is an injective ortholattice homomorphism from $\CL$ to $\CQ(\CL^\mm, \neq_\CL)$. 
Moreover, $\CL \mapsto (\CL^\mm, \neq_\CL)$ produces a bijective correspondence between the collections of complete atomistic ortholattices and atomic quantum-sets (respectively, complete atomic orthomodular  lattices and atomic hereditary quantum-sets). 
\end{prop}
\begin{proof}
(a) Obviously,  $\neq_\CL$ is a  q-distinctness relation on $\CL^\star$. 
Moreover, as $\Xi_{\CL}(p')= \Xi_{\CL}(p)^\mc$, one has $\Xi_{\CL}(p) = \Xi_{\CL}(p'')  = \Xi_{\CL}(p)^{\mc\mc} \in \CQ(\CL^\star)$. 
On the other hand, since for each $S \subseteq \CL^\star$, one has 
$$S^\mc = \{q\in \CL^\star: s\leq q', \text{ for any }s\in S \},$$ 
the embedding $\Xi_{\CL}: \CL\to \CQ(\CL^\star)$ is a disguised form of the MacNeille completion of $\CL$ (see e.g., \cite{MacN} or \cite[p.255]{Kal83}). 
Hence, $\Xi_{\CL}$ is an injective ortholattice homomorphism. 

In order to verify that the assignment $\CL\mapsto (\CL^\star, \neq_\CL)$ is an injective correspondence, we suppose that $\CL_0$ is another ortholattice and there is a quantum bijection $\Phi: (\CL_0^\star, \neq_{\CL_0}) \to (\CL^\star, \neq_\CL)$. 
Let $\bar \Phi: \CQ(\CL_0^\star)\to \CQ(\CL^\star)$ be the induced ortholattice isomorphism. 
Since $\Xi_{\CL}$ is an injective ortholattice homomorphism, if the assignment $p \mapsto \Xi_{\CL}^{-1}\big(\bar\Phi(\Xi_{\CL_0}(p)) \big)$ is well-defined, then it will be an injective ortholattice homomorphism from $\CL_0$ to  $\CL$, and by symmetry, this homomorphism is bijective. 

To show $p \mapsto \Xi_{\CL}^{-1}\big(\bar\Phi(\Xi_{\CL_0}(p)) \big)$ being well-defined, we observe that 
for every $u,p\in \CL_0$, one has $u\leq p$ if and only if $\Phi(u)\leq \Phi(p')'$, because $\Phi$ is a quantum bijection. 
In other words, $\bar \Phi(\Xi_{\CL_0}(p)) = \Xi_{\CL}(\Phi(p')')$ ($p\in \CL_0$). 
This gives the required relation  
$$\{\bar \Phi (\Xi_{\CL_0}(p)): p\in \CL_0^\star \} = \{\Xi_{\CL}(q): q\in \CL^\star \}.$$

Finally, when $\CL$ is complete, the map $\Xi_{\CL}$ is an ortholattice isomorphism because $\CQ(\CL^\star)$ is the  MacNeille completion of $\CL$.
The last conclusion follows from Lemma \ref{lem:lattice}(b). 

\smnoind
(b) As in part (a), we have $\ti \Xi_\CL(p)^\mc = \ti \Xi_\CL(p')$, which gives $\ti \Xi_\CL(p)\in \CQ(\CL^\mm)$. 
It is not hard to verify that $\ti \Xi_\CL$ is an injective ortholattice homomorphism. 

Suppose, in addition, that the lattice $\CL$ is complete. 
Consider $S\in \CQ(\CL^\mm)$. 
It is easy to see that 
$$\begin{equation}\label{eqt:u-lequantum v-bot}
v_0:= \bigvee S^\mc \leq a' \quad (a\in  S), \quad \text{ which gives } \quad u_0:= \bigvee S\leq v_0'.
\end{equation}$$
On the other hand, it follows from $S\subseteq \ti \Xi_\CL(u_0)$, $S^\mc \subseteq \ti \Xi_\CL(v_0)$ and $S^{\mc\mc} = S$ that
$\ti \Xi_\CL(v_0') \subseteq \ti \Xi_\CL(u_0)$. 
This, together with the second equality of  \eqref{eqt:u-lequantum v-bot}, produces $S = \ti \Xi_\CL(u_0)$. 
Hence, $\ti \Xi_\CL$ is surjective.

The above, together with parts (a) and (c) of Lemma \ref{lem:lattice}, establishes that $\CL \mapsto (\CL^\star, \neq_\CL)$ is a bijective correspondence from the collection of complete atomistic ortholattices to that of atomic quantum-sets. 
Moreover, the corresponding statement for orthomodular lattices follows from Lemma \ref{lem:lattice}(b). 
\end{proof}

\medskip

Note that the quantum-set $\CL^\star$ in Proposition \ref{prop:atom-hered-q-distinct}(a) is never atomic, unless $\CL = \{0,1\}$, as $\{1\}^{\mc\mc} = \CL^\star$. 
In the case when $\CL\neq \{0,1\}$ and $\CL$ is atomic, the two distinct quantum-sets $(\CL^\star, \neq_\CL)$ and $(\CL^\mm, \neq_\CL)$ give the same ortholattice $\CL$. 


\medskip

Observe also that, without the completeness assumption, the quantum-set $(\CL^\star, \neq_\CL)$ may not be hereditary even when $\CL$ is  orthomodular  (because the MacNeille completion of an orthomodular lattice need not be orthomodular; see e.g. \cite{Adams}). 



% Notice that atomic orthomodular lattice is atomistic: if $y:=\bigvee \ti \Xi(x) \lneq x$, then  as $x = y \vee (x \wedge y')$, we have $x \wedge y'\neq 0$, and it will dominate an element $a\in \ti \Xi(x)$, which gives the contradiction $a\leq y' \leq a'$ (notice that $a \leq \bigvee \ti \Xi(x) = y$). 

\medskip

\begin{comment}
\begin{rem}
Let $\CL_1$ and $\CL_2$ be complete atomistic ortholattices with their sets of atoms being denoted by $\CL^\mm_1$ and $\CL^\mm_2$ respectively. 
Proposition \ref{prop:atom-hered-q-distinct}(b) implies that any quantum bijection from the quantum-set $(\CL^\mm_1, \neq_{\CL^\mm_1})$ onto $(\CL^\mm_2, \neq_{\CL^\mm_2})$ extends to an ortholattice isomorphism from $\CL_1\cong \CQ(\CL^\mm_1)$ onto $\CL_2\cong \CQ(\CL^\mm_1)$. 

In \cite[p.500]{AV}, the notion of ``symmetry'' was introduced, in order to formulate the ``plane transitivity axiom'' that helps to exclude the non-Archimedean case in the Piron representation theorem (see \cite[Proposition 1]{AV}), through the equivalence of this axiom with the ample unitary group axiom (see \cite{Holl} and also \cite{Sol}).
The above tells us that in the case of complete atomistic ortholattices, a symmetry is precisely a quantum bijection between the corresponding quantum-sets of atoms.   
\end{rem}

\end{comment}

%\medskip

%Suppose that $\CL$ is a sublattice of a complete lattices $\CL_0$. 
%Then $\CL_0$ is called a \emph{$0$-regular completion} if any subset $S\subseteq \CL$ that have $0$ as its infimum in $\CL$, the infimum of $S$ in $\CL_0$ is again $0$. 

%\medskip

%The following result tells us that if the MacNeille completion of an ortholattice $\CL$ is orthomodular, then it is the smallest orthomodular $0$-regular completion of $\CL$. 



\begin{comment}
\medskip

As in the proof of Proposition \ref{prop:atom-hered-q-distinct}(a), if there is a quantum bijection $\Phi$ from $\CL_0^\star$ to $\CL^\star$, then $\CL_0\cong \CL$ as ortholattices. 
However, there is no guarantee that $\Phi$ is the restriction of an ortholattice isomorphism from $\CL_0$ onto $\CL$, unless the lattices are complete. 
Nevertheless, in the case of  orthomodular lattices (not necessarily complete), the map $\Phi$ is indeed the restriction of an ortholattice isomorphism. 


\medskip


\begin{cor}\label{cor:assoc-quantum-set}
Let $\CL_1$ and $\CL_2$ be ortholattices such  that $\CL_2$ is orthomodular. 
If $\Phi: \CL_1^\star \to \CL_2^\star$ is a quantum bijection, then $\Phi$ becomes an ortholattice isomorphism from $\CL_1$ onto $\CL_2$ if we set $\Phi(0):=0$.  
\end{cor}
\begin{proof}
Clearly, $\Phi(1) = 1$.  
Fix any $x\in \CL_1^\star$. 
Since $x\neq_{\CL_1} x'$, we know that 
$$\Phi(x')\leq \Phi(x)'.$$ 
Assume on the contrary that $y:= (\Phi(x)\vee \Phi(x'))'\neq 0$. 
Then it follows from $\Phi(x)\leq y'$ that $x\neq_{\CL_1} \Phi^{-1}(y)$. 
Similarly, we have $x'\neq_{\CL_1} \Phi^{-1}(y)$. 
These produce the contradiction that
$$\Phi^{-1}(y)\leq x\wedge x' =0.$$
Consequently, $\Phi(x)' \wedge \Phi(x')' = 0$. 
As $\CL_2$ is orthomodular, we conclude that $\Phi(x') = \Phi(x)'$. 

If $\bar \Phi: \CQ(\CL_1^\star)\to \CQ(\CL_2^\star)$ is the ortholattice isomorphism induced by $\Phi$, then, as in the proof of Proposition \ref{prop:atom-hered-q-distinct}(a), we have $\bar \Phi (\Xi_{\CL_1}(x)) = \Xi_{\CL_2}(\Phi(x')') = \Xi_{\CL_2}(\Phi(x))$ ($x\in \CL_1$).
Therefore, the assignment $x\mapsto \Phi(x)= \Xi_{\CL_2}^{-1}(\bar \Phi(\Xi_{\CL_1}(x)))$ is a bijective ortholattice homomorphism. 
\end{proof}

\end{comment}


\medskip


\begin{eg}\label{eg:reg-open}
(a) Let $A$ be a $C^*$-algebra, and $A^\star := A\setminus \{0\}$. 
For $a,b\in A^\star$, we define 
$$a\neq_\mathrm{z} b\quad \text{if} \quad ab=0=ba \text{ and }a^*b = 0 =ba^*.$$
A subset of $A$ belongs to $\CQ(A^\star, \neq_\mathrm{z})$ if and only if it is an annihilator hereditary $C^*$-subalgebra of $A$ minus the zero element. 
Hence, $\neq_\mathrm{z}$ is never atomic. 
On the other hand, since for an annihilator hereditary $C^*$-subalgebra $B$ of $A$, every annihilator hereditary $C^*$-subalgebra of $A$ that contained in $B$ is an 
annihilator hereditary $C^*$-subalgebra of $B$, we know that $\neq_\mathrm{z}$ is hereditary. 

Recall that every hereditary $C^*$-subalgebra of $A$ is of the form $(1-p) A^{**} (1-p) \cap A$ for a unique closed projection $p\in \CC_0(A)$ (see the Introduction). 
Moreover, a closed projection $p$ corresponds to an element in $\CQ(A^\star, \neq_\mathrm{z})$ if and only if $p = \overline{1-\overline{1-p}}$ (here, for a projection $q\in \CP_{A^{**}}$, we denote by $\overline{q}$ the smallest closed projection dominating $q$).
Thus, when $A$ is commutative, $\CQ(A^\star, \neq_\mathrm{z})$ corresponds bijectively to the set of ``regular open subsets''  (i.e., interiors of closed subsets) of the spectrum of $A$. 



\smnoind
(b) Let $X$ be a compact Hausdorff space. 
It is well-known that the collection $\CR(X)$ of all regular open subsets of $X$ is a complete Boolean algebra with 
$$U\wedge V := U\cap V, \quad U\vee V := \mathrm{Int}(\overline{U}\cup \overline{V})\quad \text{and} \quad U':= X\setminus \overline{U}$$  
(see \cite[p.103-104]{Vlad}), and it coincides with the orthomodular lattice $\CQ(C(X)^\star, \neq_\mathrm{z})$ as in part (a).   
Since every ultrafilter on $\CR(X)$ converges to a unique point in $X$, one obtains a continuous open surjection from the Stone space of $\CR(X)$, denoted by $X^\varepsilon$, onto $X$. 
Note that this construction is different from the ``Stone space associated with $X$'' as introduced in \cite{Mihara}, where the collection of clopen subsets instead of regular open subsets were considered.  

It is not hard to check that $X^\varepsilon$ is the \emph{universal extremally disconnected space associated with  $X$}, in the sense that any continuous map from an extremally disconnected space $Y$ to $X$ can be lifted to a continuous map from $Y$ to $X^\varepsilon$.
Furthermore, it can be shown that $C(X^\varepsilon)$ is precisely the regular monotone completion of the $C^*$-algebra $C(X)$, as introduced in \cite{Ham81}. 

Suppose that $Y$ is another compact Hausdorff space. 
If $\Phi$ is a quantum bijection from $\big(\CR(X)^\star, \neq_{\CR(X)}\!\!\big)$ onto $\big(\CR(Y)^\star, \neq_{\CR(Y)}\!\!\!\big)$, then one can use Proposition \ref{prop:atom-hered-q-distinct}(a) and the Stone representation theorem to obtain a homeomorphism  from $Y^\varepsilon$ onto $X^\varepsilon$ that induces $\Phi$. 
\end{eg}


\medskip

In order to define quantum topology (see Definition \ref{defn:quantum top}), we also need the notion of q-commutativity. 


\begin{comment}
\medskip

\begin{eg}
	Let $X$ be a topological space and $C(X)$ be the set of continuous functions from $X$ to $\BR$. 
	We define $\neq_C$ in the way that $x\neq_C y$ whenever there is $f\in C(X)$ with $f(x) = 0$ and $f(y) \neq 0$. 
	For each $x\in X$ and $f\in C(X)$, we set 
	$$C(X)^x:=\{g\in C(X): g(x) = 0\}\quad  \text{and} \quad Z_f:=\{z\in X: f(z) = 0 \}.$$ 
	Obviously, $\{x\}^\mc = \bigcup_{g\in C(X)^{x}} X\setminus Z_g$ and it is easy to check that $(\{x\}^\mc)^\mc = \bigcap_{g\in C(X)^{x}} Z_g$. 
	Hence, $\neq_C$ is atomic if and only if $\bigcap_{g\in C(X)^{z}} Z_g$ is a singleton set for every $z\in X$ (which is equivalent to $\neq_C$ being classical).  
	Furthermore, we have 
	$$S^\mc = X \setminus {\bigcup}_{x\in S} {\bigcap}_{g\in C(X)^x} Z_g \quad \text{and} \quad (S^\mc)^\mc = {\bigcup}_{x\in S} {\bigcap}_{g\in C(X)^x} Z_g \qquad (S\in \CP(X)).$$
	This means that for the quantum-set $(X, \neq_C)$, the q-complements of quantum subsets are their usual complements. 
	Therefore, $\neq_C$ is hereditary. 
	Moreover, the q-union coincides with ordinary union, and any two quantum subsets q-commute (although there exist non-q-commuting singleton subsets of $X$ unless $\neq_C$ is classical). 

A similar construction can be done for any vector subspace of real-valued function on a set $X$ that contains the constant functions. 
\end{eg}

\end{comment}


\medskip


\begin{defn}\label{defn:q-comm}
Let $\CL$ be an ortholattice and $p,q\in \CL$. 

\smnoind
(a) We say that $p$ \emph{q-commutes with} $q$  if $p\wedge(p\wedge q)' \leq (q\wedge (p\wedge q)')'$.

\smnoind
(b) $p\in \CL$ is said to be \emph{q-central} if it q-commutes with all other elements in $\CL$. 
\end{defn}

\medskip

For a quantum-set $(X, \neq_\mq)$, it is easy to see that both $X$ and $\emptyset$  are q-central in $\CQ(X)$. 
Moreover, in this case, we may extend the notion of q-commutativity to general subsets: 
two subsets $C,D\in \CP(X)$ are said to be \emph{q-commutes} if 
$$C\cap (C\cap D)^\mc \subseteq (D\cap (C\cap D)^\mc)^\mc.$$
Notice that if either $C\subseteq D$ or $C\subseteq D^\mc$, then $C$ will q-commute with $D$. 
Hence, if $x\in X$, then the singleton subset $\{x\}$ q-commutes with $D$ if and only if either $x\in D$ or $x\in D^\mc$. 
This clearly gives the following statements: 
\begin{itemize}
	\item a subset $D\subseteq X$ q-commutes with all singleton subsets of $X$ if and only if $D^\mc = X\setminus D$;
	
	\item for any $x,y\in X$, one has $\{x\}$ q-commutes with $\{y\}$ if and only if either $x=y$ or $x\neq_\mq y$. 
\end{itemize}
The second statement above, together with Proposition \ref{prop:atom-hered-q-distinct}(b),  tells us that 
an atomistic ortholattice is a Boolean algebra if and only if all its atoms q-commute with one another.

\medskip

Let us also recall from \cite[p.20]{Kal83} that $p$ is said to \emph{commute with} $q$ in an ortholattice if 
$$p = (p\wedge q)\vee (p\wedge q').$$ 
Note that this commutativity relation is, in general, asymmetric. 
It is natural to ask whether there is any relation between this commutativity and the q-commutativity above. 

\medskip

\begin{prop}\label{prop:cp-comm}
Let $\CL$ be an ortholattice. 
Then $\CL$ is orthomodular if and only if the q-commutativity relation coincides with the commutativity relation on $\CL$. 
\end{prop}
\begin{proof}
$\Rightarrow)$. 
Consider $p,q\in \CL$. 
Suppose that $p$ q-commutes with $q$. 
As $\CL$ is orthomodular, we have $q = (p\wedge q)\vee (q\wedge (p\wedge q)')$. 
From this, we know that
	$$q' = (p\wedge q)' \wedge (q\wedge (p\wedge q)')' \geq (p\wedge q)' \wedge (p\wedge (p\wedge q)')  = p\wedge (p\wedge q)'.$$
Thus, $(p\wedge q)\vee (p\wedge q') \geq (p\wedge q)\vee (p\wedge (p\wedge q)') = p$ (again by the unimodular law). 

Conversely, suppose that $p$ commutes with $q$. 
Then, by Lemma 1 of \cite[\S 1.4]{Kal83}, the sub-ortholattice generated by $p$ and $q$ is distributive. 
Consequently, 
$$p\wedge (p\wedge q)' = p \wedge (p' \vee q') = (p\wedge p')\vee (p\wedge q') = p\wedge q'.$$
Similarly, $q\wedge (p\wedge q)' = q \wedge p'$ and hence, $p\wedge (p\wedge q)' \leq (q\wedge (p\wedge q)')'$. 

\noindent
$\Leftarrow)$. 
Note that the q-commutativity relation is symmetric. 
However, it was shown in Theorem 2 of \cite[\S 1.3]{Kal83} that if the commutative relation is symmetric, then $\CL$ is orthomodular. 
\end{proof}

\medskip

\begin{cor}\label{cor:q-cent-hered}
Let $(X, \neq_\mq)$ be a hereditary quantum-set. 

\smnoind
(a) If two elements $S$ and $T$ in $\CQ(X)$ q-commute, then the sub-ortholattice generated by $S$ and $T$ is distributive and, in particular,  
$S$ and $T^\mc$ q-commute. 
%Consequently, $T\in \CQ(X)$ is q-central if and only if $T^\mc$ is q-central. 

\smnoind
(b) $S\in \CQ(X)$ is q-central if and only if $X\setminus (S\cup S^\mc)$ contains no non-empty element in $\CQ(X)$. 

\smnoind
(c) Suppose, in addition, that $\neq_\mq$ is atomic. 
Then a subset $D\subseteq X$ belongs to $\CQ(X)$ and is q-central if and only if $D^\mc = X \setminus D$. 
\end{cor}
\begin{proof}
(a) This follows from Proposition \ref{prop:cp-comm} and Lemma 1 of \cite[\S 1.4]{Kal83} (see also Lemma \ref{lem:lattice}(b)). 


\smnoind
(b) Lemma \ref{lem:lattice}(b) implies that $\CQ(X)$ is an orthomodular lattice. 
Suppose that $S$ is q-central. 
Consider $T\in \CQ(X)$ with $T\subseteq (X\setminus S)\cap (X\setminus S^\mc)$. 
As $T$ commutes with $S$ (by Proposition \ref{prop:cp-comm}), we see that 
$T  = (T\cap S)\vee (T\cap S^\mc)$, but $(T\cap S)\vee (T\cap S^\mc) = \emptyset$. 
Conversely, suppose that the only element in $\CQ(X)$ which is contained in $X\setminus (S\cup S^\mc)$ is the empty set. 
Consider $T\in \CQ(X)$. 
It is obvious that $T \supseteq (T\cap S)\vee (T\cap S^\mc)$.
Set 
$$R:= T \cap ((T\cap S)\vee (T\cap S^\mc))^\mc.$$ 
Then $R\cap S = T\cap S \cap (T\cap S)^\mc \cap (T\cap S^\mc)^\mc = \emptyset$, and similarly, we have $R\cap S^\mc = \emptyset$. 
Thus, $R \subseteq (X\setminus S)\cap (X\setminus S^\mc)$ and the hypothesis implies that $R =\emptyset$. 
As $\CQ(X)$ is orthomodular, we see that $T = (T\cap S)\vee (T\cap S^\mc)$; i.e., $T$ commutes with $S$. 
The conclusion now follows from Proposition \ref{prop:cp-comm}. 

\smnoind
(c) If $D\in \CQ(X)$ and is q-central, then part (b) and the atomic assumption give $D^\mc = X \setminus D$. 
Conversely, suppose that $D^\mc = X \setminus D$. 
Then it is easy to see that $D\in \CQ(X)$, and we conclude from part (b) that $D$ is q-central. 
% alternative argument
%Lemma \ref{lem:q-comm}(a) tells us that $D$ q-commutes with all elements in $\CQ(X)^\mm$. 
%Hence, we know from Proposition 4 of \cite[\S 1.3]{Kal83} as well as  Propositions \ref{prop:atom-hered-q-distinct}(b) and \ref{prop:cp-comm} that $D$ commutes with all elements in $\CQ(X)$. 
\end{proof}


\begin{comment}

\medskip

It is an interesting to ask when an atomistic orthomodular lattice is the lattice of projection of Hilbert space. 
Motivated by this, we say that a hereditary atomic quantum-set $(X, \neq_\mq)$ is a 
\emph{dimension quantum-set} if 
\begin{itemize}
	\item there is no non-trivial q-central element in $\CQ(X, \neq_\mq)$;

	\item for any $C\subseteq X$, the cardinality of every maximal q-distinct family in 
	$X[\bigvee_{c\in C} \{c\}]:= \big\{x\in X: x\in \bigvee_{c\in C} \{c\} \big\}$ is dominated by the cardinality of $C$; 
\end{itemize}
where $Y\subseteq X$ is called a \emph{q-distinct family} in $X$ if elements in $Y$ are q-distinct with one another. 
In this case, for every $T\in \CQ(X,\neq_\mq)$, the cardinalities of all every maximal q-distinct families in $T$ are the same, and it is called the \emph{dimension} of $T$.  
Moreover, we say that $(X,\neq_\mq)$ is \emph{finite dimensional} if there exists a finite number of elements $x_1,...,x_n\in X$ such that $X = \{x_1\}\vee \cdots \vee \{x_n\}$.  
The natural question whether dimension quantum-sets with the same finite dimension are isomorphic. 

\end{comment}

\medskip

\begin{eg}
	Let $\BR\BP^1$ be the one dimensional projective space.
	%, i.e. the quotient of the unit sphere of $\BR^2$ under the equivalent relation that identifies ``opposite points'' (i.e. the straight line joining the two points passing through the origin).
	The absolute value of the usual inner product on $\BR^2$ induces a function $\tau: \BR\BP^1\times \BR\BP^1 \to [0,1]$ known as the transition probability. 
	We define
	$$x \neq_\mathrm{ht}y \quad \text{when} \quad \tau(x,y) \leq 1/2.$$  
	Obviously, $\neq_\mathrm{ht}$ is a q-distinctness relation. 
	We identify $\BR\BP^1$  with the interval $[0,3)$ as sets  (i.e., not homeomorphic) through the assignment that sends $\theta$ to the image of $\mathrm{e}^{\theta\pi\mathrm{i}/3}$ in $\BR\BP^1$. 
	Define $\kappa: [0,6) \to [0,3)$ to be the map which coincides with the identity map on $[0,3)$ and $\kappa (\theta) := \theta - 3$ when $\theta \geq 3$. 
	Then 
	$$\{\theta\}^\mc = \kappa\big([\theta+ 1, \theta+ 2]\big) \quad \text{and} \quad \{\theta\}^{\mc\mc} = \{\theta\} \qquad (\theta\in [0,3)).$$ 
	This means that $\neq_\mathrm{ht}$ is atomic. 
	If $\theta_1, \theta_2\in [0,6)$ with $\theta_2 - \theta_1 \in [0,3)$, then we say that  $\kappa\big([\theta_1, \theta_2]\big)$ is an \emph{arc of length $\theta_2 - \theta_1$} (note that an arc of length zero is a singleton subset). 
	
	Pick  an arbitrary non-empty subset $D\subseteq \BR\BP^1$. 
	If $D$ is contained in an arc of length $1$, then $D^{\mc\mc}$ is the smallest arc containing $D$; otherwise, we have $D^{\mc\mc} = \BR\BP^1$. 
	Therefore, 
	$$\CQ\big(\BR\BP^1, \neq_\mathrm{ht}\!\!\big) = \big\{\emptyset, \BR\BP^1 \big\}\cup \big\{ S: S\subseteq \BR\BP^1 \text{ is an arc of length dominated by }1 \big\}. $$
	Consider $T = \kappa\big([0, 1]\big)$ and $S := \kappa\big([0, 1/2]\big)$. 
	Then $S^\mc \cap T = \emptyset$ and hence $(S^\mc \cap T)^\mc  \cap T = T\neq S$. 
	In other words, $\neq_\mathrm{ht}$ is not hereditary.
		
Suppose that $S, T\in  \CQ(\BR\BP^1, \neq_\mathrm{ht})\setminus \{\emptyset, \BR\BP^1 \}$. 
If $S$ is an arc of length strictly less than one, then $S$ commutes with $T$ if and only if either $S\subseteq T$ or $S\subseteq T^\mc$.
If $S$ is an arc of length one, then $S$ commutes with $T$ if and only if either $S = T$, $S = T^\mc$ or both $S\cap T$ and $S\cap T^\mc$ are singleton subsets. 
However,  $S$ q-commutes with $T$ if and only if either $S\cap T\neq \emptyset$ or $S\subseteq T^\mc$.  
In particular, $\emptyset$ and $\BR\BP^1$ are the only q-central elements in $\CQ(\BR\BP^1, \neq_\mathrm{ht})$. %Two singleton subsets $\{x\}$ and $\{y\}$  q-commute if and only if $\{x\}\subseteq \{y\}^\mc$; equivalently, there is no arc of length less than 1 containing both $x$ and $y$. 
%Then $S$ and $\{x\}$ q-commutes if and only if either $x\in S$ or $x\in S^\mc$. 
%On the other hand, $x$ commutes with $S$ if and only if $x\in S$, while $S$ commutes with $x$ if and only if $x\in S^\mc$. 
%Suppose that $T\subseteq \BR\BP^1$ is another arc of length dominated by $1$. 
% Note that $S\cap T$ is either an empty set, a singleton set or an arc.
% Note that if $S\cap T\neq \emptyset$, then $S\cap (S\cap T)^\mc$ is either an empty set or a  singleton subset
\end{eg}

\medskip

It is more interesting to look at the case when the q-distinctness relation is defined by elements having zero transition probability. 
This will be considered in Section \ref{sec:non-comm-Gelf} below. 
As said in the Introduction, Section \ref{sec:non-comm-Gelf} contains a non-commutative version of the Gelfand theorem, and this theorem requires a Dye's theorem for q-closed projections. 
Thus, we will consider this form of Dye's theorem in the next section. 


\medskip

\section{A Dye's theorem for $C^*$-algebras}

\medskip

Let us first give some notation. 
In the following, we denote by $A_\sa$ the set of self-adjoint elements of a $C^*$-algebra $A$, and by $\widehat{A}$ the set of (unitary equivalence classes) of irreducible $^*$-representations of $A$. 
%For any $\CS \subseteq \CP_{A^{**}}$, the smallest projection in  ${A^{**}}$ that dominate all elements in $\CS$ will be denoted by $\sup \CS$. 

\medskip

As in the previous section, $\CP^\mm_{A^{**}}$ is the set of atoms of the complete orthomodular lattice $\CP_{A^{**}}$ of projections in the bidual $A^{**}$.
We denote by $A^{**}_\ba$ the weak-$^*$-closed linear span of $\CP^\mm_{A^{**}}$. 
%Moreover,  $\bigvee$ is the supremum taken in $\CP_{A^{**}}$. 
Furthermore, we define $\bz^A:= \bigvee \CP_{A^{**}}^\mm\in \CP_{A^{**}}$ and consider the normal $^*$-homomorphism $\Lambda_A:A^{**} \to A^{**}_\ba$ given by 
$$\begin{equation}\label{eqt:def-Lambda}
\Lambda_A(x) := \bz^Ax \qquad (x\in A^{**}).
\end{equation}$$
Since the restriction of $\Lambda_A$ on the multiplier algebra $M(A)$ is injective (when $M(A)$ is considered as a subspace of $A^{**}$ in the canonical way), by abuse of notation, we will consider $M(A)$ as a $C^*$-subalgebra of $A^{**}_\ba$. 
%We warn the reader that if we consider $A^{**}_\ba\subseteq A^{**}$, the canonical image of $A$ in $A^{**}_\ba$ will not be the same as the canonical image of $A$ in $A^{**}$ under this inclusion. 
%We will also consider the multiplier algebra $M(A)$ as a $C^*$-subalgebra of $A^{**}_\ba$ in the  canonical way. 

\medskip

%Recall that an element $p\in \CP_{A^{**}}$ is called an \emph{open projection} of $A$ if there is an increasing net $\{a_i \}_{i\in \KI}$ in $A_+$ that weak-$^*$-converges to $p$ (see \cite[\S 3.11.10]{Ped79}). 
%On the other hand, an element $p\in \CP_{A^{**}}$ is called an \emph{closed projection} of $A$ if $1-p$ is open. 
Recall that an element $p\in \CP_{A^{**}}$ is an \emph{open projection} of $A$ if $1-p$ is a closed projection of $A$; i.e., there exists an increasing net $\{a_i\}_{i\in \KI}$ in $A_+$ such that $a_i$ weak-$^*$-converges to $p$. 
The image of a closed projection (respectively, an open projection) under $\Lambda_A$ is  called a \emph{q-closed projection} (respectively, a  \emph{q-open projection}). 
The set of all q-closed projections of $A$ will be denoted by $\CC(A)$. 
%Let 
%$$\CC_\mc(A):= \{p\in \CC(A): p\leq a, \text{ for some }a\in A_+ \}.$$
%Elements in $\CC_\mc$ are called \emph{quantum compact projections} of $A$. 
%Clearly, when $A$ is unital, all q-closed projections are quantum compact. 

\begin{comment}
\medskip

We also recall that there is an order preserving bijection from the set of q-open projections (respectively, central q-open projections) of $A$ to the set of closed left ideals (respectively, closed ideals) of $A$ under the assignment 
$$p\mapsto A^{**}_\ba p \cap A$$
(see e.g., \cite[Theorem II.17]{Ake69} and \cite[\S 3.11.10]{Ped79}). 

\end{comment}


\medskip

Before we start proving the main theorem (Theorem \ref{thm:main}) of this section, we first show that the corresponding statement of this theorem with ``q-closed projections'' being replaced by ``q-open projections''  is false (similarly, Corollary \ref{cor:closed-proj} fails when we consider open projections instead of closed projections). 


\medskip

\begin{eg}\label{eg:counter}
	Note that $C([0,1])^{**}_\ba \cong \ell^\infty([0,1])$.
	A projection $p\in \ell^\infty([0,1])$ is a q-open projection of $C([0,1])$ if and only if $p$ is the indicator function of an open subset of $[0,1]$. 
	Consider $\CO$ to be the collection of all open subsets of $[0,1]$. 
	Define $\varphi: \CO \to \CO$ such that 
	$\varphi$ exchanges $[0,1)$ and $(0,1]$, but $\varphi$ fixes all the other open subsets.
	Then $\varphi$ induces a quantum bijection $\hat\varphi$ from the set of all non-zero q-open projections of $C([0,1])$ onto itself. 
	However,  $\hat \varphi$ cannot be the restriction of the bidual of a $^*$-automorphism of $C([0,1])$, because a $^*$-automorphism of $C([0,1])$ is given by a homeomorphism on $[0,1]$, yet $\varphi$ cannot be induced by a homeomorphism. 
\end{eg}

\medskip

More generally, let $X$ be a compact Hausdorff space.
If $U\subseteq X$ is an open dense subset, then there is no non-trivial open subset having empty intersection with $U$. 
Thus, as in the example above, if a map permutes open dense subsets of $X$ but keeps all the other open subsets fixed, then this map induces a quantum bijection between non-zero q-open projections but this quantum bijection does not come from a homeomorphism of $X$ to itself. 
One may wonder what will happen if one gets rid of those problematic open dense subsets by considering only regular open subsets.  
In Example \ref{eg:reg-open}, we have seen that in this case, we do not get a homeomorphism from $X$ to itself, but a homeomorphism from the spectrum of the regular monotone completion of $C(X)$ to itself. 
Therefore, there seems to have no way to get a Dye's theorem for q-open projections of $C^*$-algebras. 




%\medskip

%On the other hand, let $M(A)$ be the multiplier algebra of $A$ and $i_A:A\to M(A)$ be  the canonical embedding.
%There is an injective $^*$-homomorphism $j_A:M(A)\to A^{**}_\ba$ satisfying 
%$$j_A(m)b = mb \qquad (m\in M(A), b\in A).$$  
%Clearly, $j_A\circ i_A$ is the canonical embedding from $A$ to $A^{**}$. 


%\medskip

%The following result is more or less included in \cite[Proposition 2.3]{NW}. 
%We give its argument here for clarity and completeness. 


\medskip

Let us begin our proof for the main theorem by giving some required results. 
Our first proposition in the section is an ``atomic version'' of \cite[Theorem 2.2]{APT73}. 
Although this fact could be regarded as known, we nevertheless give a clear presentation of it for completeness. 

\medskip

\begin{prop}\label{prop:atom-Thm2.2-APT}
	Let $A$ be a $C^*$-algebra and $x\in A^{**}_\ba$ be a self-adjoint element.
	Then $x\in M(A)$ if and only if all its spectral projections (in $A^{**}_\ba$) corresponding to closed subsets of $\BR$ being q-closed. 
\end{prop}
\begin{proof}
	If $x\in M(A)$, then it follows from \cite[Theorem 2.2]{APT73} that all its spectral projections (in $A^{**}_\ba$) corresponding to closed subsets of $\BR$ being q-closed. 
	Conversely, suppose that such a property holds for $x$. 
	Let $\CU_0\subseteq A^{**}_\sa$ be the monotone sequential closed (real) Jordan algebra as in \cite[Proposition 3.6]{Ped72}. 
	It was shown in \cite[Theorem 4]{Brown14} that $\CU_0$ is the self-adjoint part of a $C^*$-algebra. 
	Moreover, as noted in the statement preceding \cite[Lemma 3.5]{Ped72}, $\CU_0$ contains the weak-$^*$-limits of all increasing nets in $A_\sa$. 
	Hence, $\CU_0$ contains all the open projections of $A$. 
	
	For any Borel subset $S\subseteq \BR$, we denote by $\chi_S$ the indicator function of $S$. 
	Consider $\alpha, \beta, \gamma \in \BR$ with $\alpha < \beta < \gamma$. 
	By the hypothesis, $\chi_{(\alpha,\gamma)}(x)$ is q-open, and  we fix an open projection $p_{(\alpha, \gamma)}\in \CP_{A^{**}}$ satisfying $\Lambda_A(p_{(\alpha, \gamma)}) = \chi_{(\alpha,\gamma)}(x)$.
	Since $\chi_{(\alpha, \beta]}(x) =  \chi_{(\alpha, \gamma)}(x) - \chi_{(\beta, \gamma)}(x)$, the element $p_{(\alpha, \beta]} := p_{(\alpha,\gamma)} - p_{(\beta,\gamma)}$ in $\CU_0$ will satisfy
	$$\Lambda_A(p_{(\alpha, \beta]}) = \chi_{(\alpha,\beta]}(x).$$
	
	Let us consider a sequence $\{x_n\}_{n\in \BN}$ in $A^{**}_\ba$ whose members are real linear spans of elements of the form $\chi_{(\alpha, \beta]}(x)$ such that $\|x-x_n\| \to 0$. 
	The above then produces a sequence $\{y_n\}_{n\in \BN}$ in $\CU_0$ with $\Lambda_A(y_n) = x_n$. 
	Since the restriction of $\Lambda_A$ on $\CU_0$ is isometric (see \cite[Theorem 3.8]{Ped72}), we know that the sequence $\{y_n\}_{n\in \BN}$ is Cauchy in $\CU_0$, and hence it converges to an element $y\in \CU_0$ that satisfies $\Lambda_A(y) = x$. 
	
	Pick an arbitrary open subset $O\subseteq (-\|x\| - 1, \|x\| + 1)$. 
	There exists an increasing sequence $\{f_n\}_{n\in \BN}$ in $C_0(\BR)_+$ that converges pointwisely to $\chi_O$. 
	As $f_n(y)\in \CU_0$ and $\CU_0$ is monotone sequential closed, we know that $\chi_O(y)$ belongs to $\CU_0$. 
	Since the $^*$-homomorphism $\Lambda_A$ is weak-$^*$-continuous, we have $\Lambda_A\big(\chi_O(y)\big) = \chi_O(x)$. 
	On the other hand, the hypothesis tells us that there is an open projection $q$ of $A$ with $\Lambda_A(q) = \chi_O(x)$. 
	Now, because both $\chi_O(y)$ and $q$ belongs to $\CU_0$ and $\Lambda_A$ restricts to an injection on $\CU_0$, we know that $q = \chi_O(y)$. 
	In other words, all the spectral projections of $y$ in $A^{**}$ with respects to open subsets of $\BR$ are open projections of $A$.
	Therefore, \cite[Theorem 2.2]{APT73} tells us that $y$ belongs to the canonical image of $M(A)$ in $A^{**}$. 
	Hence, $x$ is in the canonical image of $M(A)$ in $A^{**}_\ba$.
\end{proof}

% Using \cite[Theorem 2.2]{APT73}, one can show that Jordan $^*$-isomorphism between two $C^*$-algebras extends to a Jordan $^*$-isomorphism between their multiplier algebras. 


\medskip

\begin{lem}\label{lem:C-st-alg}
	Let $A$ and $B$ be $C^*$-algebras. 
	Suppose that $\Phi:A^{**}_\ba \to B^{**}_\ba$ is a weak-$^*$-continuous unital Jordan $^*$-homomorphism.
	
	\smnoind
	(a) If $\Phi(\CC(A))\subseteq \CC(B)$, then $\Phi\big(M(A))\subseteq M(B)$. 
	
	\smnoind
	(b) If $\Phi$ is bijective and $\Phi(M(A)) =  M(B)$, then $\Phi\big(A)= B$.
\end{lem}
\begin{proof}
	(a) Consider $a\in M(A)_\sa$. 
	The unital von Neumann subalgebra $W^*(a,1)$ of $A^{**}_\ba$ generated by $a$ is commutative. 
	Hence, $\Phi$ restricts to a weak-$^*$-continuous unital $^*$-homomorphism from $W^*(a,1)$ to $W^*\big(\Phi(a),1\big)$. 
	Now, it follows from Proposition \ref{prop:atom-Thm2.2-APT} that $\Phi(a)\in M(B)$.
	
	\smnoind
	(b) Denote by  $\iota_A:M(A)\to A^{**}_\ba$ the canonical embedding, and consider $\bar \iota_A: M(A)^{**}_\ba \to A^{**}_\ba$ to be the restriction of the weak-$^*$-continuous $^*$-homomorphism $\ti\iota_A: M(A)^{**} \to A^{**}_\ba$ extending $\iota_A$. 
	It is not hard to see that the support of $\bar \iota_A$ coincides with the unique central q-open projection  $q_A$ of $M(A)$ that satisfies $A = q_AM(A)^{**}q_A\cap M(A)$. 
	Define $\Psi(x) := \iota_B^{-1}\circ \Phi\circ \iota_A(x)\in M(B)$ for $x\in M(A)$.
	Then $\ti \iota_B\circ \Psi^{**}  = \Phi\circ \ti \iota_A$ and 
	$$\bar \iota_B\circ \Psi^{**}|_{M(A)^{**}_\ba} = \Phi\circ \bar \iota_A.$$
	Thus, $\bar \iota_B(\Psi^{**}(q_A)) = 1$ which implies $\Psi^{**}(q_A)\geq q_B$.
	Hence, $\Psi(A)\supseteq B$; i.e., $B\subseteq \Phi(A)$.
	Similarly, we have $A\subseteq \Phi^{-1}(B)$. 
\end{proof}

\medskip

Note the difference between part (b) above and \cite[Proposition 2.3]{NW} that the atomic parts of the biduals are considered here (and so a different proof is required). 


\medskip

\begin{thm}\label{thm:main}
	Let $A$ and $B$ be two $C^*$-algebras such that $A$ does not have a 2 dimensional irreducible $^*$-representation. 
	If $\Phi: \CC(A)\setminus \{0\} \to \CC(B)\setminus \{0\}$ is a quantum bijection (where the q-distinctness relations are the orthogonality relations), then there exists a Jordan $^*$-isomorphism $\Theta:A\to B$ with $\Phi = \Theta^{**}|_{\CC(A)}$. 
\end{thm}
\begin{proof}
As in Proposition \ref{prop:atom-hered-q-distinct}, we consider the q-distinctness relation $\neq_{\CP_{A^{**}_\ba}}$ on $\CP_{A^{**}_\ba}^\star$. 
Obviously, this q-distinctness relation extends the one on $\CC(A)\setminus \{0\}$ as considered in the statement. 
In the following, $^\mc$ is the q-complement in $\CP\big(\CP_{A^{**}_\ba}^\star\big)$. 
Let us set 
%$$\CS^\mc^\mm := \CS^\mc\cap \CP_{{A^{**}}}^\mm = \{e\in \CP_{{A^{**}}}^\mm: e p = 0, \text{ for any }p\in \CS \} \qquad (\CS\subseteq \CP_{A^{**}_\ba}).$$
%On the other hand, we denote
$$\begin{equation}\label{eqt:def-CP[p]}
\CP^\mm_{A^{**}}[p]:=\{e\in \CP^\mm_{A^{**}}: e \leq p \} \qquad (p\in \CP_{A^{**}_\ba}).
\end{equation}$$
We recall that minimal projections in $A^{**}_\ba$ are q-closed (see e.g. \cite[Corollary 2]{Akemann68}). 
Thus, $\CP^\mm_{A^{**}}$ coincides with the set of minimum elements in the ordered subset $\CC(A)\setminus \{0\}$ of $\CP_{A^{**}_\ba}$. 
	
	Suppose that $r,s\in \CC(A)$ with $r\leq s$. 
	For any $f\in \{\Phi(s)\}^\mc\cap \CP_{B^{**}}^\mm$, one has $\Phi^{-1}(f)s = 0$ and hence  
	$$f\in \{\Phi(r)\}^\mc\cap \CP_{B^{**}}^\mm.$$ 
	As $\CP_{B^{**}_\ba}$ is an atomistic lattice,  the above implies that $\{\Phi(s)\}^\mc\subseteq \{\Phi(r)\}^\mc$, and hence, $\Phi(r)\leq \Phi(s)$. 
	This means that $\Phi$ is order preserving.
	Similarly, the same is true for $\Phi^{-1}$.
	In particular, $\Phi(\CP^\mm_{A^{**}}) = \CP^\mm_{B^{**}}$.
	%	$$\begin{equation*}\label{eqt:Phi-pres-min}
	%	\Phi(\CP^\mm_{A^{**}}) = \CP^\mm_{B^{**}}.
	%	\end{equation*}$$
	Thus, by the proof of Proposition \ref{prop:atom-hered-q-distinct}(b), there is an ortholattice isomorphism $\Upsilon: \CP_{A^{**}_\ba}\to \CP_{B^{**}_\ba}$ satisfying 
	$$\Upsilon(p):= \bigvee \Phi\big(\CP^\mm_{A^{**}} [p]\big) \qquad (p\in \CP_{A^{**}_\ba}).$$
	
	Since $\Phi$ is an order isomorphism, if $r\in \CC(A)$, then 
	$\Phi\big(\CP^\mm_{A^{**}} [r]\big) = \CP^\mm_{B^{**}} [\Phi(r)],$
	and hence $\Upsilon(r) = \bigvee \CP^\mm_{B^{**}} [\Phi(r)] = \Phi(r)$. 
	This means that  $\Upsilon$ extends $ \Phi$. 
	Moreover, we have 
	$$\begin{equation}\label{eqt:bij}
	\CP^\mm_{B^{**}}[\Upsilon(p)] =  \Phi\big(\CP^\mm_{A^{**}}[p]\big)  \qquad (p\in \CP_{A^{**}_\ba}).
	\end{equation}$$
	
	
	In the same way, the quantum bijection $\Phi^{-1}$ also extends to an ortholattice isomorphism $\Upsilon^\#: \CP_{B^{**}_\ba} \to \CP_{A^{**}_\ba}$. 
	By Relation \eqref{eqt:bij}, the map $\Upsilon^\#$ is the inverse of $\Upsilon$, which implies that $\Upsilon|_{\CP_{A^{**}_\ba}^\star}$ is a quantum bijection. 
	Therefore, it follows from Theorem \ref{thm-Dye} that $\Upsilon$ extends to a Jordan $^*$-isomorphism $\bar \Phi$ from $A^{**}_\ba$ onto $B^{**}_\ba$. 
	The assumption on $\Phi$ and Lemma \ref{lem:C-st-alg} now tells us that $\bar \Phi(A) = B$.
	
	We denote $\Theta:= \bar \Phi|_A$. 
	By the weak-$^*$-density of $A$ in $A^{**}_\ba$ and the automatic weak-$^*$-continuity of $\bar \Phi$, we know that $\bar \Phi\circ \Lambda_A = \Lambda_B\circ \Theta^{**}$.
	Finally, the fact that $\bar \Phi$ extends $ \Phi$ implies that $\Theta^{**}$ extends $\Phi$.  
\end{proof}

\medskip

\begin{rem}\label{rem:main-general-case}
	(a) As in the case of von Neumann algebra, when $A = B = \BM_2$, a quantum bijection from $\CC(A)\setminus \{0\} = \CP^A\setminus \{0\}$ onto $\CC(B)\setminus \{0\}$ may not comes from a Jordan $^*$-isomorphism from $A$ to $B$ (see Example \ref{eg:M2} in the next section). 
	
	\smnoind
	(b) The assumption on $A$ concerning irreducible $^*$-representations can be rephrased as $A$ not having $\BM_2$ as a quotient $C^*$-algebra; equivalently, for any closed ideal $I\subseteq A$ with $A/I$ not having a character, there exist $x_1, x_2, x_3, x_4\in A/I$ such that 
	$${\sum}_{\sigma\in S_4} \mathrm{sgn}(\sigma) x_{\sigma(1)}x_{\sigma(2)}x_{\sigma(3)}x_{\sigma(4)}\neq 0,$$ where $S_4$ is the permutation group on 4 elements and $\mathrm{sgn}(\sigma)$ is the sign of $\sigma$ (see \cite[Proposition 2.3]{FR}). 
\end{rem}


%\medskip


%On the other hand, notice that our proof for Theorem \ref{thm:main} only requires the Dye's theorem in the case of atomic von Neumann algebras. 
%Observe also that in the case when $A$ is a von Neumann algebra, projections in $A$ are those q-closed projections of $A$ that happens to live on $A$. 
%Therefore, Theorem \ref{thm:main} is not a direct extension of Theorem \ref{thm-Dye}. 


\medskip

We also have the corresponding statement of Theorem \ref{thm:main} concerning the set $\CC_0(A)$ of all closed projections (instead of q-closed projections). 

\medskip

\begin{cor}\label{cor:closed-proj}
	Let $A$ and $B$ be two $C^*$-algebras such that $\BM_2$ is not a quotient $C^*$-algebra of $A$. 
	If $\Phi: \CC_0(A)\setminus \{0\} \to \CC_0(B)\setminus \{0\}$ is a quantum bijection, there is a Jordan $^*$-isomorphism $\Theta:A\to B$ such that $\Phi = \Theta^{**}|_{\CC_0(A)}$. 
\end{cor}
\begin{proof}
By \cite[Theorem II.17]{Ake69}, we know that $\Lambda_A$ induces a bijection from $\CC_0(A)$ onto $\CC(A)$ (notice that although the unital assumption is needed in \cite{Ake69}, the result \cite[Theorem II.17]{Ake69} still holds without the unital assumption). 
Consider $p,q\in \CC_0(A)$ with $\Lambda_A(p) \Lambda_A(q) = 0$. 
Then $\Lambda_A(p)\leq \Lambda_A(\mathbf{1}-q)$ and \cite[Theorem II.17]{Ake69} tells us that $p \leq \mathbf{1}-q$, because $\mathbf{1} - q$ is an open projection of $A$. 
This means that $\Lambda_A: \CC_0(A)\setminus \{0\} \to \CC(A)\setminus \{0\}$ is a quantum bijection. 
Now, the argument of Theorem \ref{thm:main} implies the required conclusion. 
\end{proof}

\medskip

\section{Quan-topologial spaces and quantum spectra of $C^*$-algebras}\label{sec:non-comm-Gelf}

\medskip

\begin{defn}\label{defn:quantum top}
Let $\CL$ be an ortholattice. 
A \emph{quantum topology} on $\CL$ is a subcollection $\CC\subseteq \CL$ satisfying:
\begin{enumerate}[\ \ S1).]
	\item $0, 1\in \CC$;
	\item if $\{{p_\lambda}\}_{\lambda\in \Lambda}$ is a family in $\CC$, then $\bigwedge_{\lambda\in \Lambda}p_\lambda$ exists and belongs to $\CC$;
	\item if $p$ and $q$ are q-commuting elements in $\CC$ (see Definition \ref{defn:q-comm}(a)), then $p\vee q\in \CC$.
\end{enumerate} 
In this case, elements in $\CC$ are said to be \emph{quantum closed}, while elements of the form $p'$ for some $p\in \CC$ are said to be \emph{quantum open}. 
\end{defn}	

\medskip

By Proposition \ref{prop:cp-comm}, if $\CL$ is an orthomodular lattice, then one may replace the q-commutativity in Condition (S3) with the commutativity as defined in \cite[p.20]{Kal83}.

\medskip


\medskip 
	
\begin{defn}
(a)	Let $(X, \neq_\mq)$ be a quantum-set.  
If $\CC\subseteq \CQ(X)$ is a quantum topology, then $(X, \neq_\mq, \CC)$ is called a \emph{quantum topological space}. 

	
\smnoind
(b) A \emph{quantum homeomorphism} from a quantum topological space $(X, \neq_\mq, \CC)$ to another  quantum topological space $(Y, \neq_\mq, \mathcal{D})$ is a quantum bijection $\Psi:X\to Y$ satisfying $\mathcal{D} = \{\Psi(C): C\in \CC \}$. 
\end{defn}

%\medskip

%If $\CC$ is a quantum topology on $\CL$, then $\CL^\star$ is a quantum topological space under the induced quantum topology on $\CQ(\CL^\star)$. 

\medskip

Observe that if $\neq_\mq$ is classical, then quantum topologies on $(X, \neq_\mq)$ coincides with usual topologies on $X$, and quantum homeomorphisms are precisely ordinary homeomorphisms. 


\medskip

In the following, we consider a particular kind of quantum topological spaces. 
Let $B$ be a $C^*$-algebra and $\KP^B$ be the set of all pure states on $B$. 
For $\phi\in \KP^B$, we denote by $\bs_\phi\in \CP_{B^{**}}$ the support projection of $\phi$ (i.e. $\bs_\phi$ is the smallest projection in $B^{**}$ with $\phi(\bs_\phi) = 1$). 
We will equip $\KP^B$ with the following q-distinctness relation: 
$$\phi\neq_\mo \psi \quad  \text{if and only if} \quad \bs_\phi\bs_\psi=0;$$ 
or equivalently, the transition probability between $\phi$ and $\psi$ is zero. 


\medskip

For a closed left ideal $L\subseteq B$, we set 
$$\hull(L):= \big\{\phi\in \KP^B: L \subseteq L_\phi \big\},$$
where $L_\phi:=\{b\in B: \phi(b^*b) =0 \}$. 
It is well-known that $L = \bigcap \{L_\phi: \phi \in \hull(L)\}$. 
Let us denote 
$$\CC^B:= \{\hull(L): L \text{ is a closed left ideal of }B \}.$$ 
We call $(\KP^B, \neq_\mo, \CC^B)$ the \emph{quantum spectrum} of $B$. 

\medskip

One can rewrite the above quantum topological space in terms of modular maximal left ideals of $B$ (thanks to \cite[Theorem 5.3.5]{Mur}). 
However, be aware that in this case, the correct q-distinctness relation \emph{is not} the one given by $L_1 L_2^* = \{0\}$. 

\medskip

In the following, we will show that $\CC^B$ is a quantum topology.
Indeed, it is clear that $\CC^B$ satisfies Conditions (S1) and (S2). 
%Notice, however, that if $C_1,C_2\in \CC^B$, then $C_1\cup C_2$ may not be an element in $\CC^B$. 
%The reason is that if $L_1,L_2\subseteq B$ are closed left ideals, one has $L_1\cap L_2\subseteq L_1\cdot L_2\subseteq L_2$, but, in general, the inclusion $L_1\cdot L_2\subseteq L_1$ fails.
In the following, we will verify that all elements in $\CC^B$ are quantum subsets and that $\CC^B$ satisfies Condition (S3).  
To do these, we need the following facts. 

\medskip

\begin{lem}\label{lem:PS-min-proj}
Let $B$ be a $C^*$-algebra. 

\smnoind
(a) There is an order reversing bijection from $\CC(B)$ to the set of all closed left ideals of $B$, that associates $p\in \CC(B)$ with $L_p:= B^{**}_\ba(1-p) \cap B$, such that $p\in \CC(B)$ is central if and only if $L_p$ is an ideal.

\smnoind
(b) The assignment $\phi\mapsto \bs_\phi$ is a quantum bijection from $\KP^B$ onto $\CP_{B^{**}}^\mm$, when $\CP_{B^{**}}^\mm$ is equipped with the q-distinctness relation induced from $\CP_{B^{**}_\ba}^\star$ (see Proposition \ref{prop:atom-hered-q-distinct}). 

\smnoind
(c) $p\mapsto \KP^B[p]:= \{\phi\in \KP^B: \bs_\phi\leq p \}$ is an ortholattice isomorphism from $\CP_{B^{**}_\ba}$ onto $\CQ(\KP^B)$.  

\smnoind
(d) $\CC^B = \big\{\KP^B[q]: q\in \CC(B) \big\}$. 

\smnoind
(e) For any $p,q\in \CP_{B^{**}_\ba}$, the two quantum subsets $\KP^B[p]$ and $\KP^B[q]$ q-commute if and only if $pq = qp$. 
\end{lem} 
\begin{proof}
(a) This part is well-known (see, e.g., \cite[Theorem II.17]{Ake69} and \cite[\S 3.11.10]{Ped79}). 

\smnoind
(b) This part follows from the definitions of the two q-distinctness relations. 

\smnoind
(c) Note that $p\mapsto \CP_{B^{**}}^\mm[p]$ (see \eqref{eqt:def-CP[p]}) is an ortholattice isomorphism from $\CP_{B^{**}_\ba}$ onto $\CQ(\CP_{B^{**}}^\mm)$ because of the proof of Proposition \ref{prop:atom-hered-q-distinct}(b). 
The conclusion then follows from part (b) above.  

\smnoind
(d) This part follows from parts (a) and (b) as well as the fact that $L_\phi = L_{\bs_\phi}$, for any $\phi\in \KP^B$. 

\smnoind
(e) This follows from part (b) as well as the well-known fact that for any $p,q\in \CP_{B^{**}_\ba}$, one has $pq = qp$ if and only if $p- p\wedge q$ is orthogonal to $q - p\wedge q$.  
\end{proof}

\begin{comment}
\medskip

\begin{cor}\label{cor:q-cent-quan-sp}
Let $B$ be a $C^*$-algebra and $S\subseteq \KP^B$. 
Then $S$ is a q-central elements in $\CQ(\KP^B)$ if and only if $S^\mc = \KP^B\setminus S$. 
\end{cor}
\begin{proof}
If $S$ is a q-central element in $\CQ(\KP^B)$, then it follows from Lemma \ref{lem:q-comm}(a) that $S^\mc = \KP^B\setminus S$ (note that $\KP^B$ is atomic). 
Conversely, suppose that $S^\mc = \KP^B\setminus S$.
Then $S = S^{\mc\mc}$, and Lemma \ref{lem:q-comm}(a) tells us that $S$ commutes with all the atoms in $\CQ(\KP^B)$. 
Lemma \ref{lem:PS-min-proj}(c) gives a projection $p\in \CP_{B^{**}_\ba}$ such that $S = \KP^B[p]$. 
The above, together with parts (c) and (e) of Lemma \ref{lem:PS-min-proj}, then implies that $pe = ep$ for every $e\in \CP_{B^{**}}^\mm$. 
Consequently, $pq = qp$ for every $q\in \CP_{B^{**}_\ba}$. 
Now, we know from parts (c) and (e) of Lemma \ref{lem:PS-min-proj} that $S= \KP^B[p]$ is a q-central element in $\CQ(\KP^B)$. 
\end{proof}

\end{comment}


\medskip

We also need the following ``atomic version'' of \cite[Theorem II.7]{Ake69}. 
This result actually follows from the argument of \cite[Theorem II.7]{Ake69} (recall that $\sigma(\bz^A\cdot A^*, A)$-closed left $A$-invariant subspaces of $\bz^A\cdot A^*$  are in bijective correspondence with closed left ideals of $A$; where $\bz^A$ is the central projection as in \eqref{eqt:def-Lambda}). 

\medskip

\begin{lem}\label{lem:sum-orth-closed}
	Let $A$ be a $C^*$-algebra. 
	If $p,q\in \CC(A)$ satisfying $\|p(q-p\wedge q)\|< 1$, then $p\vee q \in \CC(A)$. 
\end{lem}



\medskip

\begin{prop}\label{prop:quantum top}
Let $B$ be a $C^*$-algebra.

\smnoind
(a) The q-distinctness relation $\neq_\mo$ on $\KP^B$ is both atomic and hereditary. 

\smnoind
(b) $(\KP^B, \neq_\mo, \CC^B )$ is a quantum topological space. 

\smnoind
(c) Suppose that $\KG: \KP^B\to \widehat{B}$ is the surjection that sends $\omega\in \KP^B$ to the equivalence class $[\pi_\omega]_\sim$ of its GNS construction $\pi_\omega$. 
Then 
$$\big\{\KG^{-1}([\pi]_\sim): [\pi]_\sim\in\widehat{B} \big\}$$ 
is the collection of all minimum q-central elements in $\CQ(\KP^B)$. 
Moreover, $Z\mapsto \KG^{-1}(Z)$ is a bijective correspondence from the collection of closed (respectively, open) subsets of $\widehat{B}$ to the collection of q-central elements in $\CC^B$ (respectively, q-central quantum open subsets of $\KP^B$).
\end{prop}
\begin{proof}
(a) This is a direct consequence of Lemma \ref{lem:PS-min-proj}(b) and Proposition \ref{prop:atom-hered-q-distinct}(b).

\smnoind
(b) Notice that elements in $\CC^B$ are quantum subsets because of parts (c) and (d) of Lemma \ref{lem:PS-min-proj}. 
It remains to verify Condition (S3). 
Suppose that $C_1,C_2\in \CC^B$ such that $C_1$ q-commutes with $C_2$. 
Lemma \ref{lem:PS-min-proj}(d) produces $p_1,p_2\in \CC(B)$ with $C_k=\KP^B[p_k]$ ($k=1,2$), and we know from Lemma \ref{lem:PS-min-proj}(e) that $p_1p_2 = p_2 p_1$. 
By Lemma \ref{lem:sum-orth-closed}, the projection $p_1 \vee p_2 = p_1 + p_2 - p_1p_2$ belongs to $\CC(B)$. 
Now, parts (c) and (d) of Lemma \ref{lem:PS-min-proj} implies that $C_1\vee C_2 = \KP^B[p_1\vee p_2]\in \CC^B$. 

\smnoind
(c) For an irreducible representation $(\pi, \KH)$ of $B$, we denote by $\ti \pi: B^{**}_\ba \to \CB(\KH)$ the weak$^*$-continuous extension of $\pi$. 
It is well-known that 
$$[\pi]_\sim \mapsto \ker \ti \pi$$ 
is a bijection from $\widehat{B}$ to the set of  maximal weak$^*$-closed ideals of $B^{**}_\ba$.
Moreover, maximal weak$^*$-closed ideals of $B^{**}_\ba$ are of the form $B^{**}_\ba (1-p)$ for a minimum central projection $p\in \CP_{B^{**}_\ba}$. 
On other hand, by parts (a), (c) and (e) of Lemma \ref{lem:PS-min-proj}, the set of minimum q-central elements in $\CQ(\KP^B)$ is precisely  
$$\big\{\KP^B[p]: p\in \CP_{B^{**}_\ba} \text{ is a minimum central projection}\big\}.$$ 
Consider a minimum central projection $p\in \CP_{B^{**}_\ba}$ and  $\phi \in \KP^B$.
One has $\bs_\phi \in \CP_{B^{**}_\ba}[p]$ (i.e, $\bs_\phi \leq p$) if and only if 
$$B^{**}_\ba(1-p)\subseteq \{x\in B^{**}_\ba: \phi(x^*x) =0 \},$$
which is equivalent to $B^{**}_\ba(1-p)\subseteq \ker \ti \pi_\phi$. 
However, as both $B^{**}_\ba(1-p)$ and $\ker \ti \pi_\phi$ are maximal weak$^*$-closed ideals of $B^{**}_\ba$, we know that they are the same. 
This gives the first statement. 

Let us consider $q\in \CC(B)$ to be a central projection, and let $L_q:= B^{**}_\ba(1-q)\cap  B$ be the corresponding closed ideal. 
For any $\phi\in \KP^B$, as in the above, one has $\phi\in \KP^B[q]$ if and only if $L_q \subseteq \ker \pi_\phi$.  
This, together with  Lemma \ref{lem:PS-min-proj}(d), gives the bijectivity between closed subsets of $\widehat{B}$ and q-central quantum closed subsets of $\KP^B$.

Finally, part (a) above and Corollary \ref{cor:q-cent-hered}(c) tell us that the q-complement of a q-central element in $\KP^B$ coincides with its ordinary complement and is also q-central. 
Thus, the statement concerning open subset of $\widehat{B}$ follows from the statement concerning closed subset of $\widehat{B}$. 
\end{proof}

\medskip


\begin{prop}
Let $A$ and $B$ be two $C^*$-algebras. 
If $\Theta: B\to A$ is a Jordan $^*$-isomorphism, then $\Theta^*|_{\KP^A}: \KP^A \to \KP^B$ is a quantum homeomorphism. 
\end{prop}

\medskip

Indeed, it is well-known that $\Theta^*(\KP^A) = \KP^B$ and $\Theta^*$ respects the q-distinctness relations. 
Moreover, as $\Theta^{**}(\CC(A)) = \CC(B)$, we know from Lemma \ref{lem:PS-min-proj}(d) that  $\Theta^*|_{\KP^A}(\CC^A) = \CC^B$. 

\medskip

Now, we can present the main theorem of this section. 
Let us denote by $\widehat{A}^2$ the set of all irreducible $^*$-representations of a $C^*$-algebra $A$ with dimensions dominated  by $2$ (i.e. including those one dimensional ones). 
Set $A_0:=\bigcap_{\pi\in \widehat{A}^2}\ker \pi$, and consider $j_A: \KP^{A_0}\to \KP^A$ to be the canonical map. 

\medskip 


\begin{thm}\label{thm:main2}
Let $A$ and $B$ be $C^*$-algebras, and $\Psi: \KP^A \to \KP^B$ be a quantum homeomorphism.

\smnoind
(a) If $\BM_2$ is not a quotient $C^*$-algebra of $A$, then there is a unique Jordan $^*$-isomorphism $\Gamma: B\to A$ such that $\Psi = \Gamma^*|_{\KP^A}$. 

\smnoind
(b) There is a Jordan $^*$-isomorphism $\Gamma_0: B_0\to A_0$ with $\Psi\circ j_A = j_B\circ \Gamma_0^*|_{\KP^{A_0}}$. 
\end{thm}
\begin{proof}
(a) For $p\in \CC(A)$, we know from the assumption on $\Psi$ and Lemma \ref{lem:PS-min-proj}(d) that there is a unique element $\Phi(p)\in \CC(B)$ satisfying
$$\KP^B\big[\Phi(p)\big]:= \Psi\big(\KP^A[p]\big).$$
The equality $\CC^B = \{\Psi(C): C\in \CC^A \}$ implies that  $\Phi$ is a bijection from $\CC(A)\setminus \{0\}$ onto $\CC(B)\setminus \{0\}$. 
Moreover, as $\Psi$ preserves the q-distinctness relations, we know that $\Phi$ is a quantum bijection. 
By Theorem \ref{thm:main}, there is a Jordan $^*$-isomorphism $\Theta:A\to B$ such that $\Theta^{**}(p) = \Phi(p)$ for every $p\in \CC(A)$. 
When $\psi\in \KP^B$, one has $\Theta^*(\psi)\in \KP^A$ and $\bs_{\Theta^*(\psi)} = \Phi^{-1}(\bs_\psi)$. 
Thus, if we set $\Gamma:= \Theta^{-1}$, then the required equality follows from Lemma \ref{lem:PS-min-proj}(b). 


\smnoind
(b) Let $\KP^A_2:= \KG^{-1}(\widehat{A}^2)$ (see Proposition \ref{prop:quantum top}(c)), and set 
$$\KP^A_0 := \KP^A\setminus \KP^A_2.$$ 
By \cite[Proposition 4.4.10]{Ped79}, $\widehat{A}^2$ is a closed subset of $\widehat{A}$. 
Hence, Proposition \ref{prop:quantum top}(c) tells us that $\KP^A_2$ is a q-central element in $\CC^A$. 
Moreover, $\widehat{A}\setminus \widehat{A}^2$ is homeomorphic to $\widehat{A_0}$, under the canonical $^*$-homomorphism $\Delta_A: A \to M(A_0)$. 
Thus, $\omega \mapsto \omega\circ \Delta_A$ induces a bijection $\ti \Delta_A: \KP^{A_0} \to \KP^A_0$. 
Since $\ti \Delta_A$ preserves support projections (when we consider $A_0^{**}\subseteq A^{**}$ in the canonical way), it preserves the q-distinctness relations in both directions. 
Furthermore, as hereditary $C^*$-subalgebras of $A_0$ are precisely hereditary $C^*$-subalgebras of $A$ that are contained in $A_0$, we know that $\ti \Delta_A$ gives a bijection between quantum open subsets of $\KP^{A_0}$ and quantum open subsets of $\KP^A$ that are contained in quantum open subset $\KP^A_0$. 

On the other hand, by Proposition \ref{prop:quantum top}(c), if $\omega\in \KP^A$, then the set of elements whose GNS constructions equal to that of $\omega$ is the minimum q-central element in $\CQ(\KP^A)$ containing $\omega$. 
As $\Psi$ preserves the q-distinctness relations in both directions, it will preserve the minimum q-central quantum subsets containing the corresponding elements in the respective quantum spectra. 
Consider $\omega\in \KP^A_2$. 
The dimension of the GNS construction of $\omega$ is either $1$ or $2$. 
If it is one dimensional, then $\{\omega \}$ is a q-central quantum subset and so is $\{\Psi(\omega) \}$, which implies that the GNS construction of $\Psi(\omega)$ is one-dimensional. 
Suppose that the GNS construction of $\omega$ is two dimensional.
Then there exists a unique element $\omega^\bot$ in the minimum q-central quantum subset containing $\omega$ such that $\omega\neq_\mo\omega^\bot$. 
From this, one  can find a unique element $\Psi(\omega)^\bot$ in the minimum q-central quantum subset containing $\Psi(\omega)$ with $\Psi(\omega)\neq_\mo \Psi(\omega)^\bot$. 
Therefore, $\Psi(\KP^A_2) \subseteq \KP^B_2$. 
By symmetry, we know that $\Psi(\KP^A_2) = \KP^B_2$.

Finally, it is not hard to check, via the maps $\ti \Delta_A$ and $\ti\Delta_B$, that $\Psi$ induces a quantum homeomorphism $\bar\Psi: \KP^{A_0} \to \KP^{B_0}$. 
The conclusion then follows from part (a).  
\end{proof}

\medskip

A similar argument as part (b) above also show that $\Psi$ induces a quantum homeomorphism from $\KP^A_2$ onto $\KP^B_2$. 
However, this restriction map may not come from a Jordan $^*$-isomorphism (see Example \ref{eg:M2} below). 


\medskip

\begin{rem}
If one wants a stronger conclusion of a $^*$-isomorphism in Theorem \ref{thm:main2}(a), one may need to add the ``orientation structure'' as in \cite{AHS} and \cite{Shu82}. 
More precisely, consider $\omega_1, \omega_2\in \KP^A$ to be  distinct elements such that they have the same GNS construction. 
As $A$ has no two dimensional irreducible representation, one can find $\omega_3\in \{\omega_1, \omega_2\}^\mc \subseteq \KP^A$ such that the GNS construction of $\omega_3$ coincides with those of $\omega_1$ and $\omega_2$. 
Then the corresponding cyclic vectors of $\omega_1, \omega_2, \omega_3$ span a 3 dimensional subspaces, and thus $\{\omega_1, \omega_2, \omega_3 \}^{\mc\mc}$ is quantum homeomorphic to $\CP_{\BM_3}$. 
By Theorem \ref{thm-Dye}, any quantum bijection from $\CP_{\BM_3}\setminus \{0\}$ to itself is induced by a Jordan $^*$-isomorphism from $\BM_3$ to itself. 
In this case, we can formulate orientation in terms of quantum bijections, and it is valid to say whether $\Psi$ preserves the orientation of $S^2(\omega_1,\omega_2)$.
However, this seems a bit complicated and we will not go into the details here.  
\end{rem}


\medskip

\begin{cor}\label{cor:main2}
Let $A$ and $B$ be $C^*$-algebras. 
Suppose that  there is a quantum homeomorphism $\Psi: \KP^A \to \KP^B$.

\smnoind
(a) If $A$ is a primitive $C^*$-algebra such that $\BM_2$ is not a quotient $C^*$-algebra of $A$, then there is a map $\Theta: A\to B$ which is either a $^*$-isomorphism or a $^*$-anti-isomorphism such that $\Psi^{-1} = \Theta^*|_{\KP^B}$. 

\smnoind
(b) If $A = \BM_2$, then $A$ and $B$ are either $^*$-isomorphic or $^*$-anti-isomorphic. 
\end{cor}
\begin{proof}
(a) By Theorem \ref{thm:main2}(a), it suffices to show that any Jordan $^*$-isomorphism $\Gamma: B\to A$ is either a $^*$-isomorphism or a $^*$-anti-isomorphism. 
In fact, as $A$ is primitive, there exists a faithful irreducible $^*$-representation $\mu$ of $A$. 
Note that $\Gamma^{**}$ restricts to a Jordan $^*$-isomorphism from $B^{**}_\ba$ onto $A^{**}_\ba$. 
Thus, one can find an irreducible $^*$-representation $\nu$ of $B$ such that $\Gamma^{**}$ will send $\CB(\KH_\nu)\subseteq B^{**}_\ba$ onto $\CB(\KH_\mu)$. 
As $\mu$ is injective and $\Gamma^{**}|_{\CB(\KH_\nu)}$ is either a $^*$-isomorphism or a $^*$-anti-isomorphism, the conclusion follows.


\smnoind
(b) Notice that 
$$\begin{equation}\label{eqt:q-closed-M2}
\CC^{\BM_2} = \{\emptyset\} \cup \big\{\KP^{\BM_2} \big\} \cup \big\{\{\omega\}: \omega\in \KP^{\BM_2} \big\}.
\end{equation}$$
For every $\omega \in \KP^{\BM_2}$, there exists exactly one $\omega^\bot\in \KP^{\BM_2}$ with  $\omega \neq_\mo \omega^\bot$. 
From this, and the existence of a quantum bijection from $\KP^{\BM_2}\setminus \{0\}$ to $\KP^B\setminus \{0\}$, we see that $B$ has exactly one irreducible $^*$-representation, and that this representation is  of dimension $2$ (see the proof of Theorem \ref{thm:main2}(b)). 
Consequently,  $B\cong \BM_2$. 
\end{proof}

\begin{comment}
\medskip

It is believed that a weaker conclusion than that of Theorem \ref{thm:main2} is true when the assumption of $A$ not having a quotient $C^*$-algebra of the form $\BM_2$ is removed. 
Let us state this clearly as a conjecture. 


\medskip

\begin{conj}\label{conj}
Let $A$ and $B$ be two $C^*$-algebras. 
If there is a quantum homeomorphism $\Psi: \KP^A \to \KP^B$, then $A$ and $B$ are Jordan $^*$-isomorphic. 
\end{conj}


\medskip


An evidence of this conjecture is that if there is a quantum bijection between the projection lattices of two von Neumann algebras, then the two algebras are Jordan $^*$-isomorphic (even if they have type $\mathrm{I}_2$ summands). 
However, as in the von Neumann algebra situation, the map $\Psi$ may not be induced by a Jordan $^*$-isomorphism.


\end{comment}

\medskip

\begin{eg}\label{eg:M2}
	Fix an element $\omega_1\in \KP^{\BM_2}$ and consider $\omega_1^\bot$ to be an element in $\KP^{\BM_2}$ as in the proof of Corollary \ref{cor:main2}(b). 
	Define $\Psi:\KP^{M_2}\to \KP^{M_2}$ such that $\Psi(\omega_1) = \omega_1^\bot$, $\Psi(\omega_1^\bot) = \omega_1$ and $\Psi(\omega) = \omega$ when $\omega\notin \{\omega_1, \omega_1^\bot\}$. 
	Then $\Psi$ is a quantum homeomorphism (see \eqref{eqt:q-closed-M2}), but it cannot be induced by a Jordan $^*$-automorphism on  $\BM_2$ (because $\Psi$ cannot be extended to an affine map on the state space of $\BM_2$). 
\end{eg}



\medskip


\section*{Acknowledgement}

\medskip

The authors are supported by the National Natural Science Foundation of China (11871285), and the second named author is also supported by China Scholarship Council (201906200101). 
The authors would also like to thank C.A. Akemann and L.G. Brown for some discussions on Proposition \ref{prop:atom-Thm2.2-APT}. 

%\medskip

\begin{thebibliography}{99}

\bibitem{Adams}
D.H. Adams, The completion by cuts of an orthocomplemented modular lattice, Bull. Austral. Math. Soc. \textbf{1} (1969), 279-280.


%\bibitem{AS98}
%E.M. Alfsen and F.W. Shultz, On orientation and dynamics in operator algebras I, Comm. Math. Phys. \textbf{194} (1998), 87-108. 

%\bibitem{AV}
%D. Aerts and B. Van Steirteghem, Quantum axiomatics and a theorem of M. P. Sol\`{e}r, Int. J. Theoret. Phys. \textbf{39} (2000), 497-502. 

\bibitem{AHS}
E.M. Alfsen, H. Hanche-Olsen and F.W. Schultz, State spaces of $C^*$--algebras, Acta Math. \textbf{144} (1980), 267-305. 
 
\bibitem{Akemann68}
C.A. Akemann, Sequential convergence in the dual of a von Neumann algebra, Comm. Math. Phys. \textbf{7} (1968), 222--224.

\bibitem{Ake69}
C.A. Akemann, The General Stone-Weierstrass Problem, J. Funct. Anal. \textbf{4} (1969), 277--294.

\bibitem{Ake71}
C.A. Akemann, A Gelfand representation theory for $C^*$-algebras,
Pacific J. Math. \textbf{39} (1971), 1--11.

\bibitem{APT73}
C.A. Akemann, G.K. Pedersen and J. Tomiyama, Multipliers of $C^*$-algebras,
J.\ Funct.\ Anal., \textbf{13} (1973), 277-301.

\bibitem{Ban}
T. Banica, Quantum automorphism groups of homogeneous graphs, J. Funct. Anal. \textbf{224} (2005), 243-280. 

\bibitem{BBC}
T. Banica, J. Bichon and G. Chenevier, Graphs having no quantum symmetry, Ann. Inst. Fourier (Grenoble) \textbf{57} (2007), 955-971. 

%\bibitem{BBCo}
%T. Banica, J. Bichon and B. Collins, The hyperoctahedral quantum group, J. Ramanujan Math. Soc. \textbf{22} (2007), 345-384.


\bibitem{BGS}
J. Bhowmick, D. Goswami and A. Skalski, Quantum isometry groups of 0-dimensional manifolds, Trans. Amer. Math. Soc. \textbf{363} (2011), 901-921.

\bibitem{Brown14}
L.G. Brown, Large $C^*$-algebras of universally measurable operators, Quart. J. Math. \textbf{65} (2014), 851-855. 

\bibitem{BW92}
L.J. Bunce and J.D.M. Wright, The Mackey-Gleason problem, Bull. Amer. Math. Soc. (N.S.) \textbf{26} (1992), 288--293. 

\bibitem{BW93}
L.J. Bunce and J.D.M. Wright, 
On Dye's theorem for Jordan operator algebras, 
Exposition. Math. \textbf{11} (1993), 91--95. 

\bibitem{DRZ}
C. Dietzel, W. Rump and X. Zhang, One-sided orthogonality, orthomodular spaces, quantum sets, and a class of Garside groups, J. Algebra \textbf{526} (2019), 51-80. 


\bibitem{Dye}
H.A. Dye, On the geometry of projections in certain operator algebras, Ann. Math. \textbf{61} (1955), 73--89.

%\bibitem{Faure}
%C.A. Faure,  Categories of closure spaces and corresponding lattices, Cahier Top. geom. Diff. Categ. \textbf{35} (1994), 309-319.

\bibitem{FR}
U.A. First and T. R\"{u}d, On Uniform admissibility of unitary and smooth representations, Arch. Math. \textbf{112} (2019), 169-179. 

\bibitem{GK}
R. Giles and H. Kummer, A non-commutative generalization of topology, Indiana Univ. Math. J. \textbf{21} (1971/72), 91-102.


\bibitem{Gud}
S. Gudder, Measure and integration in quantum set theory, in  \emph{Current issues in quantum logic (Erice, 1979)},
Ettore Majorana Internat. Sci. Ser.: Phys. Sci. \textbf{8}, Plenum, New York-London (1981), 341-352. 

\bibitem{Ham81}
M. Hamana, Regular embeddings of $C^*$-algebras in monotone complete $C^*$-algebras, 
J. Math. Soc. Japan \textbf{33} (1981), 159-183. 

%\bibitem{Ham03}
%J. Hamhalter, \emph{Quantum Measure Theory}, Kluwer (2003).

\bibitem{Ham}
J. Hamhalter, Dye's theorem and Gleason's theorem for $AW^*$-algebras, J. Math. Anal. Appl. \textbf{422} (2015), 1103--1115. 


%\bibitem{Holl}
%S.S., Jr. Holland,  Orthomodularity in infinite dimensions; a theorem of M. Sol\`{e}r, Bull. Amer. Math. Soc. \textbf{32} (1995), 205-234.

\bibitem{Kal83}
G. Kalmbach, \emph{Orthomodular lattices}, Lond. Math. Soc. Mono. \textbf{18}, Academic Press (1983). 


\bibitem{Kal98}
G. Kalmbach, \emph{Quantum measures and spaces}, Math. and its Appl. \textbf{453}, Kluwer Academic Publishers  (1998).

\bibitem{LNW16}
C.W. Leung, C.K. Ng and N.C. Wong, Transition probabilities of normal states determine the Jordan structure of a quantum system, J. Math. Phys. \textbf{57} (2016), 015212. 

\bibitem{LNW17}
C.W. Leung, C.K. Ng and N.C. Wong, The positive contractive part of a noncommutative $L^p$-space is a complete Jordan invariant, Linear Algebra Appl. \textbf{519} (2017), 102--110. 

%\bibitem{LTW}
%C.W. Leung, C.W. Tsai and N.C. Wong, Linear q-distintness preservers of von Neumann algebras, Math. Z. \textbf{270} (2012), 699--708. 

\bibitem{MacN}
H.M. MacNeille, Partially ordered sets, Trans. Amer. Math. Soc. \textbf{42} (1937), 416-460.

\bibitem{Miers}
C.R. Miers, Lie isomorphisms of factors, Trans. Amer. Math. Soc. \textbf{147} (1970), 55--63. 

\bibitem{Mihara}
T. Mihara, Characterisation of the Berkovich spectrum of the Banach algebra of bounded continuous functions, Doc. Math. \textbf{19} (2014), 769-799. 

\bibitem{Mur}
G.J. Murphy, \emph{$C^*$-algebras and operator theory}, Academic Press (1990).

\bibitem{MRV}
B. Musto, D. Reutter, D. Verdon, The Morita theory of quantum graph isomorphisms, Comm. Math. Phys. \textbf{365} (2019), 797-845. 


\bibitem{NW}
C.K. Ng and N.C. Wong, A Murray-von Neumann type classification of $C^*$-algebras, in \emph{Operator Semigroups Meet Complex Analysis, Harmonic Analysis and Mathematical Physics, Herrnhut, Germany (in honor of Prof. Charles Batty for his 60th birthday)}, Operator Theory: Advance and Applications, \textbf{250}, Springer Internat. Publ. (2015), 369--395.

\bibitem{Ped72}
G.K. Pedersen, Applications of weak$^*$-semicontinuity in $C^*$-algebra theory, Duke Math. J. \textbf{39} (1972), 431-450.

\bibitem{Ped79}
G.K. Pedersen,
\emph{$C^*$-algebras and their automorphism groups},
Academic Press (1979).

%\bibitem{Ros}
%K.I. Rosenthal, \emph{Quantales and their applications}, Pitman Res. Notes in Math. Series \textbf{234}, Longman and John Wiley \& Sons (1990).

%\bibitem{SW}
%K. Sait\^{o} and J.D.M. Wright, \emph{Monotone complete $C^*$-algebras and generic dynamics}, Springer London (2015). 

\bibitem{Schles}
K.-G. Schlesinger, Toward quantum mathematics I: From quantum set theory to universal quantum mechanics, J. Math. Phys. \textbf{40} (1999), 1344-1358.

\bibitem{Sch}
S. Schmidt, The Petersen graph has no quantum symmetry, Bull. Lond. Math. Soc. \textbf{50} (2018), 395-400. 


\bibitem{Shu82}
F.W. Shultz, Pure states as a dual object for $C^*$-algebras, Comm. Math. Phys. \textbf{82} (1981/82), 497-509. 

%\bibitem{Sol}
%M.P. Sol\`{e}r, Characterization of Hilbert spaces by orthomodular spaces, Comm. Algebra \textbf{23} (1995), 219-243. 

\bibitem{SV07}
I. Stubbe and B. Van Steirteghem, Propositional systems, Hilbert lattices and generalized Hilbert spaces, in \emph{Handbook of quantum logic and quantum structures}, Elsevier Sci. B. V. (2007), 477-523. 




%\bibitem{Take}
%M. Takesaki, \emph{Theory of operator algebras I}, Springer-Verlag (1979).

\bibitem{Takeuti}
G. Takeuti, Quantum set theory, in \emph{Current issues in quantum logic (Erice, 1979)}, 
Ettore Majorana Internat. Sci. Ser.: Phys. Sci., \textbf{8}, Plenum, New York-London (1981), 303-322.

\bibitem{TK}
S. Titani and H. Kozawa, Quantum set theory, Internat. J. Theoret. Phys. \textbf{42} (2003), 2575-2602.

\bibitem{Vlad}
D.A. Vladimirov, \emph{Boolean algebras in analysis}, Kluwer Academic (2002).

\bibitem{Walker}
J. W. Walker, From graphs to ortholattices and equivariant maps, 
J. Combin. Theory Ser. B \textbf{35} (1983), 171-192.

\bibitem{Yen}
T. Yen, Isomorphism of $AW^*$-algebras, Proc. Amer. Math. Soc. \textbf{8} (1957), 345--349. 
\end{thebibliography}

\end{document}
