%%%%%%% TEMPLATE VERSION OF 4 MAY, 2020.

%%%%%%%%%%%%%%%%%%%%%%%%%% REMINDERS %%%%%%%%%%%%%%%%%%%%%%%%%%%%%%%%%%%%%%%%%

%%%% VARIOUS DOTS

%\dotsc for "dots with commas"
%\dotsb for "dots with binary operators/relations"
%\dotsm for "multiplication dots"
%\dotsi for "dots with integrals"
%\dotso for "other dots" (none of the above)

%%%% SIZES

%\tiny
%\scriptsize
%\footnotesize
%\small
%\normalsize
%\large
%\Large
%\LARGE
%\huge
%\Huge

%%%%%%%%%%%%%%%%%%%%%%%%%% DOCUMENT CLASS %%%%%%%%%%%%%%%%%%%%%%%%%%%%%%%%%%%%%

% Choosing the draft option will mark overfull hboxes in the output.

\documentclass{amsart}
%\documentclass[draft]{amsart}

%\documentclass{article}
%\documentclass[draft]{article}

%\documentclass{report}
%\documentclass[draft]{report}

%\documentclass{letter}
%\documentclass[draft]{letter}

%%%%%%%%%%%%%%%%%%%%%% PREVENTING UNWANTED STOPPING %%%%%%%%%%%%%%%%%%%%%%%%%%%%%%%%%%%%

% The following two commands prevent LaTeX from stopping because there are too many math alphabets.
\newcommand\hmmax{0}
\newcommand\bmmax{0}

% Preventing stopping because counters are already defined
\let\counterwithout\relax
\let\counterwithin\relax

%%%%%%%%%%%%%%%%%%%%%%%%%%%    STANDARD PACKAGES      %%%%%%%%%%%%%%%%%%%%%%%%%%%%%%%


% Language

\usepackage[english]{babel} % \usepackage[UKenglish]{babel} yielded en error (why ?)
\usepackage[UKenglish]{datetime}

% Other packages

\usepackage{ae, aecompl}
\usepackage{amscd}
\usepackage{amsfonts}
\usepackage{amsmath}
\usepackage{amssymb}
\usepackage{amsthm}
\usepackage{amsxtra}
\usepackage{cases}
\usepackage{chngcntr}
\usepackage{cite}
\usepackage{color}
\usepackage{graphicx}
\usepackage{latexsym}
\usepackage{bbm} % extra blackboard symbols, serif and sans serif
\usepackage{bm}
\usepackage{bbold}
\usepackage{mathtools}
\usepackage{microtype}
\usepackage{qsymbols}
\usepackage[Symbolsmallscale]{upgreek} % for upright \pi
%\usepackage[Symbol]{upgreek} %\for other upright \pi
\usepackage[active]{srcltx}
\usepackage{tikz}
\usepackage{url}
\usepackage{verbatim} % use \begin{verbatim}...\end{verbatim}
\usepackage[all,cmtip]{xy} % generates warning 

% Packages for referencing. Cleveref must be invoked last.

%\usepackage{hyperref}
\usepackage{enumitem}
\usepackage{cleveref}


%%%%%%%%%%%%%%%%%%%%%%%%%% BIBLIOGRAPHY %%%%%%%%%%%%%%%%%%%%%%%%%%%%%%%%%%%%%%%%

% \usepackage {amsrefs}
% Use the following version to prevent dashes in case of repeated (combinations of) authors.
 \usepackage[nobysame]{amsrefs}

% Suppress MR number by the following command.
 \AtBeginDocument{\def\MR#1{}}

%%%%%%%%%%%%%%%%% WHILE WORKING ON THE PAPER %%%%%%%%%%%%%%%%%%%%%%%%%%%%%%%%

%%%%%%%%%%% OVERFULL HBOXES

% Choosing the draft option in documentclass will show overfull hboxes in the ouput

%%%%%%%%%% PAGE FRAMES

% Show page frames in the output
%\usepackage{showframe}

%%%%%%%%%% SHOW LABELS

% Display labels in output with the following commands; the second is user-friendlier.
%\usepackage{showkeys}
%\usepackage[inline]{showlabels}

%%%%%%%%% SHOW FILENAME,  AND TIME OF COMPLATION, AND PAGE NUMBER IN THE HEADER 

\setlength{\headheight}{12pt}
\usepackage{currfile}
\usepackage{fancyhdr}
\pagestyle{fancy}
\fancyhf{}
\setlength{\headheight}{19pt}
\chead{\footnotesize{{File: \currfilebase\\ Compiled: \today,\,\currenttime}}}
\rhead{\small{\thepage}}

%%%%%%%%% WATERMARK %%%%%%%%%%%%%%%%%%%%%%

%\usepackage[firstpage=true]{background}
%
%\backgroundsetup{
%	contents=SUBMITTED,
%	angle=45,
%	scale=12,
%	color=black,
%	opacity=0.1
%}

%%%%%%%%% LINE NUMBERING FOR PROOFREADING

% The following block is needed to make line numbering also work for environments such as align.
% Otherwise these are nor numberd  
% Sometimes this gives a clustering of two line numbers for one line at the end of such an environment.
\newcommand*\patchAmsMathEnvironmentForLineno[1]{%
  \expandafter\let\csname old#1\expandafter\endcsname\csname #1\endcsname
  \expandafter\let\csname oldend#1\expandafter\endcsname\csname end#1\endcsname
  \renewenvironment{#1}%
     {\linenomath\csname old#1\endcsname}%
     {\csname oldend#1\endcsname\endlinenomath}}% 
\newcommand*\patchBothAmsMathEnvironmentsForLineno[1]{%
  \patchAmsMathEnvironmentForLineno{#1}%
  \patchAmsMathEnvironmentForLineno{#1*}}%
\AtBeginDocument{%
\patchBothAmsMathEnvironmentsForLineno{equation}%
\patchBothAmsMathEnvironmentsForLineno{align}%
\patchBothAmsMathEnvironmentsForLineno{flalign}%
\patchBothAmsMathEnvironmentsForLineno{alignat}%
\patchBothAmsMathEnvironmentsForLineno{gather}%
\patchBothAmsMathEnvironmentsForLineno{multline}%
}

% The following block gives line numbers in both margins.
%\makeatletter
%\def\makeLineNumberLeft{%
%  \linenumberfont\llap{\hb@xt@\linenumberwidth{\LineNumber\hss}\hskip\linenumbersep}% left line number
%  \hskip\columnwidth% skip over column of text
%  \rlap{\hskip\linenumbersep\hb@xt@\linenumberwidth{\hss\LineNumber}}\hss}% right line number
%\leftlinenumbers% Re-issue [left] option
%\makeatother


% If the lines in the abstract should also be numbered, include the following block AND adapt the abstract part in the front matter:  make sure that the usual \begin{abstract} TEXT \end{abstract} is changed into  \abstract{TEXT}.
%\makeatletter
%\let\my@abstract=\relax
%\def\abstract#1{%
%  \def\my@abstract{%
%    \normalfont\Small
%    \list{}{\labelwidth\z@
%      \leftmargin3pc \rightmargin\leftmargin
%      \listparindent\normalparindent \itemindent\z@
%      \parsep\z@ \@plus\p@
%      \let\fullwidthdisplay\relax
%    }%
%    \item[\hskip\labelsep\scshape\abstractname.]%
%    #1
%  \endlist}}
%\def\@setabstracta{%
%  \ifx\my@abstract\relax
%  \else
%    \skip@20\p@ \advance\skip@-\lastskip
%    \advance\skip@-\baselineskip \vskip\skip@
%  \my@abstract
%    \prevdepth\z@ % because \abstractbox is a vtop
%  \fi
%}
%\makeatother


% Here is the actual activation of the line numbering:
% \usepackage[displaymath,mathlines]{lineno} option displaymath appears to be obsolete
\usepackage[mathlines]{lineno}
\linenumbers

%%%%%%%%%%%%%%%%%% COMMENTING %%%%%%%%%%%%%%%%%%%%%%%%%%%%%%%%%%%%%%%%%%%%%%%%%%%%%%%%%%%%%%%%%

\newcommand{\MdJ}[1]{{\color{purple}MdJ: {#1}}}
\newcommand{\MT}[1]{{\color{red}MT: {#1}}}
\newcommand{\VT}[1]{{\color{blue}VT: {#1}}}

%%%%%%%%%%%%%%%%%%%%%%%  FONTS %%%%%%%%%%%%%%%%%%%%%%%%%%%%%%%%%%%%%%%%%%%%%%%

%%%%%%%%%%% MAIN CHOICES

% Keep these declarations here, as these packages should be loaded after the others (some of the font declarations override one another).

%% Default is ComputerModern. Variations are:

%% Palatino for text with better math support than with its obsolete predecessor mathpple:
% \usepackage{mathpazo}


% Here is the part of the jotart.cls file where the fonts are declared; based on Palatino. Yielded nice results in paper with Nils:
% Fonts with the extended T1 and TS1 (text symbols) encodings
%\usepackage[T1]{fontenc}
%\usepackage{textcomp} 
%% Palatino Adobe by mathpazo, with slanted Greek and CM blackboard bold letters
%\RequirePackage[slantedGreek,noBBpl]{mathpazo}
%[2002/09/08 PSNFSS-v9.0a
% Palatino w/ Pazo Math (D.Puga, WaS)
%]
%\linespread{1.05}
% End of part of jotart.cls 

%% Times for text with math support: 
%\usepackage{mathptmx}

%% Very clear and simple (ComputerModern bright):
% \usepackage{cmbright}

%% Very nice: mathdesign. Options are utopia, charter and garamond. Clashes (as awarning) with amssymb and amsfonts. Garamond option will not work then; charter and utopia still seem fine. Utopia yields a less curly math font than charter; seems to have some trouble with spacing when math is inside a slanted text. Charter option gives good results then, and combines well with mathbbm choice for blackboard font.
% \usepackage[charter]{mathdesign}


% Choice for the mathcal font
% Mathpazo and cmbright leave the default unchanged.
% Mathptmx and mathdesign both yield the same curly font, the latter a bolder version of the former.
% For restoring the default mathcal font when using mathpazo or mathdesign, include the following:
%\let\mathcal\undefined
%\DeclareMathAlphabet{\mathcal}{OMS}{cmsy}{m}{n}

% Choice for the blackboard font
% Standard use is mathbb from the bbold package
% More flexible is the bbm package. Mathbbm is a serif font, mathbbmss is sans serif. Mathbbmtt is typewriter, but does not seem to be generally implemented.

\newcommand{\bbfont}{\mathbbm}

%%%%%%%%% FINE TUNING: TEXT FONT SHAPE FOR MATH IN MAIN TEXT

% Choosing the proper (math) font shape for mathematics in environments: sometimes we want it to follow the current text font shape
% Macro tfs: Text Font Shape
% Has been implemented in macro for C*-algebra(s), but not (yet?) for continuous functions etc.

\usepackage{ifthen}

\makeatletter
\newcommand{\tfs}[1]
{
\ifthenelse{\equal{\f@shape}{n}}{\ensuremath{\mathrm{#1}}}
	{\ifthenelse{\equal{\f@shape}{sc}}{\ensuremath{\mathrm{#1}}}
		{\ifthenelse{\equal{\f@shape}{it}}{\ensuremath{\mathit{#1}}}
			{\ifthenelse{\equal{\f@shape}{sl}}{\ensuremath{\mathit{#1}}}{}	
			}  
		}
	}
}
\makeatother

% Can be undone by the following
%\renewcommand{\tfs}{\mathrm}

%%%%%%%%% FINE TUNING: TEXT FONT SHAPE FOR MATH IN BIBLIOGRAPHY

% Choosing the proper (math) font shape for mathematics in references: sometimes we want it to follow the current text font shape
% Macro btfs: Bibliograhy Text Font Shape
% Has been implemented in bibfile; this can be undone by redefining \btfs to mean \mathrm in this paper template prior to the bibliography

\makeatletter
\newcommand{\btfs}[1]
{
	\ifthenelse{\equal{\f@shape}{n}}{\ensuremath{\mathrm{#1}}}
	{\ifthenelse{\equal{\f@shape}{sc}}{\ensuremath{\mathrm{#1}}}
		{\ifthenelse{\equal{\f@shape}{it}}{\ensuremath{\mathit{#1}}}
			{\ifthenelse{\equal{\f@shape}{sl}}{\ensuremath{\mathit{#1}}}{}	
			}  
		}
	}
}
\makeatother

% Can be undone by including the following
%\renewcommand{\btfs}{\mathrm}

%%%%%%%%% MORE SPACING BETWEEN THE LINES


% The following command gives more space between lines.
%\linespread{1.5} % Or any other factor to add extra space between lines

%%%%%%%%%%%%%%%%%%%%%%%    STANDARD MACRO'S   %%%%%%%%%%%%%%%%%%%%%%%%%%%%%%%



%%%%%%%%%% GREEK LOWER CASE AND VAR VERSIONS


% Overriding \th
\newcommand{\al}{{\alpha}}
\newcommand{\be}{{\beta}}
\newcommand{\ga}{{\gamma}}
\newcommand{\de}{{\delta}}
\newcommand{\ep}{{\epsilon}}
\newcommand{\ze}{{\zeta}}
\newcommand{\et}{{\eta}}
\renewcommand{\th}{{\theta}} % the old \th is a rune symbol or the like
\newcommand{\io}{{\iota}}
\newcommand{\ka}{{\kappa}}
\newcommand{\la}{{\lambda}}
%\renewcommand{\mu}{{\mu}} % not neceesary: already two letters
%\renewcommand{\nu}{{\nu}} % not neceesary: already two letters
%\renewcommand{\xi}{{\xi}} % not neceesary: already two letters
%\renewcommand{\pi}{{\pi}} % not neceesary: already two letters
\newcommand{\rh}{{\rho}}
\newcommand{\si}{{\sigma}}
\newcommand{\ta}{{\tau}}
\newcommand{\up}{{\upsilon}}
\newcommand{\ph}{{\phi}}
\newcommand{\ch}{{\chi}}
\newcommand{\ps}{{\psi}}
\newcommand{\om}{{\omega}}

\newcommand{\vep}{{\varepsilon}}
\newcommand{\vth}{{\vartheta}}
\newcommand{\vk}{{\varkappa}}
\newcommand{\vpi}{{\varpi}}
\newcommand{\vrh}{{\varrho}}
\newcommand{\vsi}{{\varsigma}}
\newcommand{\vph}{{\varphi}}

%%%%%%%%%% GREEK UPPER CASE AND VAR VERSIONS

\newcommand{\Ga}{{\Gamma}}
\newcommand{\De}{{\Delta}}
\newcommand{\Th}{{\Theta}}
\newcommand{\La}{{\Lambda}}
%\renewcommand{\Xi}{{\Xi}} % not neceesary: already two letters
%\renewcommand{\Pi}{{\Pi}} % not neceesary: already two letters
\newcommand{\Si}{{\Sigma}}
\newcommand{\Up}{{\Upsilon}}
\newcommand{\Ph}{{\Phi}}
\newcommand{\Ps}{{\Psi}}
\newcommand{\Om}{{\Omega}}

\newcommand{\vGa}{\varGamma}
\newcommand{\vDe}{\varDelta}
\newcommand{\vTh}{\varTheta}
\newcommand{\vLa}{\varLambda}
\newcommand{\vXi}{\varXi}
\newcommand{\vPi}{\varPi}
\newcommand{\vSi}{\varSigma}
\newcommand{\vUp}{\varUpsilon}
\newcommand{\vPh}{\varPhi}
\newcommand{\vPs}{\varPsi}
\newcommand{\vOm}{\varOmega}


\newcommand{\da}{\ensuremath{\digamma}}


%%%%%%%%%% BLACKBOARD LETTERS


% Overriding \AA and \SS
\renewcommand{\AA}{{\bbfont A}} % the old \AA is the upper case Angstrom
\newcommand{\BB}{{\bbfont B}} 
\newcommand{\CC}{{\bbfont C}}
\newcommand{\DD}{{\bbfont D}}
\newcommand{\EE}{{\bbfont E}}
\newcommand{\FF}{{\bbfont F}}
\newcommand{\GG}{{\bbfont G}}
\newcommand{\HH}{{\bbfont H}}
\newcommand{\II}{{\bbfont I}}
\newcommand{\JJ}{{\bbfont J}}
\newcommand{\KK}{{\bbfont K}}
\newcommand{\LL}{{\bbfont L}}
\newcommand{\MM}{{\bbfont M}}
\newcommand{\NN}{{\bbfont N}}
\newcommand{\OO}{{\bbfont O}}
\newcommand{\PP}{{\bbfont P}}
\newcommand{\QQ}{{\bbfont Q}}
\newcommand{\RR}{{\bbfont R}}
\renewcommand{\SS}{{\bbfont S}} % the old \SS is the upper case German sharp s (which does not seem to work, it just gives SS in normal font)
\newcommand{\TT}{{\bbfont T}}
\newcommand{\UU}{{\bbfont U}}
\newcommand{\VV}{{\bbfont V}}
\newcommand{\WW}{{\bbfont W}}
\newcommand{\XX}{{\bbfont X}}
\newcommand{\YY}{{\bbfont Y}}
\newcommand{\ZZ}{{\bbfont Z}}


%%%%%%%%%% MACRO'S FOR SPECIAL CHARACTERS

\newcommand{\Angstrom}{\mbox{\r{A}}}
\newcommand{\angstrom}{\mbox{\r{a}}}


%%%%%%%%%% UPRIGHT SYMBOLS IN MATH MODE

\newcommand{\upa}{{\mathrm{a}}}
\newcommand{\upb}{{\mathrm{b}}}
\newcommand{\upc}{{\mathrm{c}}}
\newcommand{\upd}{{\mathrm{d}}}
\newcommand{\upe}{{\mathrm{e}}}
\newcommand{\upf}{{\mathrm{f}}}
\newcommand{\upg}{{\mathrm{g}}}
\newcommand{\uph}{{\mathrm{h}}}
\newcommand{\upi}{{\mathrm{i}}}
\newcommand{\upj}{{\mathrm{j}}}
\newcommand{\upk}{{\mathrm{k}}}
\newcommand{\upl}{{\mathrm{l}}}
\newcommand{\upm}{{\mathrm{m}}}
\newcommand{\un}{{\mathrm{n}}} % \upn already defined
\newcommand{\upo}{{\mathrm{o}}}
\newcommand{\upp}{{\mathrm{p}}}
\newcommand{\upq}{{\mathrm{q}}}
\newcommand{\upr}{{\mathrm{r}}}
\newcommand{\ups}{{\mathrm{s}}}
\newcommand{\upt}{{\mathrm{t}}}
\newcommand{\upu}{{\mathrm{u}}}
\newcommand{\upv}{{\mathrm{v}}}
\newcommand{\upw}{{\mathrm{w}}}
\newcommand{\upx}{{\mathrm{x}}}
\newcommand{\upy}{{\mathrm{y}}}
\newcommand{\upz}{{\mathrm{z}}}

\newcommand{\upA}{{\mathrm{A}}}
\newcommand{\upB}{{\mathrm{B}}}
\newcommand{\upC}{{\mathrm{C}}}
\newcommand{\upD}{{\mathrm{D}}}
\newcommand{\upE}{{\mathrm{E}}}
\newcommand{\upF}{{\mathrm{F}}}
\newcommand{\upG}{{\mathrm{G}}}
\newcommand{\upH}{{\mathrm{H}}}
\newcommand{\upI}{{\mathrm{I}}}
\newcommand{\upJ}{{\mathrm{J}}}
\newcommand{\upK}{{\mathrm{K}}}
\newcommand{\upL}{{\mathrm{L}}}
\newcommand{\upM}{{\mathrm{M}}}
\newcommand{\upN}{{\mathrm{N}}} 
\newcommand{\upO}{{\mathrm{O}}}
\newcommand{\upP}{{\mathrm{P}}}
\newcommand{\upQ}{{\mathrm{Q}}}
\newcommand{\upR}{{\mathrm{R}}}
\newcommand{\upS}{{\mathrm{S}}}
\newcommand{\upT}{{\mathrm{T}}}
\newcommand{\upU}{{\mathrm{U}}}
\newcommand{\upV}{{\mathrm{V}}}
\newcommand{\upW}{{\mathrm{W}}}
\newcommand{\upX}{{\mathrm{X}}}
\newcommand{\upY}{{\mathrm{Y}}}
\newcommand{\upZ}{{\mathrm{Z}}}


% use \uppi for the constant \pi

%%%%%%%%%% UPRIGHT PARENTHESES ETC (FOR THEOREM ENVIRONMENT ETC.)

\newcommand{\ulb}{{\textup{\{}}}
\newcommand{\urb}{{\textup{\}}}}
\newcommand{\ulsb}{{\textup{[}}}
\newcommand{\ursb}{{\textup{]}}}
\newcommand{\ulp}{{\textup{(}}}
\newcommand{\urp}{{\textup{)}}}

\newcommand{\uppars}[1]{\ulp #1\urp}

%%%%%%%%%% PARENTHESES ETC IN MATH MODE

\newcommand{\abs}[1]{{\lvert #1 \rvert}}
\newcommand{\norm}[1]{{\lVert #1 \rVert}}
\newcommand{\angles}[1]{{\langle #1\rangle}}
\newcommand{\braces}[1]{{\{ #1\}}}
\newcommand{\brackets}[1]{{[ #1]}}
\newcommand{\pars}[1]{{( #1)}}

\newcommand{\lrabs}[1]{{\left\lvert #1 \right\rvert}}
\newcommand{\lrnorm}[1]{{\left\lVert #1 \right\rVert}}
\newcommand{\lrangles}[1]{{\left\langle #1\right\rangle}}
\newcommand{\lrbraces}[1]{{\left\{ #1\right\}}}
\newcommand{\lrbrackets}[1]{{\left[ #1\right]}}
\newcommand{\lrpars}[1]{{\left( #1\right)}}

\newcommand{\bigabs}[1]{{\big\lvert #1 \big\rvert}}
\newcommand{\bignorm}[1]{{\big\lVert #1 \big\rVert}}
\newcommand{\bigangles}[1]{{\big\langle #1\big\rangle}}
\newcommand{\bigbraces}[1]{{\big\{ #1\big\}}}
\newcommand{\bigbrackets}[1]{{\big[ #1\big]}}
\newcommand{\bigpars}[1]{{\big( #1\big)}}

\newcommand{\Bigabs}[1]{{\Big\lvert #1 \Big\rvert}}
\newcommand{\Bignorm}[1]{{\Big\lVert #1 \Big\rVert}}
\newcommand{\Bigangles}[1]{{\Big\langle #1\Big\rangle}}
\newcommand{\Bigbraces}[1]{{\Big\{ #1\Big\}}}
\newcommand{\Bigbrackets}[1]{{\Big[ #1\Big]}}
\newcommand{\Bigpars}[1]{{\Big( #1\Big)}}

\newcommand{\biggabs}[1]{{\bigg\lvert #1 \bigg\rvert}}
\newcommand{\biggnorm}[1]{{\bigg\lVert #1 \bigg\rVert}}
\newcommand{\biggangles}[1]{{\bigg\langle #1\bigg\rangle}}
\newcommand{\biggbraces}[1]{{\bigg\{ #1\bigg\}}}
\newcommand{\biggbrackets}[1]{{\bigg[ #1\bigg]}}
\newcommand{\biggpars}[1]{{\bigg( #1\bigg)}}

\newcommand{\Biggabs}[1]{{\Bigg\lvert #1 \Bigg\rvert}}
\newcommand{\Biggnorm}[1]{{\Bigg\lVert #1 \Bigg\rVert}}
\newcommand{\Biggangles}[1]{{\Bigg\langle #1\Bigg\rangle}}
\newcommand{\Biggbraces}[1]{{\Bigg\{ #1\Bigg\}}}
\newcommand{\Biggbrackets}[1]{{\Bigg[ #1\Bigg]}}
\newcommand{\Biggpars}[1]{{\Bigg( #1\Bigg)}}

% SETS

\newcommand{\set}[1]{\braces{#1}}
\newcommand{\lrset}[1]{\lrbraces{#1}}
\newcommand{\bigset}[1]{\bigbraces{#1}}
\newcommand{\Bigset}[1]{\Bigbraces{#1}}
\newcommand{\biggset}[1]{\biggbraces{#1}}
\newcommand{\Biggset}[1]{\Biggbraces{#1}}

\newcommand{\desset}[1]{\braces{\,#1\,}}
\newcommand{\lrdesset}[1]{\lrbraces{\,#1\,}}
\newcommand{\bigdesset}[1]{\bigbraces{\,#1\,}}
\newcommand{\Bigdesset}[1]{\Bigbraces{\,#1\,}}
\newcommand{\biggdesset}[1]{\biggbraces{\,#1\,}}
\newcommand{\Biggdesset}[1]{\Biggbraces{\,#1\,}}


%%%%%%%%%% COMMONLY USED SPECIAL SYMBOLS

\newcommand{\intn}{\int\!}

\newcommand{\di}[1]{\,\upd #1}

\newcommand{\cont}{\tfs{C}}
\newcommand{\conto}{\cont_0}
\newcommand{\contb}{\cont_{\tfs{b}}}
\newcommand{\contc}{\cont_{\tfs{c}}}

\newcommand{\bounded}{\tfs{B}}
\newcommand{\linear}{\tfs{L}}
\newcommand{\regular}{\linear_{\tfs{r}}}

\newcommand{\Ell}{\tfs{L}}


\newcommand{\Cstar}{\ensuremath{\tfs{C}^\ast}}
\newcommand{\Calgebra}{\Cstar\!-algebra}
\newcommand{\Calgebras}{\Cstar\!-algebras}

\newcommand{\onefunction}{{\mathbf 1}}
\newcommand{\zerofunction}{{\mathbf 0}}

\newcommand{\idmap}{{\mathrm{id}}}
\newcommand{\idop}{\mathbf{1}}
\newcommand{\zeop}{\mathbf{0}}


% Redefine \Re and \Im and add lower case versions

\renewcommand{\Im}{\operatorname{Im}} % the old \Im is the fraktur capital I
\renewcommand{\Re}{\operatorname{Re}} % the old \Re is the fraktur capital R
\newcommand{\im}{\operatorname{im}}
\newcommand{\re}{\operatorname{re}}

% A few operations, upper case versions


\newcommand{\Aut}{\operatorname{Aut}}
\newcommand{\Co}{\operatorname{Co}}
\newcommand{\clCo}{\operatorname{\overline{Co}}}
\newcommand{\Hom}[1]{\operatorname{Hom_{#1}}}
\newcommand{\End}[1]{\operatorname{End_{#1}}}
\newcommand{\Inv}{\operatorname{Inv}}
\newcommand{\Ker}{\operatorname{Ker}}
\newcommand{\Ran}{\operatorname{Ran}}

\newcommand{\Spuc}[1]{\operatorname{Sp_{#1}}}
\newcommand{\Spnuc}[1]{\operatorname{Spn_{#1}}}
\newcommand{\Spanuc}[1]{\operatorname{Span_{#1}}}
\newcommand{\clSpuc}[1]{\operatorname{\overline{Sp}_{#1}}}
\newcommand{\clSpnuc}[1]{\operatorname{\overline{Spn}_{#1}}}
\newcommand{\clSpanuc}[1]{\operatorname{\overline{Span}_{#1}}}

% A few operations, lower case versions

\newcommand{\aut}{\operatorname{aut}}
\newcommand{\co}{\operatorname{co}}
\newcommand{\clco}{\operatorname{\overline{co}}}
\renewcommand{\hom}[1]{\operatorname{hom_{#1}}} % the old \hom gives hom in math mode
\newcommand{\Endlc}[1]{\operatorname{end_{#1}}} % LaTeX restriction: cannot introduce command with a name beginning with end accepting an optional argument.
\newcommand{\inv}{\operatorname{inv}}
\renewcommand{\ker}{\operatorname{ker}} % the old \ker gives ker in math mode
\newcommand{\ran}{\operatorname{ran}}

\newcommand{\splc}[1]{\operatorname{sp_{#1}}} % the old \sp gives superscript in math mode: DO NOT REDEFINE THIS
\newcommand{\spn}[1]{\operatorname{spn_{#1}}}
\newcommand{\spanlc}[1]{\operatorname{span_{#1}}} % the old \span is used in column environments: DO NOT REDEFINE THIS 
\newcommand{\clsp}[1]{\operatorname{\overline{sp}_{#1}}}
\newcommand{\clspn}[1]{\operatorname{\overline{spn}_{#1}}}
\newcommand{\clspan}[1]{\operatorname{\overline{span}_{#1}}}


%%%%%%%%%%%%%% ENVIRONMENTS %%%%%%%%%%%%%%%%%%%%%%%%%%%%%%%%%%%%%%%%%%


\theoremstyle{plain}

\newtheorem{theorem}{Theorem}[section]
\newtheorem{proposition}[theorem]{Proposition}
\newtheorem{lemma}[theorem]{Lemma}
\newtheorem{corollary}[theorem]{Corollary}
\newtheorem{conjecture}[theorem]{Conjecture}
%\newtheorem{definition}[theorem]{Definition}
%\newtheorem{example}[theorem]{Example}
%\newtheorem{remark}[theorem]{Remark}
\newtheorem{assumption}[theorem]{Assumption}
\newtheorem{hypothesis}[theorem]{Hypothesis}
\newtheorem{question}[theorem]{Question}
\newtheorem{problem}[theorem]{Problem}
\newtheorem{task}[theorem]{Task}
\newtheorem{addendum}[theorem]{Addendum}
\newtheorem{idea}[theorem]{Idea}
\newtheorem{suggestion}[theorem]{Suggestion}
\newtheorem{context}[theorem]{Context}
\newtheorem{exercise}[theorem]{Exercise}

% Unnumbered versions
\newtheorem*{theorem*}{Theorem}
\newtheorem*{proposition*}{Proposition}
\newtheorem*{lemma*}{Lemma}
\newtheorem*{corollary*}{Corollary}
\newtheorem*{conjecture*}{Conjecture}
%\newtheorem*{definition*}{Definition}
%\newtheorem*{example*}{Example}
%\newtheorem*{remark*}{Remark}
\newtheorem*{assumption*}{Assumption}
\newtheorem*{hypothesis*}{Hypothesis}
\newtheorem*{question*}{Question}
\newtheorem*{problem*}{Problem}
\newtheorem*{task*}{Task}
\newtheorem*{addendum*}{Addendum}
\newtheorem*{idea*}{Idea}
\newtheorem*{suggestion*}{Suggestion}
\newtheorem*{context*}{Context}
\newtheorem*{exercise*}{Exercise}

\theoremstyle{definition}

%\newtheorem{theorem}{Theorem}[section]
%\newtheorem{proposition}[theorem]{Proposition}
%\newtheorem{lemma}[theorem]{Lemma}
%\newtheorem{corollary}[theorem]{Corollary}
%\newtheorem{conjecture}[theorem]{Conjecture}
\newtheorem{definition}[theorem]{Definition}
\newtheorem{example}[theorem]{Example}
\newtheorem{remark}[theorem]{Remark}
%\newtheorem{assumption}[theorem]{Assumption}
%\newtheorem{hypothesis}[theorem]{Hypothesis}
%\newtheorem{question}[theorem]{Question}
%\newtheorem{problem}[theorem]{Problem}
%\newtheorem{task}[theorem]{Task}
%\newtheorem{addendum}[theorem]{Addendum}
%\newtheorem{idea}[theorem]{Idea}
%\newtheorem{suggestion}[theorem]{Suggestion}
%\newtheorem{context}[theorem]{Context}
%\newtheorem{exercise}[theorem]{Exercise}

% Unnumbered versions
%\newtheorem*{theorem*}{Theorem}
%\newtheorem*{proposition*}{Proposition}
%\newtheorem*{lemma*}{Lemma}
%\newtheorem*{corollary*}{Corollary}
%\newtheorem*{conjecture*}{Conjecture}
\newtheorem*{definition*}{Definition}
\newtheorem*{example*}{Example}
\newtheorem*{remark*}{Remark}
%\newtheorem*{assumption*}{Assumption}
%\newtheorem*{hypothesis*}{Hypothesis}
%\newtheorem*{question*}{Question}
%\newtheorem*{problem*}{Problem}
%\newtheorem*{task*}{Task}
%\newtheorem*{addendum*}{Addendum}
%\newtheorem*{idea*}{Idea}
%\newtheorem*{suggestion*}{Suggestion}
%\newtheorem*{context*}{Context}
%\newtheorem*{exercise*}{Exercise}


%%%%%%%%%%%%%%%%%%%%%%%% MODALITIES FOR ENUMITEM AND CLEVEREF %%%%%%%%%%%%%%%%%%%%%%%%%%%%%%%%%%%


\setlist[enumerate,1]{label=\textup{(\arabic*)},ref=\arabic*}
\setlist[enumerate,2]{label=\textup{(\alph*)},ref=\arabic{enumi}.\alph*}
\setlist[enumerate,3]{label=\textup{(\roman*)},ref=\arabic{enumi}.\alph{enumii}.\roman*}
\setlist[enumerate,4]{label=\textup{(\Alph*)},ref=\arabic{enumi}.\alph{enumii}.\roman{enumiii}.\Alph*}

% Use of \cref or \Cref is irrelevant for environments. Since \Cref will use the capitalized versions if not otherwise specified, all references will automatically be in capitals as a consequence of the following:
\crefname{theorem}{Theorem}{Theorems}
\crefname{proposition}{Proposition}{Propositions}
\crefname{lemma}{Lemma}{Lemmas}
\crefname{corollary}{Corollary}{Corollaries}
\crefname{conjecture}{Conjecture}{Conjectures}
\crefname{definition}{Definition}{Definitions}
\crefname{example}{Example}{Examples}
\crefname{remark}{Remark}{Remarks}
\crefname{assumption}{Assumption}{Assumptions}
\crefname{hypothesis}{Hypothesis}{Hypotheses}
\crefname{question}{Question}{Questions}
\crefname{problem}{Problem}{Problems}
\crefname{task}{Task}{Tasks}
\crefname{addendum}{Addendum}{Addenda}
\crefname{idea}{Idea}{Ideas}
\crefname{suggestion}{Suggestion}{Suggestions}
\crefname{context}{Context}{Contexts}
\crefname{exercise}{Exercise}{Exercises}

\crefname{section}{Section}{Sections}
\crefname{subsection}{Section}{Sections}
\crefname{subsubsection}{Section}{Sections}

% Equations and parts are treated differently: these can have lower case. 
% Use \cref and \Cref to get `1', etc.: the number of the part without delimiters.
% Use \partref below to get `(1)', etc.: the number of the part with delimiters.
\crefname{equation}{equation}{equations}
\crefname{enumi}{part}{parts}
\crefname{enumii}{part}{parts}
\crefname{enumiii}{part}{parts}
\crefname{enumiv}{part}{parts}

\newcommand{\enclosepart}[1]{(#1)}
\newcommand{\partref}[1]{\enclosepart{\ref{#1}}}
\newcommand{\uppartref}[1]{\textup{\partref{#1}}}

\newcommand{\detailedref}[2]{\ref{#1}.\ref{#2}}
\newcommand{\detailedcref}[2]{\cref{#1}.\ref{#2}}
\newcommand{\detailedCref}[2]{\Cref{#1}.\ref{#2}}
\newcommand{\updetailedref}[2]{\textup{\detailedref{#1}{#2}}}
\newcommand{\updetailedcref}[2]{\textup{\detailedcref{#1}{#2}}}
\newcommand{\updetailedCref}[2]{\textup{\detailedCref{#1}{#2}}}

% The Oxford or serial comma will be used if the following command is included.
\newcommand{\creflastconjunction}{, and\nobreakspace}


%%%%%%%%%%%%%%%%%%%%%%%%%%%%%%%%%%%%% LAYOUT CHOICES %%%%%%%%%%%%%%%%%%%%%%%%%%%%%

\numberwithin{equation}{section}

\allowdisplaybreaks % displayed equations can be broken across pages

%%%%%%%%%%%%%% PACKAGES FOR THIS DOCUMENT %%%%%%%%%%%%%%%%%%%%%%%%%%%%%%%%%






%%%%%%%%%%%%%% MACROS FOR THIS DOCUMENT %%%%%%%%%%%%%%%%%%%%%%%%%%%%%%%%%%

\newcommand{\veps}{\varepsilon}
\newcommand{\dc}{{\mathrm d}}

\newcommand{\supp}{\operatorname{supp}}
%\newcommand{\span}{\operatorname{span}}
\newcommand{\bfs}{Banach function space }
\newcommand{\bfss}{Banach function spaces }
\newcommand{\fg}{compact Hausdorff abelian topological group }
\newcommand{\quotient}[1]{#1^*}
\newcommand{\I}[1]{#1'}
%%%%%%%%%%%%%%%%%%%%%%%%%%%%%%%%%%%%% START OF ACTUAL TEXTT %%%%%%%%%%%%%%%%%%%%%%%%


\begin{document}



%%%%%%%%%%%%%%%%%%%%%%%%%%%%%%%%% BEGIN FRONTMATTER %%%%%%%%%%%%%%%%%%%%%%%%%%%%%%

\title [SHORT TITLE]{FULL TITLE}

\author{Chun Ding}

\address{Mathematical Institute, Leiden University, P.O.\ Box 9512, 2300 RA Leiden, the Netherlands}
\email{c.d.ding@math.leidenuniv.nl}

\author{Marcel de Jeu}

\address{Mathematical Institute, Leiden University, P.O.\ Box 9512, 2300 RA Leiden, the Netherlands;
	and Department of Mathematics and Applied Mathematics, University of Pretoria, Cor\-ner of Lynnwood Road and Roper Street, Hatfield 0083, Pretoria, South Africa}
\email{mdejeu@math.leidenuniv.nl}

% Continue adding authors as above

%\dedicatory{DEDICATION GOES HERE}

\date{\tt {File: \currfilebase; Compiled: \today,\,\currenttime}}

%\makeatletter{\renewcommand*{\@makefnmark}{}
%	\date{}\footnote{{File: \currfilebase. Compiled: \today,\,\currenttime.}}
%	\makeatother} % cause problems on diagrams

\subjclass[2010]{Primary FIRST CLASS; Secondary SECONDARY CLASS(ES)}

\keywords{KEYWORDS GO HERE}

%\thanks{GRATITUDE GOES HERE}


% Use the following when numbering lines in the abstract (uncomment the pertinent part above)
% \abstract{THE ABSTRACT GOES HERE}

% Use the folllowing when not numbering lines in the abstract (comment the pertinent part above)
\begin{abstract} THE ABSTRACT GOES HERE \end{abstract}

\maketitle

%%%%%%%%%%%%%%%%%%%%%%%%%%%%%%%%% END FRONTMATTER AND START OF THE BODY OF THE TEXT %%%%%%%%%%%%%%%%%%%%%%%%%%%%%%%%%%%%%%

%\section{preliminary}
 Denote $\mathbf{Ban_1}$ the category Banach spaces and contractive linear maps, $\mathbf{L_1}$ the category of Banach lattices and contractive lattice homomorphisms and $\mathbf{BL_1}$ the category of Banach lattices and almost interval preserving contractions. Recall that a positive linear map $\phi:E\to F$ between normed Riesz spaces is called \emph{almost interval preserving} if $\phi([0, x])$ is dense in $[0,\phi(x)]$ for every $x\in E_+$. It follows from \cite[Proposition 1.3.13]{meyer-nieberg_BANACH_LATTICES:1991} immediately that the adjoint, denoted by $^*$, is a contravariant functor between $\mathbf{BL_1}$ and $\mathbf{L_1}$. Also, it is easy to see that both $\mathbf{BL_1}$  and $\mathbf{L_1}$ are subcategories of $\mathbf{Ban_1}$. 
 
\begin{lemma}\label{direct_limit_from_Ban_1}
$((E_i),(\phi_{ji})_{j\geq i})$ is a direct system indexed by a directed set in $\mathbf{BL_1}$. If $E$ is a Banach lattice, $\phi_i:E_i\to E$ is an almost interval preserving map for each index $i$, then $(E,\phi_i)$ is a direct limit of $((E_i),(\phi_{ji})_{j\geq i})$ in $\mathbf{Ban_1}$ is equivalent to that in $\mathbf{BL_1}$.
\end{lemma}
\begin{proof}
Let $(E,(\phi_i))$ is a direct limit  of $((E_i),(\phi_{ji})_{j\geq i})$ in $\mathbf{Ban_1}$, then $\bigcup_{i}\phi_i(E_i)$ is dense in $E$ by Banach space theory.
Suppose $F$ is a Banach lattice and $\psi_i:E_i\to F$ is an almost interval preserving contraction for each $i$ such that $\psi_j\circ\phi_{ji}=\psi_i$ whenever $j\geq i$. Let $\psi:E\to F$ be the unique contractive linear map satisfying $\psi\circ\phi_i=\psi_i$ for each $i$, we will prove that $\psi$ is almost interval preserving, that is, $(E,(\phi_i))$ is also a direct limit  of $((E_i),(\phi_{ji})_{j\geq i})$ in $\mathbf{BL_1}$.
Thanks to \cite[Proposition 1.3.13]{meyer-nieberg_BANACH_LATTICES:1991}, we need only prove that $\psi^*:F^*\to E^*$ is a lattice homomorphism.
Pick any $\tau\in F^*$, then \begin{align*}\phi_i^*\circ\psi^*(|\tau|)=\psi_i^*(|\tau|)=|\psi_i^*(\tau)|=|\phi_i^*\circ\psi^*(\tau)|=\phi_i^*(|\psi^*(\tau)|),\end{align*} i.e., $\psi^*(|\tau|)\circ\phi_i=|\psi^*(\tau)|\circ\phi_i$. Therefore, $\psi^*(|\tau|)=|\psi^*(\tau)|$ on $\bigcup_i\phi_i(E_i)$ and thus on $E$, which finishes the proof.
\end{proof}

%The following theorem is a corollary of the above lemma.
\begin{theorem}\label{direct_limit}
Every direct system $((E_i),(\phi_{ji})_{j\geq i})$ admits a direct limit in $\mathbf{BL_1}$. Specially, if each $E_i$ is closed sublattice of a Banach lattice $E$ and $E_i$ is an order ideal of $E_j$ whenever $j\geq i$, then the norm closure of $\bigcup_iE_i$ is a direct limit of $(E_i)$ and inclusion maps.
\end{theorem} 
\begin{proof}
    It's easy to verify that $\prod_{i}E_i:=\{(E_i):\sup_{i}{\|E_i\|}<\infty\}$ with pointwise order and the supremum norm is a Banach lattice and that $\bigoplus_{i}E_i:=\{(E_i):\|E_i\|\to 0 \mbox{ as } i\to\infty\}$ is a closed order ideal of $\prod_{i}E_i$. By \cite[Proposition 1.3.13]{meyer-nieberg_BANACH_LATTICES:1991}, the quotient $\prod_{i}E_i/\bigoplus_{i}E_i$ is a Banach lattice and the quotient map $q:\prod_{i}E_i\to \prod_{i}E_i/\bigoplus_{i}E_i$ is a lattice homomorphism.
For each $i$, there is a natural positive linear opertaor $\Phi_i:E_i\to \prod_{i}E_i$ defined by setting the component of $\Phi_i(a)$ in $E_j$ to be $\phi_{ji}(a)$ if $j\geq i$ and $0$ otherwise. 
Let $\phi_i=q\circ\Phi_i$ and $E=\overline{\bigcup_{i}\phi_i(E_i)}$, then $(E,(\phi_i))$ is a direct limit of  $((E_i),(\phi_{ji})_{j\geq i})$ in $\mathbf{Ban_1}$ by Banach space theory.

Given an index $i_0$, $a\in E_{i_0+}$ and $\epsilon>0$. Suppose $y=q((y_i))\in [0,\phi_{i_0}(a)]\cap E, (y_i)\in \prod_{i}E_i$. Since $q$ is a lattice homomorphism, we can assume $y_i\in [0,\phi_{ii_0}(a)]$ if $i\geq i_0$ and $y_i=0$ otherwise.
	Choose an index $i_\epsilon$ and $a_\epsilon\in A_{i_\epsilon}$ satisfying
	$\|y-\phi_{i_\epsilon}(a_\epsilon)\|< \epsilon,$
	%By definition of quotient norm, there exists some $(z_i)\in\bigoplus_{i}E_i$ such that 
	%$\|(y_i)-\phi_{i_\epsilon}(a_\epsilon)-(z_i)\|<\epsilon.$ 
	then
	$\|y_{i_\infty}-\phi_{i_\infty i_\epsilon}(a_\epsilon)\|<\epsilon$
	for some sufficiently large $i_\infty\geq i_\epsilon, i_0$.
	%By the definition of almost interval preserving, %
	Since $\phi_{i_\infty i_0}$ is almost interval preserving, 
	there exists some $x\in [0, a]$ such that
	$\|y_{i_\infty}-\phi_{i_\infty i_0}(x)\|<\epsilon$.
	Consequently, 
	\begin{align*}
	\|y-\phi_{i_0}(x)\|
	& =\|y-\phi_{i_\epsilon}(a_\epsilon)+\phi_{i_\infty}(\phi_{i_\infty i_\epsilon}(a_\epsilon)-y_{i_\infty}+y_{i_\infty}-\phi_{i_\infty i_0}(x))\|\\
	& \leq \|y-\phi_{i_\epsilon}(a_\epsilon)\|+\|\phi_{i_\infty i_\epsilon}(a_\epsilon)-y_{i_\infty}\|+\|y_{i_\infty}-\phi_{i_\infty i_0}(x)\|\leq  3\epsilon.
	\end{align*}
	That is, $\overline{\phi_{i_0}([0,a])}\supset [0,\phi_{i_0}(a)]\cap E$.
	
	Pick $b\in \bigcup_{i}\phi_{i}(E_i)$ and suppose $b=\phi_{i}(a)$ for some index $i$ and $a\in E_i$, then $|b|\in [0,\phi_{i}(|a|)]$. Hence, for arbitrary $\epsilon>0$, there is some $x\in [0,|a|]$ such that $\||b|-\phi_{i}(x)\|<\epsilon$, which follows that $|b|\in E$. The continuity of lattice operators implies that $|y|\in E$ whenever $y\in E$, i.e., the Banach space $E$ is in fact a Banach lattice. Together with the conclusion of last paragraph, it follows that every $\phi_i:E_i\to E$ is an %arrow in \mathbf{BL}.%
	almost interval preserving contraction between Banach lattices. 
	Our proof finishes with application of Lemma \ref{direct_limit_from_Ban_1}.
\end{proof}

\begin{theorem}\label{direct_limit_order_continuous}
    $((E_i),(\phi_{ji})_{j\geq i})$ is a direct system with a direct limit $(E,(\phi_i))$ in $\mathbf{BL_1}$.  If each $E_i$ is order continuous,  so is $E$.
\end{theorem}
\begin{proof}
Let us look into the following diagram in $\mathbf{Ban_1}$,  where $((E_i),(\phi_{ji})_{j\geq  i}))$ is a direct system with a direct limit $(E,\phi_i)$ in $\mathbf{BL_1}$ and $\iota_i$ is the canonical embedding for each $i$.
      $$\xymatrix{ 
        E_i\ar[r]^{\phi_i}\ar[d]_{\iota_i}  & E\ar@{-->}[d]\\
        E_i^{**}\ar[r]^{\phi_i^{**}} & E^{**}
        }$$
    \cite[Proposition 1.3.13]{meyer-nieberg_BANACH_LATTICES:1991} follows that $\phi_i^{**}:E_i^{**}\to E^{**}$ is a morphism in $\mathbf{BL_1}$ and \cite[Theorem 2.4.1]{meyer-nieberg_BANACH_LATTICES:1991} follows that $\iota_i(E_i)$ is an order ideal of $E^{**}$ which is equivalent to $\iota_i: E_i\to E_i^{**}$ is a morphism in $\mathbf{BL_1}$. By Theorem \ref{direct_limit}, the dash arrow in the above commutative diagram  can be filled in with a morphism $\iota: E\to E^{**}$ in $\mathbf{BL_1}$ . However, the canonical embedding is the unique continuous linear map filling in the dash arrow in $\mathbf{Ban_1}$, so, $\iota$ coincides with the canonical embedding. Using \cite[Theorem 2.4.1]{meyer-nieberg_BANACH_LATTICES:1991} again, $E$ is order continuous.
\end{proof}
An observation is that the order continuity of a direct limit in $\mathbf{BL_1}$ is independent on the choice.%, since it is an invariant property of positive linear bijection.


 
 
%Suppose $T:A\to  B$ is a positive operator between normed Riesz spaces. $T$ is said to be \emph{almost solid} if the closure of $TA$ is an ideal of $B$; $T$ is said to be \emph{almost interval preserving} if $T[0, x]$ is dense in $[0,Tx]$ for every $x\in A_+$.
%An almost interval preserving operator is always almost solid by \cite[Theorem 2.1]{bouras2018some}. 

%Denote by $\mathbf{BL}$ the category whose objects are Banach lattices and whose arrows are almost solid contractions. 
%Recall that a positive linear operator $T:A\to B$ between Riesz spaces is called \emph{almost interval preserving} if $T[0, x]$ is dense in $[0,Tx]$ for every $x\in A_+$.
%if $[0, Tx]=T[0, x]$ for every $x\in A_+$;

%
%
% 
%Denote by $\mathbf{{BL}_c}$ the category whose objects are order continuous Banach lattices and whose arrows are interval preserving contractions. Recall that a positive linear operator $T:A\to B$ between Riesz spaces is called \emph{interval preserving} if $T[0, x]=[0,Tx]$ for every $x\in A_+$.
%%if $[0, Tx]=T[0, x]$ for every $x\in A_+$;
%
%\begin{theorem}
%	For any direct system $((E_i),(\phi_{ji})_{j\geq i})$ in $\mathbf{{BL}_c}$, its direct limit exists.  
%\end{theorem}
%\begin{proof}
%	Since  $\prod_{i}E_i:=\{(E_i):\sup_{i}{\|E_i\|}<\infty\}$ is a Banach lattice and $\bigoplus_{i}:=\{(E_i):\|E_i\|\to 0 \mbox{ as } i\to\infty\}$ is a closed order ideal of $\prod_{i}E_i$, the quotient $\prod_{i}E_i/\bigoplus_{i}E_i$ is a Banach lattice and the quotient map $q:\prod_{i}E_i\to \prod_{i}E_i/\bigoplus_{i}E_i$ is a lattice homomorphism by \cite[Proposition 1.3.13]{meyer-nieberg_BANACH_LATTICES:1991}.
%	For each $i$, there is a natural positive linear opertaor $\Phi_i:E_i\to \prod_{i}E_i$ whose component of $\Phi_i(a)$ in $A_j$ is $\phi_{ji}(a)$ if $j\geq i$ and $0$ otherwise. 
%	Define $\phi_i=q\circ\Phi_i$ and $A=\overline{\bigcup_{i}\phi_i(E_i)}$.
%	A direct computaion yields each $\phi_i$ is a positive linear contraction satisfying $\phi_{ji}\circ\phi_i=\phi_j$ for $j\geq i$.
%	We will prove that $(A,(\phi_i))$ is a direct limit of  $((E_i),(\phi_{ji})_{j\geq i})$.
%	
%	Given an index $i_0$, $a\in A_{i_0+}$ and $\epsilon>0$. Suppose $y=q((y_i))\in [0,\phi_{i_0}(a)]\cap A, (y_i)\in \prod_{i}E_i$. Since $q$ is a lattice homomorphism, we can assume $y_i\in [0,\phi_{ii_0}(a)]$ if $i\geq i_0$ and $y_i=0$ otherwise.
%	Choose an index $i_\epsilon$ and $a_\epsilon\in A_{i_\epsilon}$ satisfying
%	$\|y-\phi_{i_\epsilon}(a_\epsilon)\|< \epsilon.$
%	%By definition of quotient norm, there exists some $(z_i)\in\bigoplus_{i}E_i$ such that 
%	%$\|(y_i)-\phi_{i_\epsilon}(a_\epsilon)-(z_i)\|<\epsilon.$ 
%	Therefore, 
%	$\|y_{i_\infty}-\phi_{i_\infty i_\epsilon}(a_\epsilon)\|<\epsilon$
%	for some sufficiently large $i_\infty\geq i_\epsilon, i_0$.
%	Since $\phi_{i_\infty i_0}$ is almost interval preserving, there exists some $x\in [0, a]$ such that
%	$y_{i_\infty}=\phi_{i_\infty i_0}(x)$.
%	Consequently, 
%	\begin{align*}
%	\|y-\phi_{i_0}(x)\|
%	& =\|y-\phi_{i_\epsilon}(a_\epsilon)+\phi_{i_\infty}(\phi_{i_\infty i_\epsilon}(a_\epsilon)-\phi_{i_\infty i_0}(x))\|\\
%	& \leq \|y-\phi_{i_\epsilon}(a_\epsilon)\|+\|\phi_{i_\infty i_\epsilon}(a_\epsilon)-y_{i_\infty}\|\leq  2\epsilon.
%	\end{align*}
%	That is, 
%	$\overline{\phi_i([0,a])}\supset [0,\phi_i(a)]\cap A$.
%	Since a linear operator between normed space is weakly continuous iff it is norm continuous, and \cite[Theorem 2.4.2]{meyer-nieberg_BANACH_LATTICES:1991} follows that $[0,a]$ is weakly compact in the order continuous Banach lattice $A_{i_0}$,
%	$\phi_i([0,a])$ is weakly compact and thus norm closed. This results in 
%	$\phi_i([0,a])=[0,\phi_i(a)]$ in $A$.
%	
%	
%	Pick $b\in \bigcup_{i}\phi_{i}(E_i)$ and suppose $b=\phi_{i}(a)$ for some index $i$ and $a\in E_i$, then $|b|\in [0,\phi_{i}(|a|)]$. %Hence, for arbitrary $\epsilon>0$, there is some $x\in [0,|a|]$ such that $\||b|-\phi_{i}(x)\|<\epsilon$, which follows that $|b|\in A$.
%	Hence, $|b|=\phi_{i}(x)$ for some $x\in [0,|a|]$, which follows that $|b|\in \bigcup_{i}\phi_i(E_i)$. Consequently, the linear vector space $\bigcup_{i}\phi_i(E_i)$ is a Riesz space and $A$ is a Banach lattice.
%	%The continuity of lattice operators implies that $|y|\in A$ whenever $y\in A$, i.e., the Banach space $A$ is in fact a Banach lattice.
%	
%	Suppose $B$ is a Banach lattice and $\psi_i:E_i\to B$ is an interval preserving contraction for each $i$ such that $\psi_j=\psi_{ji}\circ\psi_i$ if  $j\geq i$.  
%	Define 
%	$\psi:\bigcup_{i}\phi_i(E_i)\to B, \psi(\phi_i(a))=\psi_i(a).$
%	If $\phi_i(a)=\phi_j(b)$,
%	$\|\phi_{ki}(a)-\phi_{kj}(b)\|$ vanishes as $k\to\infty$.
%	Since $\|\psi_i(a)-\psi_j(b)\|=\|\psi_k(\phi_{ki}(a)-\phi_{kj}(b))\|\leq\|\phi_{ki}(a)-\phi_{kj}(b)\|$  whenever $k\geq i,j$, $\psi_i(a)=\psi_j(b)$, yielding that $\psi$ is a well defined linear operator. 
%	
%	If there exists some  $\phi_i(a)\in \bigcup_i\phi_i(E_i)$ and $\epsilon>0$ such that 
%	$\|\phi_i(a)\|\leq 1$ but $\|\psi_i(a)\|>1+\epsilon$, then $\|\phi_{ji}(a)\|<1+\epsilon$ for sufficiently large $j$ but
%	$\|\psi_j(\phi_{ji}(a))\|=\|\psi_i(a)\|>1+\epsilon$, a contradiction to that $\psi_j$ is a contraction. Therefore, $\psi$ is a contraction and thus admits an extention, which we also denote by $\psi$, on $A$.
%	
%	For any positive element $\phi_i(a)\in \bigcup_{i}\phi_i(E_i)$, $\phi_i(a)\in[0,\phi_i(|a|)]$. Hence for any $\epsilon>0$, there exists some $x\in [0,|a|]\cap A$ such that 
%	$\|\phi_i(x)-\phi_i(a)\|<\epsilon$. This yields that 
%	$\|\psi_i(a)-\psi_i(x)\|=\|\psi(\phi_i(a)-\phi_i(x))\|<\epsilon.$ Since $\psi_i(x)\geq 0$ and the positive cone of a normed Riesz space is closed,  $\psi_i(a)\geq 0$. So, $\psi$ is positive.
%\end{proof}
%%%


%\section{Main result}
%\begin{lemma}
%Suppose $F$ is an order ideal of a $\sigma$-Dedekind complete vector lattice $E$. Let $(f_n)$ be a order bounded sequence in $F$ and $f\in F$,  then $f_n$ is order convergent to $f$ in $F$ is equivalent to that in $E$.
%\end{lemma}

%\begin{lemma}
%Suppose $E$ is an order continuous \bfs over a measurable space $(X,\Sigma,\mu)$, then for every $\epsilon >0$ there exists $\delta>0$ such that $\mu(A)<\epsilon$ for any  measurable set $A$ satisfies $\mu(A)<\delta$. 
%\end{lemma}

%\begin{proof}
%Suppose $\epsilon >0$ for which such $\delta>0$ does  not exist. Then there exists $A_n\in \Sigma$,  for $n\in \mathbb{N}, $ satisfying $\mu(A_n)<2^{-n}$ and $\|\chi_{A_n}\|\geq \epsilon$. 
%Define $B_n=\bigcup_{k=n}^\infty A_n$ for $n\in \mathbb{N}$,($B_n\in E$???) then $\mu(B_n)$ vanishes as $\n\to\infty$  which implies  that $\chi_{B_n}$ converges to $0$ in order. By the order continuity of $E$, we have $\|\chi_{B_n}\|$ converges to $0$. However, $\chi_{B_n}\geq\chi_{A_n}\geq 0$  and so $\|\chi_{B_n}\|\geq \|\chi_{A_n}\|\geq \epsilon$ for all  $n$. This is a contradiction.
%\end{proof} 

\begin{theorem}
	Suppose $X$ be a Hausdorff space and $\mu$ is a $\sigma$-finite Radon measure on a $\sigma$-algebra $\Sigma$ of subsets of $X$ which contains all compact subsets of $X$.  Let $E$ be a Banach function space over $(\Sigma,\mu)$ that contains $C_c(X)$ as a subspace. If $E$ is order continuous, then $C_c(X)$ is dense in $E$.
\end{theorem}
\begin{proof}
It is well known that for any positive measurable function $f$, there is an increasing sequence $(s_n)$ of simple functions that converges to $f$ everywhere,  and moreover, we can require that each simple function has a support with finite measure if the measure space is $\sigma$-finite.
Since $E$ is order continuous, $\operatorname{span}\{\chi_S\in E:\mu(S)<\infty\}$ is dense in $E$. The inner regularity of $\mu$ follows that there is an increasing sequence $(K_n)$ of compact subsets of $S$ such that $\mu(K_n)\to \mu(S)$  as $n\to  \infty$. Therefore, $\chi_{K_n}$ converges to $\chi_S$, for each subset $S$ of $X$ satisfying $\chi_S\in E$ and $\mu(S)<\infty$, in order and thus in norm. So, $\operatorname{span}\{\chi_K\in E:K\mbox{ is a compact subset of } X\}$ is dense in $E$. By Urysohn's lemma, each $\chi_K$ where $K$ is a compact subset of $X$, is a limit almost everywhere, or equivalently, is an order limit, of a decreasing sequence in $C_c(X)$. As a result, $C_c(X)$ is a dense subspace of $E$.
\end{proof}
\begin{remark}
In fact, the above proof gives a conclusion that if $\operatorname{span}\{\chi_S\in E:\mu(S)<\infty\}$ is dense in the order continuous \bfs $E$, then $C_c(X)$ is dense in $E$ regardless of whether $\mu$ is $\sigma$-finite or not. Hence a \bfs like $L^p(1\leq p<\infty)$ space over a Borel measure space always contains $C_c(X)$ as a dense subspace. 
\end{remark}



% When $E$ is a Banach lattice, and $B$ is a projection band in $E$, then we let $P_B:E\to B$ denote the associated order projection.

% \begin{lemma}\label{res:localisation_of_order_continuity}
% 	Let $E$ be a Banach lattice. Take a collection $\desset{B_i:i\in I}$ of projection bands in $E$. The following are equivalent:
% 	\begin{enumerate}
% 		\item[\uppars{a}] $\bigcup_{i\in I}B_i$ is dense in $E$;
% 		\item[\uppars{b}] for every $x\in E$ and every $\veps>0$, there exists an $i\in I$ such that  $\norm{P_{B_i^\dc}x}<\veps$.
% 	\end{enumerate}
% When this is the case, then the following are equivalent:
% 	\begin{enumerate}
% 		\item $E$ is order continuous;
% 		\item $B_i$ is order continuous for every $i\in I$.
% 	\end{enumerate}
% \end{lemma}

% A collection of projections bands as in \cref{res:localisation_of_order_continuity} always exists. One can take $\{E\}$ for this, for example. 

% \begin{proof}
% 	We prove that (a) implies (b). Take $x\in E$ and $\veps>0$. Choose an $i\in I$ and a $y\in B_i$ such that $\norm{x-y}<\veps/2$. Then
% 	\begin{align*}
% 	\norm{P_{B_i^\dc}x}&=\norm{x-P_{B_i}x}\\
% 	&\leq \norm{x-y}+\norm{y-P_{B_i}x}\\
% 	&=\norm{x-y}+\norm{P_{B_i}(y-x)}\\
% 	&<\veps/2+\veps/2\\
% 	&=\veps.
% 	\end{align*}
	
% It is evident that (b) implies (a) because $\norm{x-P_{B_i}x}=\norm{P_{B_i^\dc}x}$ for all $x$ and $i$, 
	
% 	We prove that (1) implies (2). 	Since bands in a Banach lattice are closed, each $B_i$ is a Banach sublattice of $E$. As is well known (see \cite[Theorem~2.4.2]{meyer-nieberg_BANACH_LATTICES:1991}, for example), the order continuity of $E$ then implies that each $B_i$ is order continuous.
	
% 	We prove that (2) implies (1). Suppose that $E$ is not order continuous. Then, by  \cite[Theorem~2.4.2]{meyer-nieberg_BANACH_LATTICES:1991}, there exists an $x\in E$ and a disjoint sequence $(x_n)$ in $E$ such that $0\leq x_n\leq x$ for all $n$ and $\alpha\coloneqq \inf_n \norm{x_n}>0$. Choose an $i\in I$ such that $\norm{P_{B_i^\dc}x}<\alpha/2$. Then, for all $n$, we have $0\leq P_{B_i}x_n\leq P_{B_i}x$ and
% 	\[
% 	\norm{P_{B_i}x_n}\geq \norm{x_n}-\norm{P_{B_i^\dc}x_n}\geq\alpha-\norm{P_{B_i^\dc}x}>\alpha-\alpha/2=\alpha/2.
% 	\]
% 	Note that $(P_{B_i}x_n)$ is a disjoint sequence in $B_i$.
% 	Again by \cite[Theorem~2.4.2]{meyer-nieberg_BANACH_LATTICES:1991}, this shows that $B_i$ is not order continuous. This contradiction implies that $E$ is order continuous.
% \end{proof}

\begin{theorem}
	Suppose $X$ is a Hausdorff topological space, and  $\mu$ is a measure on a $\sigma$-algebra $\Sigma$ of subsets of $X$  that contains all compact subsets of $X$. Let $E$ be a Banach function space over $(\Sigma,\mu)$ such that $C_c(X)$ is contained in and dense in $E$. If every compact subset of $X$ is metrisable, then $E$ is order continuous.
\end{theorem}
\begin{proof}
	It is easy to verify $E_S$ is a band of $E_T$ if $S\subset T$, where \[E_S:=\{f\chi_S : f\in E\}\] for each subset $S$ of $X$.  
Since $f\in E_{\supp f}$ for every $f\in C_c(X)$, we have 
\[
E=\overline{C_c(X)}\subseteq \overline{\bigcup_{K \mbox{ compact }}E_K}\subseteq E.
\] 
In view of \cref{direct_limit_order_continuous}, we need only show that $E_K$ is order continuous for every compact subset $K$ of $X$. Take such a $K$. Since the band projection $f\mapsto \chi_K f$ from $E$ onto $E_K$ is continuous, the density of $C_c(X)$ in $E$ implies that $\{f\chi_K : f\in C_c(X)\}$ is dense in $E_K$. Hence $C(K)$ is dense in $E_K$. Since $K$ is metrisable, $C(K)$ is separable with respect to $\|\cdot\|_\infty$ (see, e.g., \cite[Theorem~26.15]{jameson_TOPOLOGY_AND_NORMED_SPACES:1974}). Since the positive inclusion map from the Banach lattice $(C(K), \|\cdot\|_\infty)$ into the Banach lattice $E_K$ is continuous, we see that $E_K$ is separable. Consequently, it cannot contain a closed subspace that is isomorphic to $\ell^\infty$. Since $L^0(X,\mu)$ is  $\sigma$-Dedekind complete, so is its order ideal $E_K$. It now follows from \cite[Corollary~2.4.3]{meyer-nieberg_BANACH_LATTICES:1991} that $E_K$  is order continuous, as required.
\end{proof}


%XXX We are too imprecise about $C_c(X)$ being included in and dense in $E$. The map $j$ that assigns to a continuous function its $\mu$-a.e. equivalence class need not be injective. This should be adapted, where the hypothesis is then that the image of $C_c(X)$ under $j$ is dense in $E$.



Suppose $\pi:X\to Y$ is a continuous map between topological spaces.
If $\mu_X$ is Borel measure on $X$, then $\mu_Y:=\mu\circ\pi^{-1}$ is a Borel measure on $Y$  and 
\begin{align*}
\pi_*: &L^0(Y,\mu_Y)\to L^0(X,\mu_X)\\
& g\mapsto g\circ\pi\end{align*}  
is an injective lattice homomorphism between Riesz spaces. 
Let $(E_X,\|\cdot\|_X)$ be a \bfs over $(X,\mu_X)$.  Define $$E_Y:=\pi_*^{-1}(E_X)$$ and $$\|\cdot\|_Y=\|\cdot\|_X\circ\pi_*,$$ then $(E_Y, \|\cdot\|_Y)$ is a \bfs over $(Y,\mu_Y)$ (In particular, $E_Y=L^1(Y,\mu_Y)$ if $E_X=L^1(X,\mu_X)$).
Only completion needs to be varified. Suppose $(g_n)$ is a Cauchy sequence in $E_Y$,  then $(f_n):=(\pi_*g_n)$ is a Cauchy sequence in $(E_X,\|\cdot\|_X)$ and hence convergent to some $f$ in $E_X$ with respect to $\|\cdot\|_X$. Suppose $(f_{k_n})$ is a subsequence of $(f_n)$ convergent to $f$ $\mu_X$-a.e., then $(g_{k_n})$ is convergent $\mu_Y$-a.e., say $g$ is the pointwise limit.  Since $$\{x\in X:(\pi_*g_{k_n})(x)\nrightarrow(\pi^* g)(x)\}\subset \pi^{-1}\{y\in  Y: g_{k_n}(y)\nrightarrow g(y)\},$$ $(\pi_*g_{k_n})$ is convergent to $\pi_*(g)$,  yielding $\pi(g)=f$. %$\pi_*g=\lim_{n\to\infty}\pi_*g_{k_n}=\lim_{n\to\infty}f_{k_n}=f$. 
Therefore,  $g\in E_Y$, $(g_n)$ is convergent to $g$ with respect to $\|\cdot\|_Y$ and  $(E_Y, \|\cdot\|_Y)$ is norm complete.
Consequently, $\pi_*:(E_Y,\|\cdot\|_Y)\to (E_X,\|\cdot\|_X)$ is a homomorphism between \bfss.
With this,  %we construct a functor that assigns to a continuous map $\pi:X\to Y$ between topological spaces to an isometric lattice homomorphism $\pi_*:(E_Y,\|\cdot\|_Y)\to (E_X,\|\cdot\|_X)$ between \bfss.
we induce an isometric lattice homomorphism $\pi_*:(E_Y,\|\cdot\|_Y)\to (E_X,\|\cdot\|_X)$ between \bfss from a continuous map $\pi:X\to Y$ between topological spaces.


\begin{lemma}\label{lemma1}
	Suppose that $E$ is a translation invariant \bfs over a locally compact Hausdorff topological group $G$ with  a Haar measure  $\mu$, that $C_c(G)$ is dense in $E$ and that maps $x\to \|\lambda_x\|$ and $x\to \|\rho_x\|$  are bounded on compact subsets of $G$. If $K$ is a compact subset of $G$ and $(K_n)$ is a sequence of compact subsets contained in $K$ such that $\lim_{n\to\infty}\mu(K_n)=0$, then $\lim_{n\to\infty}\|\chi_{K_n}\|=0$.
\end{lemma}

\begin{proof} 
	Let $\I{G}$ be a $\sigma$-compact clopen subgroup of $G$ that contains $K$(the existence of $\I{G}$ follows from \cite[Theorem A and Theorem B, Section 57]{Measure_Theory}), then $\I{G}$ is a locally compact Hausdorff topological group with a Haar measure $\I{\mu}=\mu|_{\I{G}}$ and $E':=\{f\chi_{\I{G}}|f\in E\}$ is a translation invariant \bfs over $\I{G}$ with the norm inherited from $E$. Obviously,  $x\to \|\lambda_x\|$ and $x\to \|\rho_x\|$  are bounded on compact subsets of $G'$. And one can also easily obtain $\I{E}=\I{E}_{s,0}$ if $E=E_{s,0}$,   where
	\begin{align*}E_{s,0}:=\{ & f\in E:\lim\limits_{x\to e}\|\lambda_x f-f\|=\lim\limits_{x\to e}\|\rho_x f-f\|=0,\\ &\text{ and }\forall \varepsilon>0 ~ \exists K\subset G \text{ compact s.t. }\|f\chi_{G\backslash K}\|<\varepsilon\}.
	\end{align*} From \cite[Theorem 5.4]{David}, we know that $C_c(G)$ is dense $E$ if and only if $E=E_{s,0}$.
	 It follows that $C_c(\I{G})$ is dense in $\I{E}$. 
	 Therefore, we can always assume $G$ is $\sigma$-compact.

For each $n\in \mathbb{N}$, let $O_n$ be an open subset of $G$ such that $K_n\subset O_n$ and $\mu(O_n)<\mu(K_n)+2^{-n}$. Since $K_n$ is compact and $O_n$ is open, the continuity of group multiplication implies there is an open neighbourhood $U_n$ of the identity $e$ of $G$ such that $K_nU_n\subset O_n$. By \cite{Measure_Theory}*{Theorem 8.7, Chapter II},    there exists a compact normal subgroup $H$ of $G$ such that $H\subset \cap_{n=1}^\infty U_n$ and $\quotient{G}:=G/H$ is metrizable. 

Let $\pi:G\to\quotient{G}$ be the quotient map, $(\quotient{E},\quotient{\|\cdot\|})$ be the induced \bfs over $(\quotient{G},\quotient{\mu})$ and $\pi_*:\quotient{E}\to E$ be the induced isometric lattice homomorphism. By \cite{Measure_Theory}*{Theorem C, Section 63}, $\mu^*=\mu\circ\pi$ is a Haar measure on $\quotient{G}$. 
A direct computaion yields, for any $x\in G$ and $g\in \quotient{G}$, that  
$\lambda_x(\pi_*g)=\pi_*(\lambda_{\pi(x)}g)$ and
$\rho_x( \pi_*g)=\pi_*(\rho_{\pi(x)}g)$,
from  which it follows that $\quotient{E}$ is also translation invariant. Since  
$$\quotient{\|\lambda_{\pi(x)} g\|}=\|\pi_*(\lambda_{\pi(x)}g)\|=\|\lambda_x(\pi_*g)\|\leq\|\lambda_x\|\|\pi_*g\|=\|\lambda_x\|\quotient{\|g\|},$$
$\pi(x)\mapsto\|\lambda_{\pi(x)}\|$ is bounded on compact subsets of $\quotient{G}$,  noting that every compact subset $L$ of $\quotient{G}$ is the image under $\pi$ of a compact    subset $HL$ of $G$. Similarly, $\pi(x)\mapsto\|\rho_{\pi(x)}\|$ is bounded on compact subsets of $\quotient{G}$.
   Given $f\in \quotient{E}$ and $\varepsilon>0$. Since $\pi_*f\in E=E_{s,0}$, there exists a compact subset $K$ of $G$ and an open neighbourhood $U$ of the identity such that $\|(\pi_*f)\chi_{G\backslash K}\|<\varepsilon$, $\|\lambda_x(\pi_*f)-\pi_*f\|<\varepsilon$ and $\|(\rho_x\pi_*f)-\pi_*f\|<\varepsilon$ whenever $x\in U$. Hence $$\quotient{\|f\chi_{\quotient{G}\backslash \pi(K)}\|}=\|\pi_*(f\chi_{\quotient{G}\backslash \pi(K)})\|=\|f\chi_{G\backslash \pi^{-1}(\pi(K))}\|\leq\|f\chi_{G\backslash K}\|< \varepsilon,$$ $$\quotient{\|\lambda_{\pi(x)}f-f\|}=\|\pi_*(\lambda_{\pi(x)}f)-\pi_*f\|=\|\lambda_x(\pi_*f)-\pi_*f\|<\varepsilon$$ and $$\quotient{\|\rho_{\pi(x)}f-f\|}=\|\pi_*(\rho_{\pi(x)}f)-\pi_*f\|=\|\rho_x(\pi_*f)-\pi_*f\|<\varepsilon$$
   whenever $\pi(x)$ is in the open neighborhood $\pi(U)$ of the identity $\pi(e)$ of $\quotient{G}$. That is,
$\quotient{E}=\quotient{E}_{s,0}$, i.e., $C_c(X)$ is dense in $\quotient{E}$. 
 From the metric case, we know that $\quotient{E}$ is order continuous.

Since $\mu(K_nH)\leq\mu(O_n)$, $(\mu(K_nH))$ converges to $0$, i.e., $(\pi_*\chi_{\pi(K)})=(\chi_{K_nH})$ converges to $0$ $\mu$-a.e., resulting in $(\chi_{\pi(K_n)})$ converges to $0$ $\mu^*$-a.e.. Because $\chi_{\pi(K_n)}\leq\chi_{\pi(K)}$, $(\chi_{\pi(K_n)})$ converges to $0$ in $\quotient{E}$  in order and thus in norm. Consequently, $(\chi_{K_nH})=(\pi_*\chi_{\pi(K)})$ converges to $0$ in norm. It follows from $\chi_{K_n}\leq \chi_{K_nH}$  that $(\chi_{K_n})$ converges to $0$  in norm.
\end{proof}

\begin{lemma}\label{lemma2}
	Suppose $K$ is a compact Hausdorff space with a Borel measure $\mu$ and $E$ is a \bfs over $(K,\mu)$ that contains $C(K)$ as a dense subspace and is included in $ L^1(K,\mu)$ continuously. If for each sequence $(K_n)$ of compact subsets of $K$ satisfying $\lim_{n\to\infty}\mu(K_n)=0$ we have $\lim_{n\to\infty}\|\chi_{K_n}\|=0$, then $E$ is order continuous.
\end{lemma}
\begin{proof} 
	Let $(f_n)$ be a sequence in $E$ dominated by $\chi_K$ that decreases to $0$ in order  and $(g_n)$ be a sequence in $C(K)$ such that $(f_n-g_n)$ converges to $0$ in the norm of $E$. Since $E$ is continuously included into $L^1(K,\mu)$, $(f_n-g_n)$ converges to $0$ in the norm of $L^1(K,\mu)$ too.  Since $E$ is an order ideal of $L^1(\mu)$, $(f_n)$ also decreases to $0$ in order in $L^1(K,\mu)$.  Hence $(f_n)$ converges to $0$ in the norm of $L^1(K,\mu)$ by the order continuity of $L^1(K,\mu)$. 

Given $\varepsilon>0$, define the compact sets $K_n=g^{-1}[\varepsilon,\infty)$ for $n\in \mathbb{N}$. Since $$\|g_n\|_{L^1(K,\mu)}\geq \int_{K_n} g_n\dc\mu\geq \varepsilon \mu(K_n),$$ it follows that $\lim_{n\to \infty}\mu(K_n)=0$ and thus that $\lim_{n\to\infty}\|\chi_{K_n}\|\to 0$. Furthermore, since $$0\leq g\leq \chi_{K_n}+\varepsilon \chi_{K\backslash K_n},$$ $\|g_n\|\leq \|\chi_{K_n}\|+\varepsilon \|\chi_K\|<(1+\|\chi_K\|)\varepsilon$ for $n$ sufficiently large. Therefore, $(g_n)$ and $(f_n)$ converges to $0$ in the norm of $E$. Hence $\chi_K\in E_a$, the absolutely continuous part of $E$. Noting that $E_a$ is norm closed in $E$, $E=\overline{C(K)}\subset E_a$, i.e., $E$ is order continuous.
\end{proof}


\begin{theorem}
 Suppose $E$ is a translation invariant \bfs on a locally compact Hausdorff topological group $G$, and that the maps $x\mapsto \|\lambda_x\|$ and $x\mapsto\|\rho_x\|$ are both  bounded on compact subsets of $G$. If $C_c(G)$ is a dense subspace of $E$, then $E$ is order continuous.	
\end{theorem}

\begin{proof} It is sufficient to show that  $E_K=\{f\chi_K |f\in E\}$ is order continuous for each compact subset $K$ of $G$. By \cite{David}*{Lemma 3.14}, $E_K\subset L^1(K,\mu)$ with continuous inclusion. Let $(K_n)$ be a sequence of compact subsets of $K$ such that $\mu(K_n)\to 0$ as $n\to \infty$, then $\|\chi_{K_n}\|\to 0$ as $n\to \infty$ by \cref{lemma1}. Our proof ends by applying of \cref{lemma2}.
\end{proof}



%\begin{lemma}
	%Suppose $G$ is a \fg, $\mu$ is a Borel measure on $G$  and $E$ is a \bfs over $\mu$. If $C(G)$ is dense in $E$ and $E$ is continuously embedded in $L^1(G,\mu)$, then $E$ is order continuous.
%\end{lemma}
%We split the proof in 3 steps
	%Step 1. If $(K_n)$ is a sequence of compact subsets of $G$ such that $\lim_{n\to\infty}\mu(K_n)=0$, then $\lim_{n\to\infty}\|\chi_{K_n}\|=0$.
	%
	%For each $n\in \mathbb{N}$, let $O_n$ be an open subset of $X$ such that $K_n\subset O_n$ and $\mu(O_n)<\mu(K_n)+2^{-n}$. Since $K_n$ is compact and $O_n$ is open, the continuity of product implies there is an open neighbourhood $U_n$ of the neuture element of  $G$ such that $K_n+U_n\subset O_n$. By Theorem 8.7, Chapter II, [], there exists a compact subgroup $H$ of $G$ such that $H\subset \cap_{n=1}^\infty U_n$ and $\quotient{G}=G/H$ is metrizable. 
	%
	%Let $\quotient{\pi}:\quotient{E}\to E$ be the isometric order continuous lattice homomorphism that the above functor assigns to the quotient map $\pi:G\to\quotient{G}$. Let $g\in \quotient{G}$, then $\quotient{\pi}g\in E$. Given $\varepsilon>0$, since $C(G)$ is dense in $L^1(G,\mu)$, there exists some $f'\in C(G)$ such that $\|\pi^*g-f'\|_1\leq \varepsilon$. By [], there is a continuous function $f$ on $\quotient{G}$ such that 
	% $i^*f(x)=\int_H f(x+y)\dc m(y) ~ (x\in \quotient{G})$, where $m$ is the normalized Haar measure on $H$.
	%\begin{align*}
	%\|g-f\|& =\|\pi^*g-\pi^*f\|\\
	%&=\|\int_Y g(\pi(\cdot))-f'(\cdot +  y)\dc m(y)\|\\
	%&=\|\int_Y g(\pi(\cdot+y))-f(\cdot+y)\dc m(y)\|	
	%\end{align*}
	%
	
%before is order continuous.
%	
%	Since $\quotient{\mu}(\pi(K_n))=\mu(\pi^{-1}\pi(K_n))=\mu(K_nH)\leq \mu(O_n)\to 0$ as $n\to\infty$ and $0\leq\chi_{\pi(K_n)}\leq \chi_{\pi(K)}\in \quotient{E}$, $\quotient{\|\chi_{\pi(K)}\|}\to 0$ as $n\to \infty$. Noting that $\chi_{K_n}\leq \chi_{K_nH}=\chi_{\pi^{-1}\pi(K_n)}=\pi_*\chi_{\pi(K_n)}$, $\|\chi_{K_n}\|\leq \|\pi_*\chi_{\pi(K_n)}\|=\quotient{\|\chi_{\pi(K_n)}\|} \to 0$ as $n\to \infty$.
%
%\begin{lemma}
%	Suppose $X$ is a compact Hausdorff space and $E$ be a \bfs  over a Borel measure $\mu$ on $X$.  If $C(X)$ is dense in $E$ and $E$ is continuously embedded in $L^1(X,\mu)$, then $E$ is order continuous.
%\end{lemma}
%\begin{proof}
%	By \cite[Theorem~4.1~and~Theorem~4.2]{Huang2017}, there exists a continuous injection $i$ from $X$ to a \fg $G$ such that for any continuous map $h$ from $X$ to any \fg $G'$ there exists a unique group homomorphism $h':G\to G'$  such that $h=h'\circ i$. In particular, every $f$ in $C(X)$ there exists $g$ in $C(G)$ such that $f=i^*g$. Hence $C(G)$ is dense in $i^*E$.
	
	%Given $f\in F$ ($F$ is $E$ or $L^1(X)$), let $g(y)=f(x)$ if $y=i(x)$ for some $x\in X$ and $0$ elsewhere. Since $i(X)$ is closed in $G$, $g$ is a Borel measurable function on $G$. Noting that $f=i^*g$, $g$ is an element of $E(G)$. That is, $i^*$ is a isometric lattice isomorphism. Hence $i^*E$ is continuously embeded into $L^1(G)=i*L^1(X)$. Together with the above,  we have $i^*E$ is order continuous, so is $E$.
%\end{proof}


%%%%%%%%%%%%%%%%%%%%%%%%%%%%%%%%% ACKNOWLEDGEMENTS %%%%%%%%%%%%%%%%%%%%%%%%%%%%%%%%%%%%%

\subsection*{Acknowledgements} TEXT OF THE ACKNOWLEDGEMENTS.

%%%%%%%%%%%%%%%%%%%%%%%%%%%%%%%% END OF THE ACTUAL TEXT %%%%%%%%%%%%%%%%%%%%%%

%%%%%%%%%%%%%%%%%%%%%%%%%%%%%%%% BIBLIOGRAPHY %%%%%%%%%%%%%%%%%%%%%%%%%%%%%%%%%%%%%%%%%%

% When using amsrefs, no \bibliographystyle command is necessary.


% Undo the effect of \btfs in the bibfile by including the following
% \renewcommand{\btfs}{\mathrm}

% Avoid tt font for url's
\urlstyle{same}

\bibliography{general_bibliography}

%%%%%%%%%%%%%%%%%%%%%%%%%%%%%%%%%%%%%%%%%%%%%%%%%%%%%%%%%%%%%%%%%%%%%%%%%%%%%%%%%%%%%%%%

\end{document}
